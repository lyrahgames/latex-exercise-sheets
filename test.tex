\documentclass[a4paper,fleqn, 11pt]{article}

\usepackage[utf8]{inputenc}
\usepackage[T1]{fontenc}
\usepackage[ngerman]{babel}
\usepackage[bottom=25mm,left=30mm,right=30mm,bottom=35mm]{geometry}
\usepackage{times}
\linespread{1.15}

\usepackage{turnthepage}
\renewcommand{\turnthepage}{\it bitte wenden}



\usepackage{import}

\usepackage{multicol}

\usepackage{float}
\usepackage{import}
\usepackage{graphicx}

\usepackage[footnotesize,hang]{caption}


\usepackage[pgfkeys,custom]{ati}
% \SetupExSheets{solution/print=true, question/type=exam}
% \usepackage{utilities}
% \usepackage{uniinput}
\allowdisplaybreaks

\begin{document}
  \[
    \set{1}{}
    \separate
    \set{2}{3}
  \]
  \[
    \define
    \separate
    \reverseDefine
    \separate
    \demand
    \separate
    \function{f}{X}{Y}
  \]
  \[
    \setReal
    \separate
    \setNatural
    \separate
    \setInteger
    \separate
    \setRational
  \]
  \[
    \roundBrackets{a}
    \separate
    \curlyBrackets{b}
    \separate
    \boxBrackets{c}
    \separate
    \angleBrackets{d}
    \separate
    \floorBrackets{e}
    \separate
    \ceilBrackets{f}
  \]
  \[
    \absolute{a}
    \separate
    \norm{b}
    \separate
    \inverse{c}
  \]
  \[
    \dotProduct{a}{b}
    \separate
    \scalarProduct{c}{d}
    \separate
    \crossProduct{e}{f}
  \]
  \[
    \timeDerivative{a}
    \separate
    \timeSecondDerivative{b}
    \separate
    \leibnizDerivative{y(t)}{t}
    \separate
    \leibnizDerivative{}{t}y(t)
  \]
  \[
    a\appendUnit{m}
    \separate
    b\appendUnit{km\cdot s^{-2}}
    \separate
    c\appendUnit{\frac{m}{s^2}}
  \]

  % \begin{multicols}{2}
    \begin{atiTask}[
  title = Räuber und Beute
]
  Betrachten Sie die folgenden gekoppelten Differentialgleichungen mit den Koeffizienten $\alpha,\beta,\gamma,\delta\in\setReal^+$.
  \begin{align*}
    \timeDerivative{H} &= \alpha H - \beta HF \\
    \timeDerivative{F} &= -\gamma F + \delta HF
  \end{align*}
  \begin{atiSubtasks}
    \item{
      Interpretieren Sie kurz, was die einzelnen Terme im Kontext eines einfachen Räuber-Beute-Modells bedeuten.
      Erweitern Sie die Differentialgleichungen der Räuber, so dass ein konstanter Abschluss (etwa durch Jäger) berücksichtigt wird.
    }
    \item{
      Leiten Sie aus dem gegebenem Differentialgleichungssystem (ohne Jäger) eine neue Differentialgleichung für $F(H)$ her.
      Bringen Sie diese in die bekannte Form einer exakten Differentialgleichung.
    }
    \item{
      Zeigen Sie, dass $\Lambda = \frac{1}{FH}$ ein integrierender Faktor ist und lösen Sie die gefundene Differentialgleichung nach der Ihnen bekannten Methode.
    }
    \item{
      Bestimmen Sie die Gleichgewichtspunkte, an denen die folgende Gleichung gilt, ohne die Verwendung der impliziten Lösung.
      \[
        \timeDerivative{F} + \timeDerivative{H} = 0
      \]
    }
  \end{atiSubtasks}
\end{atiTask}
    \begin{atiTask}[
  title = Spieglein, Spieglein...
]
  Bestimmen Sie die Form des Spiegels, der parallel einfallende Strahlen in den Punkt $O$ reflektiert.

  \begin{atiNote}
    Wählen Sie den Ursprung im Punkt $O$.
    Es gilt die folgende Gleichung.
    \[
      \tan 2\vartheta = \frac{2\tan \vartheta}{1-\tan^2 \vartheta}
    \]
    Benutzen Sie die Substitution $y^2 = r^2 - x^2$.
  \end{atiNote}
\end{atiTask}
    \begin{atiTask}[
  title = Quadratisches Reibungsgesetz
]
  Betrachten Sie die folgende Differentialgleichung mit den Koeffizienten $m,\gamma, k\in\setR^+$.
  \[
    m\ddot{y} \pm \gamma (\dot{y})^2 + ky = 0
  \]
  Das Vorzeichen vor dem quadratischen Reibungsterm wirke immer so, dass die Reibung die Bewegung behindert.
  Diese nichtlineare Differentialgleichung kann gelöst werden, wenn man sich zunutze macht, dass die unabhängige Variable $t$ nicht vorkommt.
  Substituieren Sie $p=\dot{y}$, um auch die Ableitungen nach $t$ zu eliminieren.
  Vergleichen Sie Ihr Ergebnis mit der Bernoulli-Gleichung und substituieren Sie angemessen.
  Bestimmen Sie die Lösung für die Anfangsbedingungen
  \[
    \dot{y}(0) = 0
  \]
  und $y>0$ für kleine $t$ (so kleine Zeiten, dass sich das Vorzeichen im Reibungsterm nicht umkehrt).
  Es soll hier genügen $\dot{y}$ zu bestimmen.
\end{atiTask}
    \begin{atiTask}[
  title = Lösungsmethode der inhomogenen Differentialgleichung 2.Ordnung
]
  Lösen Sie die folgende Differentialgleichung, indem Sie diese auf zwei Differentialgleichungen erster Ordnung zurückführen.
  \[
    y'' - 5y' + 6y = e^{2x} \cos 3x
  \]
\end{atiTask}
    \begin{atiTask}[
  title = Magnetfeld eines Leiters
]
  Gegeben sei das folgende Vektorfeld.
  \[
    \function{F}{\setReal^2\setminus\{(0,0)\}}{\setReal^2}
    \separate
    F(x,y)\define \frac{-y\vectorX}{x^2+y^2} + \frac{x\vectorY}{x^2+y^2}
  \]
  \begin{atiSubtasks}
    \item{
      Prüfen Sie, ob die Integrationsbedingungen erfüllt sind.
    }
    \item{
      Berechnen Sie das Kurvenintegral für die beiden folgenden Integrationswege und skizzieren Sie die zugehörigen Kurven.
      \begin{atiSubequations}
        \item{
          \function{r}{[0,\pi]}{\setReal^2}
          \separate
          r(\varphi) \define \vectorX \cos\varphi + \vectorY \sin\varphi
        }
        \item{
          \function{r}{[0,\pi]}{\setReal^2}
          \separate
          r(\varphi) \define \vectorX \cos\varphi - \vectorY \sin\varphi
        }
      \end{atiSubequations}
    }
    \item{
      Vergleichen Sie die Integrationswege und finden Sie heraus, ob das Vektorfeld konservativ ist.
    }
  \end{atiSubtasks}
\end{atiTask}
    \begin{atiTask}[
  title = Aufstellen von Differentialgleichungen
]
  \begin{atiSubtasks}
    \item{
      Eine Bakterienpopulation zeige exponentielles Wachstum, das heißt die Anzahl der Bakterien erhöht sich proportional zur Anzahl der vorhandenen Bakterien mit der Rate $r$.
      Zusätzlich werden der Bakterienkultur kontinuierlich $k$ Bakterien pro Zeiteinheit zugefügt.
      Allerdings seien die Bakterien empfindlich gegen Lichteinfall.
      Sie sterben deshalb proportional zu Anzahl der vorhandenen Bakterien mit der Rate $1+\sin t$, welche den Rhythmus der Tageszeiten simuliert.
      Stellen Sie die Differentialgleichung für die Anzahl $N$ der Bakterien auf.
    }
    \item{
      Ein Zylinder von Grundkreisradius $r$ und Masse $m$ schwimmt mit vertikaler Achsenlage im Wasser.
      Seine Eintauchtiefe sei $l$.
      Gesucht ist die Periode der Schwingung, die sich ergibt, wenn man den Zylinder ein Wenig in das Wasser eintaucht und danach loslässt.
      Der Bewegungswiderstand sei angenähert gleich Null anzunehmen.
      Wählen Sie die $y$-Achse vertikal nach unten mit dem Nullpunkt auf der Wasseroberfläche.
    }
  \end{atiSubtasks}
\end{atiTask}
    \begin{atiTask}[
  title = Konservative Vektorfelder
]
  \begin{atiSubtasks}
    \item{
      Überprüfen Sie, welche der folgenden Vektorfelder $\function{F}{\setReal^3}{\setReal^3}$ konservativ sind.
      \begin{atiSubequations}
        \item{
          F(x,y,z) \define q \roundBrackets{\crossProduct{v}{B}}
        }
        \item{
          F(x,y,z) \define 2y^2 z^3 \vectorX + 4xyz^3 \vectorY + 6xy^2z^2 \vectorZ
        }
        \item{
          F(x,y,z) \define 2(y+x)\vectorX + 2x\vectorY
        }
        \item{
          F(x,y,z) \define x^2 \cos y \vectorX + 2x\sin y \vectorY + z^2\vectorZ
        }
      \end{atiSubequations}
    }
    \item{
      Berechnen Sie das Wegintegral für das folgende Vektorfeld $F$ und den Weg $C = C_1 + C_2$.
      \[
        \function{F}{\setReal^2}{\setReal^2}
        \separate
        F(x,y) = 2(y+x)\vectorX + 2x\vectorY
      \]
      $C_1$ verläuft vom Punkt $(0,0)$ zum Punkt $(1,1)$ und erfüllt $y=x^2$.
      $C_2$ verläuft vom Punkt $(1,1)$ zum Punkt $(0,0)$ und erfüllt $y=x^4$.
      Vergleichen Sie das Ergebnis mit Ihren Erwartungen und begründen Sie es.
    }
  \end{atiSubtasks}
\end{atiTask}
    \begin{atiTask}[
  title = Parametrisierung gegeben, Kurve gesucht
]
  Skizzieren Sie die Kurven $xy$-Ebene, die durch die folgenden Parametrisierungen gegeben werden.
  \begin{atiSubequations}
    \item{
      \function{r}{[-1,1]}{\setReal^3}
      \separate
      r(s) \define \frac{1}{2}\boxBrackets{(1-s)\vectorX + (-7+3s)\vectorY}
    }
    \item{
      \function{r}{\boxBrackets{\frac{\pi}{2},\frac{\pi}{2}}}{\setReal^3}
      \separate
      r(t) \define 2\cos t \vectorX + 2\sin t \vectorY
    }
    \item{
      \function{r}{[0,2\pi]}{\setReal^3}
      \separate
      r(t) \define t\vectorX + \roundBrackets{\frac{1}{2}\cos t + \frac{3}{2}}\vectorY
    }
    \item{
      \function{r}{[1,2\pi]}{\setReal^3}
      \separate
      r(t) \define t\vectorX + \frac{7t^2 - 2 - 20\pi^2}{4\pi^2 - 1}\vectorY
    }
  \end{atiSubequations}
\end{atiTask}
    \begin{atiTask}[
  title = {Kurve gegeben, Parametrisierung gesucht}
]
  Geben Sie für jede der nachfolgend genannten Kurven eine Parametrisierung an.
  \begin{atiSubtasks}
    \item{
      Die Verbindungsstrecke vom Punkt $P_1 \define (1,1)$ zum Punkt $P_2\define (2,5)$.
    }
    \item{
      Die obere Halbellipse mit Halbachsen $a$ und $b$.
    }
    \item{
      Die Schnittkurve des Zylinders $x^2 + y^2 = 4$ mit dem Paraboloid $z = x^2 + y^2$.
    }
    \item{
      Die Schnittkurve der Ebene $x+y=1$ mit dem Kegel $z^2 = x^2 + y^2$.
    }
  \end{atiSubtasks}
\end{atiTask}
    \begin{atiTask}[
  title = Wegintegrale berechnen
]
  Berechnen Sie das Kurvenintegral für das Vektorfeld $F$ und die im Folgenden gegebenen Kurven.
  \[
    \function{F}{\setR^2}{\setR^2}
    \separate
    F(x,y)\define (x^2+y^2)\vec{i} + 4xy\vec{j}
  \]
  \begin{atiSubequations}
    \item{
      2y = x^2
    }
    \item{
      y = x
    }
    \item{
      test
    }
  \end{atiSubequations}
\end{atiTask}
    \begin{atiTask}[
  title = Wie ein Fisch im Wasser
]
  Die Temperaturverteilung in einem See sei gegeben durch die folgende Funktion.
  \[
    D \define \left\{ (x,y,z)\in\setR^3 \middle| z < 0 \right\}
  \]
  \[
    \function{T}{D}{\setR}
    \separate
    T(x,y,z)\define -\curvb{x^2 + \frac{y^2}{4} + 2z^2}
  \]
  \begin{atiSubtasks}
    \item{
      Bestimmen Sie die Isothermen und fertigen Sie eine Skizze dieser in der $yz$-Ebene an.
    }
    \item{
      Ein Fisch im Wasser befinde sich am Punkt $(1,2,-1)$.
      In welche Richtung verändert sich die Temperatur am stärksten?
      Wird es wärmer oder kälter?
    }
    \item{
      Der Fisch bewege sich auf dem folgenden Weg.
      \[
        \function{r}{[0,2\pi]}{\setR^3}
        \separate
        r(t)\define 2\cos t \vec{i} + \sin t \vec{j} + (\cos t - 2)\vec{k}
      \]
      Wann ist die Temperatur für ihn am größten und wann am kleinsten?
      Wo befindet er sich zu diesem Zeitpunkt?
      Skizzieren Sie die Temperatur $T\circ r$, die der Fisch entlang seines Weges erfährt.
    }
  \end{atiSubtasks}
\end{atiTask}
    \begin{atiTask}[
	title = Ähnlichkeitsdifferentialgleichung
]
	Gegeben sei eine gewöhnliche nicht-separable Differentialgleichung mit der freien Variable $t$ und der folgenden Form.
	Beachten Sie, dass $\timeDerivative{y}=\leibnizDerivative{y(t)}{t}$ die erste Ableitung nach der Zeit beschreibt.
	\[
		t\timeDerivative{y} = y\roundBrackets{1+\ln y - \ln t}
	\]

	Lösen Sie diese Differentialgleichung, indem Sie sie durch die folgende Substitution in eine separable Differentialgleichung überführen und überprüfen Sie Ihr Ergebnis, indem Sie eine Probe durchführen.
	\[
		z(t)\define\frac{y(t)}{t}\separate t\in\setReal^+
	\]
\end{atiTask}
\begin{atiSolution}
	Durch die Verwendung der Substitution lassen sich die folgenden Aussagen für alle $t\in\setReal^+$ treffen.
	\[
		y(t) = tz(t) \implies y'(t) = z(t) + tz'(t)\atiPoints[1]
	\]
	Das Umformen der ursprünglichen Differentialgleichung und Einsetzen der Substitution führt dann zur gewünschten separablen Differentialgleichung, die sich durch die Methode der Trennung der Variablen lösen lässt.
	\[
		ty'(t) = y(t)\boxBrackets{1+\ln \roundBrackets{\frac{y(t)}{t}}} \implies t\boxBrackets{z(t) + tz'(t)} = tz(t)\boxBrackets{1+\ln z(t)}
	\]
	\[
		\implies tz'(t) = z(t)\ln z(t) \implies \frac{z'(t)}{z(t)\ln z(t)} = \frac{1}{t}\atiPoints[1]
	\]
	\[
		\implies \integral{t_0}{t}{\frac{z'(s)}{z(s)\ln z(s)}}{s} = \integral{z_0}{z(t)}{\frac{1}{s\ln s}}{s} = \integral{t_0}{t}{\frac{1}{s}}{s}
	\]
	\[
		\implies \ln\roundBrackets{\frac{\ln z(t)}{\ln z_0}} = \ln \roundBrackets{\frac{t}{t_0}} \implies z(t) = \exp\roundBrackets{\frac{t\ln z_0 }{t_0}} = z_0^{\frac{t}{t_0}}\atiPoints[2]
	\]
	\[
		\implies y(t) = tz(t) = t\exp\roundBrackets{\frac{t\ln z_0 }{t_0}}\atiPoints[1]
	\]
	Für die Probe leitet man nun die Lösung ab und substituiert den erhaltenen Term mithilfe der berechneten Lösung.
	\[
		y'(t) = \exp\roundBrackets{\frac{t\ln z_0}{t_0}} + \frac{t\ln z_0}{t_0}\exp\roundBrackets{\frac{t\ln z_0}{t_0}}\atiPoints[\frac{1}{2}]
	\]
	\[
		\implies y'(t) = \frac{y(t)}{t} + \ln\roundBrackets{\frac{y(t)}{t}}\frac{y(t)}{t} = \frac{y(t)}{t}\boxBrackets{1 + \ln\roundBrackets{\frac{y(t)}{t}}}
	\]
	\[
		\implies ty'(t) = y(t)\boxBrackets{1+\ln\roundBrackets{\frac{y(t)}{t}}} = y(t)\boxBrackets{1 + \ln y(t) - \ln t}\atiPoints[\frac{1}{2}]
	\]
\end{atiSolution}
    \import{database/mathematische_methoden_der_physik_i/ordinary_deq/separable_deq/}{task-eine_zombieapokalypse.tex}
    \begin{atiTask}[
	topic = Gewöhnliche Differentialgleichungen,
	subtopic = Separable Differentialgleichungen,
	title = Homogene Differentialgleichungen,
	language = Deutsch
]
	Eine Funktion von zwei Variablen heißt homogen vom Grad $k$, wenn für einen beliebigen Parameter $\lambda$ das Folgende gilt.
	\[
		f(\lambda x, \lambda y) = \lambda^k f(x,y)
	\]
	Dementsprechend nennt man eine Differentialgleichungen der folgenden Form auch homogen, wenn $f$ und $g$ homogene Funktionen vom gleichen Grad sind.
	\[
		y' = \frac{f(x,y)}{g(x,y)}
	\]

	\begin{atiSubtasks}
		\item{
		\label{subtask:homogene-differentialgleichung-a}
			Lösen Sie die folgende homogene Differentialgleichung.
			\[
				2xyy^\prime = 3y^2 - x^2
			\]
			Anleitung: Führen Sie eine neue Variable $z(x)$ gemäß $y(x)\reverseDefine x\cdot z(x)$ ein und behandeln Sie die für $z(x)$ entstehende Differentialgleichung mit der Methode der Trennung der Variablen.
		}
		\item{
			Lösen Sie die folgende Differentialgleichung, die nicht homogen ist.
			\[
				y' = \frac{y+x-2}{y-x+4}
			\]
			Schuld daran sind die beiden additiven Konstanten in Zähler und Nenner des Bruches auf der rechten Seite.
			Gehen Sie in zwei Schritten nach folgender Anleitung vor.
			\begin{atiItems}
				\item Führen Sie die neue Variablen $v\define y-y_0$ und $u\define x-x_0$ ein und bestimmen Sie $x_0$ und $y_0$, sodass die neue Differentialgleichung in den Variablen $u$ und $v$ homogen ist (Gleichungssystem mit zwei Unbekannten).

				\item Verfahren Sie mit der Substitution $v(u)\reverseDefine u\cdot z(u)$ weiter, wie in Teilaufgabe \ref{subtask:homogene-differentialgleichung-a}.
			\end{atiItems}
		}
		\item{
			Machen Sie in beiden Fällen die Probe durch Einsetzen Ihrer Lösung $y(x)$ in die ursprüngliche Differentialgleichung.
		}
	\end{atiSubtasks}
\end{atiTask}
    \begin{atiTask}[
	title = Klassifikation von gewöhnlichen Differentialgleichungen,
	language = Deutsch,
]
	Klassifizieren Sie die folgenden gewöhnlichen Differentialgleichungen durch ihre Ordnung, Homogenität, Linearität und Separabilität.
	Lösen Sie zudem die separablen Differentialgleichungen und führen Sie die Probe durch Einsetzen ihrer Lösung in die ursprüngliche Differentialgleichung durch.
	\begin{atiSubequations}
	\begin{multicols}{2}
		% \begin{minipage}{0.5\textwidth}
		\item{
			\nfdrv[2]{y(x)}{x} + 2\fdrv{y(x)}{x} = y
		}
		\item{
			\fdrv{y(x)}{x} + \sin y - x^2 = 0
		}
		\item{
			y' + \tan(x)\cdot y = 0
		}
		% \end{minipage}
		% \begin{minipage}{0.5\textwidth}
		\item{
			\frac{y''}{y'} + x = 0
		}
		\item{
			yy'-x = 0
		}
		\item{
			\frac{x+1}{y+2} = \fdrv{y(x)}{x}
		}
		% \end{minipage}
	\end{multicols}
		\item{
			\sqrt{y^2 + 3a^2 + ya\curvb{2-\frac{4a}{2y}}} + \fdrv{y(x)}{x}\sqrt{x^2-a^2} = \sqrt{x+a}
		}
	\end{atiSubequations}
\end{atiTask}
    \begin{atiTask}[
	topic = Gewöhnliche Differentialgleichungen,
	subtopic = Separable Differentialgleichungen,
	title = Die Methode der Variablentrennung,
	language = deutsch,
]
	Lösen Sie die folgenden Differentialgleichungen durch Trennung der Variablen und bestimmen Sie gegebenenfalls die Integrationskonstante, sodass die nebenstehenden Anfangsbedingungen erfüllt sind.
	\begin{atiSubequations}
		\item{
			\label{dgl-1}
			y' = \frac{x e^{-y}}{x^2 + 1} \separate y(1) = 0
		}
		\item{
			\label{dgl-2}
			xyy' = \frac{x^2 + 2}{y-1}
		}
		\item{
			\label{dgl-3}
			y' = \frac{x+y}{x+y+2} \separate y(1) = -1
		}
	\end{atiSubequations}
	Machen Sie in allen Fällen die Probe durch Einsetzen Ihrer Lösung in die ursprüngliche Differentialgleichung. Dies ist auch dann verlangt, wenn die Lösung nur in impliziter Form angebbar ist.

	\begin{atiNote}
		Führen Sie in Teilaufgabe \ref{dgl-3} die neue Variable $z(x)\define x + y(x)$ ein.
	\end{atiNote}
\end{atiTask}

\begin{atiSolution}
	\begin{atiSubtaskSolutions}
		\item[\ref{dgl-1}]{
			Separieren Sie $x$ und $y(x)$ auf jeweils eine Seite der Differentialgleichung und integrieren Sie die erhaltene Gleichung.
			\[
				y'(x) = \frac{xe^{-y(x)}}{x^2 + 1} \implies e^{y(x)} y'(x) = \frac{x}{x^2 + 1}
			\]
			\[
				\implies \integral{}{}{e^{y(x)}y'(x)}{x} = \integral{}{}{\frac{x}{x^2+1}}{x}
				\atiPoints[1]
			\]
			Lösen Sie das Integral mithilfe einer logarithmischen Integration oder durch Substitution, indem Sie $x^2$ durch eine geeignete Variable ersetzen.
			\[
				\integral{}{}{e^y}{y} = \frac{1}{2} \integral{}{}{\frac{2x}{x^2+1}}{x}
			\]
			\[
				\implies e^{y(x)} = \frac{1}{2}\ln(x^2 + 1) + C = \ln\sqrt{x^2 + 1} + C
				\atiPoints[1]
			\]
			Notieren Sie die explizite Lösung durch die Anwendung von $\ln$.
			\[
				y(x) = \ln\boxb{\ln\curvb{A\sqrt{x^2+1}}} \separate A\define e^C
				\atiPoints[1]
			\]
			Fordern Sie nun $y(1)\demand 0$, bestimmen Sie die Konstante $A$ und setzen Sie die erhaltene Lösung in die explizite allgemeine Form ein.
			\[
				y(1) \demand 0 \implies 1 = \ln \curvb{A\sqrt{2}} \implies A = \frac{\sqrt{2}}{2}e
				\atiPoints[1]
			\]
			\[
				y(x) = \ln\boxb{\ln\curvb{\frac{e}{2}\sqrt{2(x^2 + 1)}}} = \ln\boxb{1 + \ln\curvb{\frac{1}{2}\sqrt{2(x^2+1)}}}
				\atiPoints[1]
			\]

			Sei $y\definedby \ln u$ mit $u(x) = \ln\curvb{A\sqrt{x^2+1}}$.
			Dann erhält man durch die Anwendung der Kettenregel die folgende Aussage.
			\[
				y'(x) = u'(x) \ln' u(x) = \frac{1}{A\sqrt{x^2+1}} \cdot A \cdot \frac{2x}{2\sqrt{x^2+1}} \cdot \frac{1}{u(x)} = e^{-y(x)} \frac{x}{x^2+1}
				\atiPoints[1]
			\]
		}

		\item[\ref{dgl-2}]{
			Separieren Sie $x$ und $y(x)$ wieder auf jeweils eine Seite der Differentialgleichung
			\[
				xy(x)y'(x) = \frac{x^2+2}{y(x)-1} \implies y(x)\boxb{y(x)-1}y'(x) = \frac{x^2+2}{x}
				\atiPoints[1]
			\]
			Integrieren Sie die rechte Seite der erhaltenen Gleichung durch Polynomintegration und der Umkehrregel.
			\[
				\integral{}{}{\boxb{y^2(x)-y(x)}y'(x)}{x} = \integral{}{}{\curvb{x+\frac{2}{x}}}{x}
			\]
			\[
				\implies \integral{}{}{y^2-y}{y} = \frac{x^2}{2} + 2\ln\abs{x}
				\atiPoints[1]
			\]
			Lösen Sie nun auch das Integral der rechten Seite durch Polynomintegration und notieren Sie die allgemeine Lösung in impliziter Form.
			\[
				\frac{y^3}{3} - \frac{y^2}{2} = \frac{x^2}{2} + 2\ln\abs{x} + C
			\]
			\[
				\implies 2y^3 - 3y^2 = 3x^2 + 12 \ln \abs{x} + D
				\atiPoints[1]
			\]

			Durch implizite Ableitung der allgemeinen Form erhalten Sie Folgendes.
			\[
				6y^2(x)y'(x) - 6y(x)y'(x) = 6x + \frac{12}{x} \implies y(x)\boxb{y(x)-1}y'(x) = x + \frac{2}{x}
			\]
			\[
				\implies xy(x)\boxb{y(x)-1}y'(x) = x^2+2
				\atiPoints[1]
			\]
		}

		\item[\ref{dgl-3}]{
			Definieren Sie $z(x)\define x + y(x)$ und bestimmen Sie die Ableitung von $z$.
			\[
				z'(x) = 1 + y'(x) \implies y'(x) = z'(x) - 1
			\]
			Substituieren Sie nun $x+y(x)$ in der Differentialgleichung durch $z(x)$.
			\[
				y'(x) = \frac{x+y(x)}{x+y(x)+2} \implies z'(x) - 1 = \frac{z}{z+2}
				\atiPoints[1]
			\]
			Führen Sie für die erhaltene Differentialgleichung das Verfahren der Trennung der Variablen durch.
			Separieren Sie $z(x)$ und $x$ auf jeweils eine Seite und integrieren Sie die erhaltene Gleichung.
			\[
				z'(x) = \frac{z(x)}{z(x)+2} + 1 = \frac{2z(x)+2}{z+2} = 2\,\frac{z(x)+1}{z(x)+2}
				\atiPoints[1]
			\]
			\[
				\implies \integral{}{}{\frac{z(x)+2}{z(x)+1}z'(x)}{x} = \integral{}{}{\curvb{1 + \frac{1}{z+1}}}{z} = \integral{}{}{2}{x}
			\]
			\[
				\implies z(x) + \ln\abs{z(x)+1} = 2x + C
				\atiPoints[1]
			\]
			Führen Sie die Resubstitution durch und geben Sie die allgemeine Lösung in impliziter Form an.
			\[
				x + y(x) + \ln\abs{x+ y(x) + 1} = 2x + C
			\]
			\[
				\implies x+y(x)+1 = \exp(x-y(x)+C) = Ae^{x-y(x)} \separate A\define e^C
			\]
			\[
				\implies x+y(x) = Ae^{x-y(x)} -1
				\atiPoints[1]
			\]
			Fordern Sie die gegebenen Anfangsbedingungen und bestimmen Sie die Konstante $A$.
			\[
				y(1)\demand -1 \implies 0=Ae^2-1 \implies A= e^{-2}
			\]
			\[
				\implies x+y(x) = \frac{e^{x-y(x)}}{e^2}-1 = e^{x-y(x)-2}-1
				\atiPoints[1]
			\]

			Auch hier ist wieder eine implizite Ableitung notwendig.
			\[
				1+y'(x) = Ae^{x-y(x)} \boxb{1-y'(x)}
			\]
			Durch Verwendung der allgemeinen Lösung erhalten Sie für die Konstante $A$ den folgenden Ausdruck.
			\[
				A = e^{y(x)-x}\boxb{x+y(x)+1}
			\]
			Das Einsetzen dieser Gleichung resultiert dann in der gewünschten Differentialgleichung.
			\[
				1+y'(x) = e^{y(x)-x}\boxb{x+y(x)+1} e^{x-y(x)} \boxb{1-y'(x)}
			\]
			\[
				\implies 1+y'(x) = x+y(x)+1 - \boxb{x+y(x)+1}y'(x)
			\]
			\[
				\implies y'(x)\boxb{x+y(x)+2} = x+y(x)
				\atiPoints[1]
			\]
		}
	\end{atiSubtaskSolutions}
\end{atiSolution}
    \import{database/mathematische_methoden_der_physik_i/ordinary_deq/separable_deq/}{task-orthogonaltrajektorien}
    \import{database/mathematische_methoden_der_physik_i/ordinary_deq/separable_deq/}{task-orthogonaltrajektorien_und_richtungsfeld}
    \begin{atiTask}[
	title = Zwei separable Differentialgleichungen,
	language = Deutsch
]
	Lösen Sie die folgenden Differentialgleichungen mittels Trennung der Variablen.
	\begin{atiSubequations}
		\medskip
		\begin{minipage}[t]{0.5\textwidth}
		\item{
			\frac{1}{\cos x}\fdrv{y(x)}{x} = -\tan x \cdot y^{-2}
		}
		\end{minipage}
		% \begin{minipage}{0.5\textwidth}
		\item{
			xyy' = y-1
		}
		% \end{minipage}
	\end{atiSubequations}
	Überprüfen Sie Ihr Ergebnis, indem Sie die Probe durch Einsetzen Ihrer Lösung in die ursprüngliche Differentialgleichung durchführen.
\end{atiTask}
    % \end{multicols}
\end{document}