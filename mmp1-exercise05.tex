\documentclass[a4paper,fleqn, 10pt]{article}

\usepackage[utf8]{inputenc}
\usepackage[T1]{fontenc}
\usepackage[ngerman]{babel}
\usepackage[bottom=25mm,left=35mm,right=33mm,bottom=35mm]{geometry}
\usepackage{times}
\linespread{1.15}

\usepackage{turnthepage}
\renewcommand{\turnthepage}{\it bitte wenden}

\usepackage[pgfkeys,custom]{ati}
% \SetupExSheets{solution/print=true, question/type=exam}

\allowdisplaybreaks

\usepackage{import}

\usepackage{multicol}
\usepackage{dsfont}


\begin{document}

	\pagestyle{empty}

	\hrule
	\section*{\centering Mathematische Methoden der Physik I \\ Übungsserie 5}
	\medskip
	Dr. Agnes Sambale \hfill Wintersemester 17/18\\
	agnes.sambale@uni-jena.de \hfill Abgabe: Mittwoch, 22.11.17
	\bigskip
	\hrule
	\bigskip
	\bigskip

	\atiShowSolutionsfalse

	\begin{atiTask}[
  title = Der integrierende Faktor
]
  Untersuchen Sie die folgenden Differentialgleichungen auf Exaktheit.
  % und lösen Sie die nicht-exakten Differentialgleichungen.
  \begin{atiSubequations}
    \item{
      \curvb{x+y}x^2y' + xy^2 + 3x^2y = 0
    }
    \item{
      yx^3 - 2x^4 = \curvb{3y^2x^3-x^4} y'
    }
    \item{
      \curvb{x\cos y - xy\sin y}y' + 2y\cos y + x = 0
    }
  \end{atiSubequations}
  Berechnen Sie die allgemeinen Lösungen der nicht-exakten Differentialgleichungen in impliziter Form, indem Sie die folgende Anleitung verwenden.
  \begin{atiItems}
    \item{
      Bestimmen Sie einen integrierenden Faktor, der nur von der ersten freien Variable abhängt.
    }
    \item{
      Notieren Sie die neue, mit dem integrierenden Faktor multiplizierte, Differentialgleichung.
    }
    \item{
      Zeigen Sie die Exaktheit der neuen Differentialgleichung.
    }
    \item{
      Lösen Sie die neue Differentialgleichung durch das Auffinden einer Potentialfunktion.
    }
    \item{
      Führe Sie die Probe durch, indem Sie die erhaltene Lösung implizit differenzieren und auf die ursprüngliche Differentialgleichung zurückführen.
    }
  \end{atiItems}
  % \begin{atiSubtasks}
  %   \item{
  %     Überprüfen Sie, ob die gegebenen Differentialgleichungen exakt sind.
  %   }
  %   \item{
  %     Lösen Sie die nicht-exakten Differentialgleichungen, indem Sie nach der folgenden Anleitung vorgehen.
  %     \begin{atiItems}
  %       \item{
  %         Bestimmen Sie einen integrierenden Faktor, der nur von der ersten freien Variable $x$ abhängt.
  %       }
  %       \item{
  %         Überprüfen Sie die Exaktheit der mit dem integrierenden Faktor multiplizierten Differentialgleichung.
  %       }
  %       \item{
  %         Lösen Sie nun die mit dem integrierenden Faktor multiplizierte Differentialgleichung durch das Auffinden einer Potentialfunktion.
  %       }
  %       \item{
  %         Führe Sie die Probe durch, indem Sie Ihre Lösung implizit differenzieren und auf die ursprüngliche Differentialgleichung zurückführen.
  %       }
  %     \end{atiItems}
  %   }
  % \end{atiSubtasks}
\end{atiTask}
	\begin{atiTask}[
	title = Integrierende Faktoren,
	topic = Gewöhnliche Differentialgleichungen,
	subtopic = Exakte Differentialgleichungen: Der integrierende Faktor,
	language = Deutsch
]
	\begin{atiSubtasks}
		\item{
			Die folgende Differentialgleichung sei gegeben.
			\[
				A(x,y) + B(x,y)y' = 0
			\]
			Zeigen Sie durch Verwendung der Integrabilitätsbedingung die beiden folgenden Aussagen.
			\begin{atiItems}
				\item{
					Kann der folgende Ausdruck als eine Funktion $f$ der Variablen $z(x,y)\define xy$ geschrieben werden, so hängt auch der integrierende Faktor $\lambda$ nur von dieser Variable $z$ ab.
					\[
						\frac{1}{xA(x,y)-yB(x,y)}\curvb{\partial_x B(x,y) - \partial_y A(x,y)}
					\]
					Geben Sie den Zusammenhang von $\lambda$ und $f$ an.
				}
				\item{
					Kann der folgende Ausdruck als eine Funktion $g$ der Variablen $w(x,y)\define x + y$ geschrieben werden, so hängt auch der integrierende Faktor $\lambda$ nur von dieser Variable $z$ ab.
					\[
						\frac{1}{A(x,y)-B(x,y)}\curvb{\partial_x B(x,y) - \partial_y A(x,y)}
					\]
					Geben Sie den Zusammenhang von $\lambda$ und $g$ an.

				}
			\end{atiItems}
		}
		\item{
			Eine der beiden zuvor genannten Eigenschaften trifft auf eine der beiden folgenden Differentialgleichungen zu.
			Finden Sie diesen Fall heraus und berechnen Sie einen integrierenden Faktor mit dem Ergebnis aus dem vorherigen Aufgabenteil.
			\begin{atiSubequations}
				\item{
					(xy-1) + \curvb{x^2-xy}y' = 0
				}
				\item{
					y + \curvb{x - 2x^2y^3}y' = 0
				}
			\end{atiSubequations}
			Lösen Sie die Differentialgleichung mit diesem integrierenden Faktor und machen Sie anschließend die Probe anhand der ursprünglichen Differentialgleichung.
			Die Lösung der verbleibenden Differentialgleichung ist nicht verlangt.
		}
	\end{atiSubtasks}
\end{atiTask}

	% \begin{atiTask}[
  title = Freier Fall mit Reibung
]
 Beim freien Fall mit Fallbeschleunigung $g\in\setR^+$ unter dem Einfluss einer geschwindigkeitsproportionalen Reibungskraft mit Reibungskoeffizient $\gamma\in\setR^+$ genügt die abwärts gerichtete Geschwindigkeitsfunktion $v$ eines Massenpunktes mit Masse $m\in\setR^+$ der folgenden Differentialgleichung.
 \[
   m\tdrv{v} + \gamma v = mg
 \]
 \begin{atiSubtasks}
   \item{\locallabel{a}
    Lösen Sie diese Differentialgleichung durch die Methode der Variation der Konstanten und bestimmen Sie eine Lösung, die den Anfangsbedingungen $t_0\define 0$ und $v_0\define v(t_0)=0$ genügt.
  }
  \item{\locallabel{b}
    Berechnen Sie die stationäre Endgeschwindigkeit $v_\infty$ mithilfe der folgenden Definition und der Lösung der Differentialgleichung.
    \[
      v_\infty \define \lim_{t\conv\infty}v(t)
    \]
    Berechnen Sie ein zweites Mal $v_\infty$ unter der Annahme, dass die Lösung des Anfangswertproblems nicht bekannt ist.
    Betrachten Sie hierfür den Grenzwert der Differentialgleichung.
  }
  \item{\locallabel{c}
    \textbf{Zusatz:}
    Lösen Sie noch einmal die Differentialgleichung des freien Falls durch die Methode der Trennung der Variablen.
  }
 \end{atiSubtasks}
\end{atiTask}
\begin{atiSolution}
  \begin{atiSubtaskSolutions}
    \item[\localref{a}]{
      Die zur gegebenen Differentialgleichung zugehörigen homogene Differentialgleichung kann durch die folgende Form beschrieben werden.
      \[
        m\tdrv{v} + \gamma v = 0 \implies \tdrv{v} = -\frac{\gamma}{m}v\atiPoints[\frac{1}{2}]
      \]
      Diese Formel lässt sich separieren und durch die Methode der Trennung der Variablen lösen.
      \[
        \frac{\tdrv{v}(t)}{v(t)} = -\frac{\gamma}{m} \implies \integral{t_0}{t}{\frac{\tdrv{v}(t)}{v(t)}}{t} = \integral{v_0}{v(t)}{\frac{1}{s}}{s} = -\integral{t_0}{t}{\frac{\gamma}{m}}{t}\atiPoints[\frac{1}{2}]
      \]
      \[
        \implies \ln\curvb{\frac{v(t)}{v_0}} = -\frac{\gamma}{m}(t-t_0) \implies v(t) = v_0 e^{-\frac{\gamma}{m}(t-t_0)}\atiPoints[\frac{1}{2}]
      \]
      Nach der Methode der Variation der Konstanten gehen wir nun davon aus, dass sich die Lösung der ursprünglichen Differentialgleichung durch die folgende Funktion beschreiben lässt.
      \[
        v(t) = \varphi(t)e^{-\frac{\gamma}{m}(t-t_0)}\atiPoints[\frac{1}{2}]
      \]
      \[
        \implies \tdrv{v}(t) = \tdrv{\varphi}(t)e^{-\frac{\gamma}{m}(t-t_0)} - \varphi(t)\frac{\gamma}{m}e^{-\frac{\gamma}{m}(t-t_0)}\atiPoints[\frac{1}{2}]
      \]
      Das Einsetzen der Ableitung in die ursprüngliche Differentialgleichung ergibt dann das Folgende.
      \[
        m\tdrv{\varphi}(t)e^{-\frac{\gamma}{m}(t-t_0)} = mg \implies \tdrv{\varphi}(t) = ge^{\frac{\gamma}{m}(t-t_0)}\atiPoints[\frac{1}{2}]
      \]
      \[
        \implies \varphi(t) = \frac{mg}{\gamma}e^{\frac{\gamma}{m}(t-t_0)} - \frac{mg}{\gamma} + \varphi(t_0)
      \]
      \[
        \implies v(t) = \varphi(t)e^{-\frac{\gamma}{m}(t-t_0)} = \frac{mg}{\gamma}\boxb{1 - e^{-\frac{\gamma}{m}(t-t_0)}} + \varphi(t_0)e^{-\frac{\gamma}{m}(t-t_0)} \atiPoints[\frac{1}{2}]
      \]
      Zu beachten ist hier, dass $v_0 = v(t_0) = \varphi(t_0)$ gilt.
      Die Lösung des Anfangswertproblems ist damit wegen $t_0=0$ und $v_0=0$ gegeben durch die folgende Funktion.
      \[
        v(t) = \frac{mg}{\gamma}\curvb{1-e^{-\frac{\gamma}{m}t}}\atiPoints[\frac{1}{2}]
      \]
    }
    \item[\localref{b}]{
      Die stationäre Endgeschwindigkeit lässt sich nun einfach mithilfe des Limes berechnen.
      \[
        v_\infty = \lim_{t\conv\infty}v(t) = \lim_{t\conv\infty} \frac{mg}{\gamma}\curvb{1-e^{-\frac{\gamma}{m}t}} = \frac{mg}{\gamma}\curvb{1-\lim_{t\conv\infty}e^{-\frac{\gamma}{m}t}} = \frac{mg}{\gamma} \atiPoints[1]
      \]
      Ohne die Kenntnis der Lösung lässt sich nun das Folgende notieren.
      \[
        \lim_{t\conv\infty}\boxb{m\tdrv{v}(t) + \gamma v(t)} = \lim_{t\conv\infty}mg \implies m\lim_{t\conv\infty}\tdrv{v}(t) + \gamma \lim_{t\conv\infty}v(t) = mg
      \]
      Da der Limes existiert und $v$ eine stetig differenzierbare Funktion ist, muss demnach das Folgende gelten.
      \[
        \lim_{t\conv\infty}\tdrv{v}(t)=0 \implies v_\infty = \lim_{t\conv\infty}v(t) = \frac{mg}{\gamma} \atiPoints[1]
      \]
    }
    \item[\localref{c}]{
      Die in der Aufgabe gegebene Differentialgleichung ist separierbar und damit durch die Methode der Trennung der Variablen lösbar.
      \[
        \tdrv{v}(t) = g-\frac{\gamma}{m}v(t) \implies \frac{\tdrv{v}(t)}{g-\frac{\gamma}{m}v(t)} = 1 \implies \integral{t_0}{t}{\frac{\tdrv{v}(t)}{g-\frac{\gamma}{m}v(t)}}{t} = \integral{t_0}{t}{}{t} \atiPoints[+\frac{1}{2}]
      \]
      \[
        \implies -\frac{m}{\gamma}\ln\boxb{\frac{g-\frac{\gamma}{m}v(t)}{g-\frac{\gamma}{m}v_0}} = t-t_0
      \]
      \[
        \implies v(t) = \frac{m}{\gamma}\boxb{g - \curvb{g-\frac{\gamma}{m}v_0}e^{-\frac{\gamma}{m}(t-t_0)}}
      \]
      \[
        \implies v(t) = \frac{mg}{\gamma}\boxb{1-e^{-\frac{\gamma}{m}(t-t_0)}} + v_0e^{-\frac{\gamma}{m}(t-t_0)}\atiPoints[+\frac{1}{2}]
      \]
    }
  \end{atiSubtaskSolutions}
\end{atiSolution}
	% \bigskip
	% \begin{atiTask}[
  title = Exakte Differentialgleichungen
]
  \begin{atiSubtasks}
    \item{\locallabel{a}
      Zeigen Sie, dass die folgenden Differentialgleichungen exakt sind, und lösen Sie diese durch das Auffinden einer Potentialfunktion.
      \begin{atiSubequations}
        \item{\locallabel{ai}
          \curvb{x+y^3}y' + y = x^3
        }
        \item{\locallabel{aii}
          0 = \sin\curvb{xy^2} + xy^2\cos\curvb{xy^2} + \boxb{2x^2y\cos\curvb{xy^2} + 2y}y'
        }
      \end{atiSubequations}
    }
    \item{\locallabel{b}
      Zeigen Sie, dass es sich bei den folgenden Gleichungen um nicht exakte Differentialgleichungen handelt, indem Sie die Integrabilitätsbedingung überprüfen.
      \begin{atiSubequations}
        \item{\locallabel{bi}
          0 = 2\cos y + 4x^2y\sin y + \curvb{yx^3\cos y + x^3\sin y}y' - xy'\sin y
        }
        \item{\locallabel{bii}
          x\arctan\curvb{\frac{x}{y^2}} + \frac{x^2y^2}{x^2+y^4} = \frac{2x^3y}{x^2+y^4}y'
        }
      \end{atiSubequations}
    }
  \end{atiSubtasks}
\end{atiTask}
\begin{atiSolution}
  \begin{atiSubtaskSolutions}
    \item[\localref{a}]{
      \begin{atiSubtaskSolutions}
        \item[\localref{ai}]{
          Wir definieren zwei Funktionen $\func{f,g}{\setR^2}{\setR}$, sodass für alle $x,y\in\setR$ folgende Gleichungen gelten.
          \[
            f(x,y)\define y-x^3\separate g(x,y)\define x+y^3
          \]
          \[
            \implies f(x,y) + g(x,y)y' = 0
          \]
          Die Integrabilitätsbedingung ist nach der folgenden Aussage für alle $x,y\in\setR$ erfüllt.
          Die Differentialgleichung ist damit exakt.
          \[
            \partial_2f(x,y) = \pdrvopb{\tilde{y}}{f(x,\tilde{y})}{y} = 1 = \pdrvopb{\tilde{x}}{g(\tilde{x},y)}{x} = \partial_1g(x,y)
          \]
          Es gibt nun eine zweimal stetig differenzierbare Potentialfunktion $\func{\varphi}{\setR^2}{\setR}$, die die folgende Bedingung erfüllt.
          \[
            \partial_1\varphi = f \separate \partial_2\varphi = g
          \]
          Um diese zu berechnen, kann man zum Beispiel einen der beiden folgenden Wege verwenden.
          \[
            \varphi(x,y) = \iintegralb{\partial_1\varphi(s,y)}{s}{x} = \iintegralb{f(s,y)}{s}{x} = yx - \frac{x^4}{4} + c(y)
          \]
          \[
            \implies \partial_2\varphi(x,y) = x + c'(y) = g(x,y) = x + y^3
          \]
          \[
            \implies c'(y) = y^3 \implies c(y) = \frac{y^4}{4} + K
          \]
          \[
            \implies \varphi(x,y) = xy - \frac{x^4}{4} + \frac{y^4}{4} + K
          \]
          \[
            \varphi(x,y) = \iintegralb{\partial_2\varphi(x,s)}{s}{y} = \iintegralb{g(x,s)}{s}{y} = xy + \frac{y^4}{4} + d(x)
          \]
          \[
            \implies \partial_1\varphi(x,y) = y + d'(x) = f(x,y) = y - x^3
          \]
          \[
            \implies d'(x) = -x^3 \implies d(x) = -\frac{x^4}{4} + K
          \]
          \[
            \implies \varphi(x,y) = xy - \frac{x^4}{4} + \frac{y^4}{4} + K
          \]
          Jetzt gehen wir davon aus, dass es sich bei $y$ um eine stetig differenzierbare Funktion handelt.
          Nach Verwendung der Kettenregel und der gegebenen Differentialgleichung können wir auf das Folgende für alle $x$ einer offenen Teilmenge $U\subset\setR$ schließen.
          \[
            \fdrvopb{s}{\varphi(s,y(s))}{x} = \partial_1\varphi(x,y(x)) + \partial_2\varphi(x,y(x)) y'(x) = 0
          \]
          Dies ist nur dann möglich, wenn es für alle $x\in U$ eine Konstante $C\in\setR$ gibt, sodass das Folgende gilt.
          \[
            \varphi(x,y(x)) = C = xy(x) - \frac{x^4}{4} + \frac{y^4(x)}{4} + K
          \]
          Dies ist auch gleichzeitig die Funktion $y$ in impliziter Form.
        }
        \item[\localref{aii}]{
          Wir definieren zwei Funktionen $\func{f,g}{\setR^2}{\setR}$, sodass für alle $x,y\in\setR$ folgende Gleichungen gelten.
          \[
            f(x,y) \define \sin\curvb{xy^2} + xy^2\cos\curvb{xy^2} \separate g(x,y) \define 2x^2y\cos\curvb{xy^2} + 2y
          \]
          \[
            \implies f(x,y) + g(x,y)y' = 0
          \]
          Die Integrabilitätsbedingung ist nach der folgenden Aussage für alle $x,y\in\setR$ erfüllt.
          Die Differentialgleichung ist damit exakt.
          \[
            \partial_2f(x,y) = \pdrvopb{\tilde{y}}{f(x,\tilde{y})}{y} = 4xy\cos\curvb{xy2} - 2x^2y^3\sin\curvb{xy^2} = \pdrvopb{\tilde{x}}{g(\tilde{x},y)}{x} = \partial_1g(x,y)
          \]
          Es gibt nun eine zweimal stetig differenzierbare Potentialfunktion $\func{\varphi}{\setR^2}{\setR}$, die die folgende Bedingung erfüllt.
          \[
            \partial_1\varphi = f \separate \partial_2\varphi = g
          \]
          Um diese zu berechnen, kann man zum Beispiel einen der beiden folgenden Wege verwenden.
          \[
            \varphi(x,y) = \iintegralb{\partial_1\varphi(s,y)}{s}{x} = \iintegralb{f(s,y)}{s}{x} = -\frac{\cos\curvb{xy^2}}{y^2} + c(y)
          \]
          \[
            \implies \partial_2\varphi(x,y) = + c'(y) = g(x,y) =
          \]
          \[
            \implies c'(y) = y^3 \implies c(y) = \frac{y^4}{4} + K
          \]
          \[
            \implies \varphi(x,y) = xy - \frac{x^4}{4} + \frac{y^4}{4} + K
          \]
          \[
            \varphi(x,y) = \iintegralb{\partial_2\varphi(x,s)}{s}{y} = \iintegralb{g(x,s)}{s}{y} = xy + \frac{y^4}{4} + d(x)
          \]
          \[
            \implies \partial_1\varphi(x,y) = y + d'(x) = f(x,y) = y - x^3
          \]
          \[
            \implies d'(x) = -x^3 \implies d(x) = -\frac{x^4}{4} + K
          \]
          \[
            \implies \varphi(x,y) = xy - \frac{x^4}{4} + \frac{y^4}{4} + K
          \]
        }
      \end{atiSubtaskSolutions}
    }
    \item[\localref{b}]{
      \begin{atiSubtaskSolutions}
        \item[\localref{bi}]{
          Wir definieren zwei Funktionen $\func{f,g}{\setR^2}{\setR}$, sodass für alle $x,y\in\setR$ folgende Gleichungen gelten.
          \[
            f(x,y) \define 2\cos y + 4x^2y\sin y \separate g(x,y) \define yx^3\cos y + x^3\sin y - x\sin y
          \]
          \[
            \implies f(x,y) + g(x,y)y' = 0
          \]
          Wir überprüfen wieder die Integrabilitätsbedingung.
          Es gibt also $x,y\in\setR$, sodass die folgende Implikation gilt.
          \[
            \partial_2f(x,y) = \pdrvopb{\tilde{y}}{f(x,\tilde{y})}{y} = -2\sin y + 4x^2\sin y + 4x^2y\cos y
          \]
          \[
            \partial_1g(x,y) = \pdrvopb{\tilde{x}}{g(\tilde{x},y)}{x} = 3x^2y\cos y + 3x^2\sin y - \sin y
          \]
          \[
            \implies \partial_2f(x,y) - \partial_1g(x,y) = \curvb{x^2-1}\sin y + \curvb{1-3x^2}y\cos y \neq 0
          \]
          Die Integrabilitätsbedingung ist damit nicht erfüllt.
          Demzufolge ist diese Differentialgleichung nicht exakt.
        }
        \item[\localref{bii}]{
          Wir definieren zwei Funktionen $\func{f,g}{\setR\times\setR\setminus\set{0}}{\setR}$, sodass für alle $x,y\in\setR$ mit $y\neq 0$ folgende Gleichungen gelten.
          \[
            f(x,y) \define x\arctan\curvb{\frac{x}{y^2}} + \frac{x^2y^2}{x^2+y^4} \separate g(x,y) \define -\frac{2x^3y}{x^2+y^4}
          \]
          \[
            \implies f(x,y) + g(x,y)y' = 0
          \]
          Wir überprüfen wieder die Integrabilitätsbedingung.
          Es gibt also $x,y\in\setR$ mit $y\neq 0$, sodass die folgende Implikation gilt.
          \[
            \partial_2f(x,y) = \pdrvopb{\tilde{y}}{f(x,\tilde{y})}{y} = -\frac{4x^4y^5}{\curvb{x^2+y^4}^2}
          \]
          \[
            \partial_1g(x,y) = \pdrvopb{\tilde{x}}{g(\tilde{x},y)}{x} = -\frac{6x^2y}{x^2+y^4} + \frac{4x^4y}{\curvb{x^2+y^4}^2}
          \]
          % \[
          %   \pdrvopb{\tilde{x}}{g(\tilde{x},y)}{x} = -\frac{6x^2y}{x^2+y^4} + \frac{4x^4y}{\curvb{x^2+y^4}^2}
          % \]
          \[
            \implies \partial_2f(x,y) - \partial_1g(x,y) = \frac{6x^2y}{x^2+y^4} - \frac{8x^4y}{\curvb{x^2+y^4}^2} \neq 0
          \]
          Die Integrabilitätsbedingung ist damit nicht erfüllt.
          Demzufolge ist diese Differentialgleichung nicht exakt.
        }
      \end{atiSubtaskSolutions}
    }
  \end{atiSubtaskSolutions}
\end{atiSolution}
	% \newpage
	% \begin{atiTask}[
  title = Vollständiges Differential
]
  Es sei die folgende skalare Funktion für alle $x,y,z\in\setReal$ gegeben.
  \[
    \function{U}{\setReal^3}{\setReal}
    \separate
    U(x,y,z)\define x^4yz^2 + 2y^2x^3e^z
  \]
  \begin{atiSubtasks}
    \item{\locallabel{a}
      Bestimmen Sie das totale Differential $\infinitesimal U$ der Funktion $U$.
    }
    \item{\locallabel{b}
      Untersuchen Sie, ob es eine Funktion $\function{V}{\setReal^3}{\setReal}$ gibt, sodass das totale Differential $\infinitesimal V$ von $V$ für alle $x,y,z\in\setReal$ mit $x\neq 0$ die folgende Gleichung erfüllt.
      \[
        \infinitesimal V(x,y,z)\define \frac{1}{x^2} \infinitesimal U(x,y,z)
      \]
      \begin{atiNote}
        Überprüfen Sie dafür die Integrabilitätsbedingung des gegebenen Differentials.
      \end{atiNote}
    }
  \end{atiSubtasks}
\end{atiTask}
\begin{atiSolution}
  \begin{atiSubtaskSolutions}
    \item[\localref{a}]{
      Der Gradient der skalaren Funktion $U$ berechnet sich wie folgt, da es sich bei $U$ um eine stetig differenzierbare Funktion handelt.
      \[
        \nabla U(x,y,z) =
        \begin{pmatrix}
          \partial_1U(x,y,z) \\
          \partial_2U(x,y,z) \\
          \partial_3U(x,y,z)
        \end{pmatrix}
        =
        \begin{pmatrix}
          4x^3yz^2 + 6x^2y^2e^z \\
          x^4z^2 + 4x^3ye^z \\
          2x^4yz + 2x^3y^2e^z
        \end{pmatrix}
      \]
      Das totale Differential existiert, da sich $U$ zweimal stetig differenzieren lässt und damit durch den Satz von Schwarz die Integrabilitätsbedingung automatisch erfüllt ist.
      Es ist dann durch den folgenden Ausdruck gegeben.
      \[
        \infinitesimal U = \dotProduct{\nabla U}{\infinitesimal r}
        \separate
        \infinitesimal r \define
        \begin{pmatrix}
          \infinitesimal x \\
          \infinitesimal y \\
          \infinitesimal z
        \end{pmatrix}
      \]
      \[
        \implies \infinitesimal U(x,y,z) = \partial_1U(x,y,z)\infinitesimal x + \partial_2U(x,y,z)\infinitesimal y + \partial_3U(x,y,z)\infinitesimal z\atiPoints[1]
      \]
      \[
        \implies \infinitesimal U(x,y,z) = \roundBrackets{4x^3yz^2 + 6x^2y^2e^z}\infinitesimal x + \roundBrackets{x^4z^2 + 4x^3ye^z}\infinitesimal y + \roundBrackets{2x^4yz + 2x^3y^2e^z}\infinitesimal z \atiPoints[1]
      \]
    }
    \item[\localref{b}]{
      Für alle $x,y,z\in\setReal$ mit $x\neq 0$ gilt nun die folgende Aussage.
      Wir definieren hierfür die Funktionen $\function{f,g,h}{\setReal\setminus\set{0}{}\times\setReal^2}{\setReal}$.
      \[
        \frac{1}{x^2}\infinitesimal U(x,y,z) = \underbrace{\roundBrackets{4xyz^2 + 6y^2e^z}}_{\reverseDefine f(x,y,z)}\infinitesimal x + \underbrace{\roundBrackets{x^2z^2 + 4xye^z}}_{\reverseDefine g(x,y,z)}\infinitesimal y + \underbrace{\roundBrackets{2x^2yz + 2xy^2e^z}}_{\reverseDefine h(x,y,z)}\infinitesimal z
      \]
      Ein notwendiges (aber nicht hinreichendes) Kriterium für die Existenz der in der Aufgabenstellung beschriebenen Funktion $V$, ist die Erfüllung der Integrabilitätsbedingung.
      Es gibt nun $x,y,z\in\setReal$ mit $x\neq 0$, die die folgenden Implikationen erfüllen.
      \[
        \partial_2f(x,y,z) = \leibnizPartialDerivativeOperatorValue{s}{\roundBrackets{\frac{\partial_1U(x,s,z)}{x^2}}}{y} = \frac{\partial_2\partial_1U(x,y,z)}{x^2}
      \]
      \[
        \partial_1g(x,y,z) = \leibnizPartialDerivativeOperatorValue{s}{\roundBrackets{\frac{\partial_2U(s,y,z)}{s^2}}}{x} = \frac{\partial_2\partial_1U(x,y,z)}{x^2} - \frac{2\ \partial_2U(x,y,z)}{x^3}
      \]
      \[
        \implies \partial_2f(x,y,z) - \partial_1g(x,y,z) = \frac{2\ \partial_2U(x,y,z)}{x^3} = xz^2 + 4ye^z \neq 0
      \]
      \[
        \partial_3f(x,y,z) = \leibnizPartialDerivativeOperatorValue{s}{\roundBrackets{\frac{\partial_1U(x,y,s)}{x^2}}}{z} = \frac{\partial_3\partial_1U(x,y,z)}{x^2}
      \]
      \[
        \partial_1h(x,y,z) = \leibnizPartialDerivativeOperatorValue{s}{\roundBrackets{\frac{\partial_3U(s,y,z)}{s^2}}}{x} = \frac{\partial_3\partial_1U(x,y,z)}{x^2} - \frac{2\ \partial_3U(x,y,z)}{x^3}
      \]
      \[
        \implies \partial_3f(x,y,z) - \partial_1h(x,y,z) = \frac{2\ \partial_3U(x,y,z)}{x^3} = 2xyz + 2y^2e^z \neq 0
      \]
      \[
        \partial_3g(x,y,z) = \leibnizPartialDerivativeOperatorValue{s}{\roundBrackets{\frac{\partial_2U(x,y,s)}{x^2}}}{z} = \frac{\partial_3\partial_2U(x,y,z)}{x^2}
      \]
      \[
        \partial_2h(x,y,z) = \leibnizPartialDerivativeOperatorValue{s}{\roundBrackets{\frac{\partial_3U(x,s,z)}{x^2}}}{y} = \frac{\partial_2\partial_3U(x,y,z)}{x^2}
      \]
      \[
        \implies \partial_3g(x,y,z) - \partial_2h(x,y,z) = 0
      \]
      \atiPoints[1]Die ersten beiden Folgerungen zeigen, dass die Integrabilitätsbedingung nicht erfüllt ist.
      Es reicht eine dieser beiden Aussagen zu zeigen.
      \atiPoints[1]Demzufolge kann es eine solche Funktion $V$ nicht geben.
    }
  \end{atiSubtaskSolutions}
\end{atiSolution}

	% \begin{atiTask}[
	title = Die Methode der Variation der Konstanten
]
	Lösen Sie die folgenden Differentialgleichungen mithilfe der Methode der Variation der Konstanten.
	Führen Sie in allen Fällen eine Probe durch.
	\begin{atiSubequations}
		% \begin{multicols}{2}
		\item{
			\frac{y'}{x^3} - \frac{2y}{x^4} = \sin x
		}
		\item{
			y' = y\sin x - 2 \sin x
		}
		% \end{multicols}
		\item{
			(x+1)y' = 2y + (x+1)^{\frac{5}{2}}
		}
	\end{atiSubequations}
\end{atiTask}
	% \begin{atiTask}[
	title = Eine nichtlineare Differentialgleichung
]
	Lösen Sie die folgende nichtlineare Differentialgleichung, indem Sie diese auf eine lineare Differentialgleichung durch die Substitution $z = y^{-2}$ zurückführen und mithilfe der Methode der Variation der Konstanten behandeln.
	\[
		y' - y + 3x^2y^3 = 0
	\]
	Machen Sie die Probe durch Einsetzen Ihrer Lösung in die ursprüngliche Differentialgleichung.
\end{atiTask}
	% \begin{atiTask}[
	title = Ein Ausweg aus der Zombieapokalypse
]
	Wenig später bricht auch auf einer Nachbarinsel eine merkwürdige Krankheit aus.
	In einer  Zeitspanne $\tau\in\setR^+$ infiziert sich ein Anteil $\delta$ der gesunden Menschen auf der Insel und mutiert zu Zombies.
	Dieses Mal wird sofort mit der Evakuierung begonnen.
	Während derselben Zeitspanne $\tau$ können $N_0$ Menschen die Insel per Schiff verlassen.
	Infizierte können die Insel nicht verlassen.
	\begin{atiSubtasks}
		\item{
			Stellen Sie die zugehörigen Differentialgleichungen für $M(t)$ und $Z(t)$ auf und lösen Sie diese für die folgenden Anfangsbedingungen.
			\[
				t_0 \define 0 \separate M_0\define M(0)\define N \separate Z_0\define Z(t) \define 0
			\]
		}
		\item{
			\textbf{Zusatz:} Skizzieren Sie $M(t)$ und $Z(t)$ für $t\in\setR^+$ und den folgenden Parametern.
			\[
				N = 10000\separate \delta = \frac{1}{3} \separate N_0 = 100
			\]
		}
		\item{
			\textbf{Zusatz:} Wie viele gesunde Menschen konnten gerettet werden? Wie verändert sich diese Zahl, wenn mit einem zweiten Schiff doppelt so viele Menschen pro $\tau$ evakuiert werden können?
		}
	\end{atiSubtasks}
\end{atiTask}

	% \begin{atiTask}[
	title = Klassifikation von gewöhnlichen Differentialgleichungen
]
	Klassifizieren Sie die folgenden gewöhnlichen Differentialgleichungen durch ihre Ordnung, Homogenität, Linearität und Separabilität.
	Lösen Sie zudem die separablen Differentialgleichungen.
	\begin{atiSubequations}
	\begin{multicols}{2}
		\item{\locallabel{dgl1}
			\leibnizNDerivative[2]{y(x)}{x} + 2\leibnizDerivative{y(x)}{x} = y
		}
		\item{\locallabel{dgl2}
			\leibnizDerivative{y(x)}{x} + \sin y - x^2 = 0
		}
		\item{\locallabel{dgl3}
			y' + \tan(x)\cdot y = 0
		}
		\item{\locallabel{dgl4}
			\frac{y''}{y'} + x = 0
		}
		\item{\locallabel{dgl5}
			yy'-x = 0
		}
		\item{\locallabel{dgl6}
			\frac{x+1}{y+2} = \leibnizDerivative{y(x)}{x}
		}
	\end{multicols}
		\item{\locallabel{dgl7}
			\sqrt{y^2 + 3a^2 + ya\roundBrackets{2-\frac{4a}{2y}}} + \leibnizDerivative{y(x)}{x}\sqrt{x^2-a^2} = \sqrt{x+a}
		}
	\end{atiSubequations}
\end{atiTask}
\begin{atiSolution}
	\begin{atiSubtaskSolutions}
		\item[\localref{dgl1}]{
			Durch eine einfache Umformung erhalten wir die folgende Differentialgleichung.
			\[
				y'' + 2y' -y = 0
			\]
			\atiPoints[1]Damit handelt es sich bei dieser Differentialgleichung um eine gewöhnliche, lineare, homogene, nicht-separable Differentialgleichung 2.Ordnung.
		}
		\item[\localref{dgl2}]{
			Durch eine einfache Umformung erhalten wir die folgende Differentialgleichung.
			\[
				y'+ \sin y = x^2
			\]
			\atiPoints[1]Damit handelt es sich bei dieser Differentialgleichung um eine gewöhnliche, nicht-lineare, nicht-separable Differentialgleichung 1.Ordnung.
		}
		\item[\localref{dgl3}]{
			Durch eine Äquivalenzumformung lässt sich die Differentialgleichungen in die beiden folgenden Formen bringen.
			\[
				y' + \tan (x)y = 0 \separate y' = -y\tan x
			\]
			\atiPoints[1]Damit handelt es sich bei dieser Differentialgleichung um eine gewöhnliche, lineare, homogene, separable Differentialgleichung 1.Ordnung.
			Durch Anwendung der Methode der Trennung der Variablen, erhält man dann die folgende Lösung.
			\[
				\frac{y'(x)}{y(x)} = -\tan x \implies \integral{y_0}{y}{\frac{1}{s}}{s} = \integral{x_0}{x}{-\tan s}{s}
			\]
			\[
				\implies \ln\absolute{\frac{y(x)}{y_0}} = \ln\absolute{\frac{\cos x}{\cos x_0}} \implies \absolute{y(x)} = \absolute{\frac{y_0}{\cos x_0}} \absolute{\cos x}\atiPoints[1]
			\]
		}
		\item[\localref{dgl4}]{
			Wir formen wieder die gegebene Differentialgleichung um und erhalten die folgenden beiden Ausdrücke.
			\[
				y'' = -xy' \separate y'' + xy' = 0
			\]
			\atiPoints[1]Damit handelt es sich bei dieser Differentialgleichung um eine gewöhnliche, lineare, homogene, nicht-separable Differentialgleichung 2.Ordnung.
			\atiPoints[+\frac{1}{2}]Durch die Substitution mit $z\define y'$ lässt sich zudem noch zeigen, dass sie in eine separable Differentialgleichung umgeformt werden kann.
		}
		\item[\localref{dgl5}]{
			\atiPoints[1]Es handelt sich um eine gewöhnliche, nicht-lineare, separable Differentialgleichung 1.Ordnung.
			Es lässt sich hier keine Homogenität definieren, da sie nicht linear ist.
			Die Methode der Trennung der Variablen liefert dann die folgende Lösung.
			\[
				\integral{x_0}{x}{y(s)y'(s)}{s} = \integral{x_0}{x}{s}{s} \implies \integral{y_0}{y(x)}{s}{s} = \frac{y^2(x)- y^2_0}{2} = \frac{x^2-x_0^2}{2}
			\]
			\[
				\implies y^2(x) = x^2 - x_0^2 + y_0^2 \implies y(x) = \pm\sqrt{x^2-x_0^2+y_0^2}\atiPoints[1]
			\]

		}
		\item[\localref{dgl6}]{
			Durch die Trennung von Zähler und Nenner lässt sich die Differentialgleichung in eine Form bringen, an der sich ihre Eigenschaften ablesen lassen.
			\[
				y' = (x+1)\frac{1}{y+2}
			\]
			\atiPoints[1]Es handelt sich um eine gewöhnliche, nicht-lineare, separable, Differentialgleichung 1.Ordnung.
			Die Methode der Trennung der Variablen liefert dann die folgende Lösung.
			\[
				\boxBrackets{y(x)+2}y'(x) = x+1 \implies \integral{y_0}{y(x)}{s+2}{s} = \integral{x_0}{x}{s+1}{s}
			\]
			\[
				\implies \frac{1}{2}\roundBrackets{\boxBrackets{y(x)+2}^2 - \boxBrackets{y_0 + 2}^2} = \frac{1}{2}\roundBrackets{(x+1)^2 - (x_0+1)^2}
			\]
			\[
				\implies \boxBrackets{y(x)+2}^2 = (x+1)^2 - (x_0+1)^2 + \roundBrackets{y_0+2}^2
			\]
			\[
				\implies y(x) = -2 \pm \sqrt{(x+1)^2 - (x_0+1)^2 + \roundBrackets{y_0+2}^2}\atiPoints[1]
			\]
		}
		\item[\localref{dgl7}]{
			Berechnet man den ersten Term auf der linken Seite dieser Gleichung, so ist es möglich die zweite binomische Formel zu verwenden.
			In diesem Falle erhält man das folgende Resultat.
			\[
				\absolute{y+a} + y'\sqrt{x^2-a^2} = \sqrt{x+a}
			\]
			\atiPoints[1]Damit handelt es sich bei dieser Differentialgleichung um eine gewöhnliche, nicht-lineare, nicht-separable Differentialgleichung 1.Ordnung.
		}
	\end{atiSubtaskSolutions}
\end{atiSolution}
	% \begin{atiTask}[
	title = Zwei separable Differentialgleichungen,
	language = Deutsch
]
	Lösen Sie die folgenden Differentialgleichungen mittels Trennung der Variablen.
	Überprüfen Sie Ihr Ergebnis, indem Sie die Probe durch Einsetzen Ihrer Lösung in die ursprüngliche Differentialgleichung durchführen.
	\begin{atiSubequations}
		% \medskip
		% \begin{minipage}[t]{0.5\textwidth}
		% \begin{multicols}{2}
		\item{
			\frac{1}{\cos x}\fdrv{y(x)}{x} = -\tan x \cdot y^{-2}
		}
		% \end{minipage}
		% \begin{minipage}{0.5\textwidth}
		\item{
			xyy' = y-1
		}
		% \end{minipage}
		% \end{multicols}
	\end{atiSubequations}
\end{atiTask}

	% \begin{atiTask}[
	topic = Gewöhnliche Differentialgleichungen,
	subtopic = Separable Differentialgleichungen,
	title = Homogene Differentialgleichungen,
	language = Deutsch
]
	Eine Funktion von zwei Variablen heißt homogen vom Grad $k$, wenn für einen beliebigen Parameter $\lambda$ das Folgende gilt.
	\[
		f(\lambda x, \lambda y) = \lambda^k f(x,y)
	\]
	Dementsprechend nennt man eine Differentialgleichungen der folgenden Form auch homogen, wenn $f$ und $g$ homogene Funktionen vom gleichen Grad sind.
	\[
		y' = \frac{f(x,y)}{g(x,y)}
	\]

	\begin{atiSubtasks}
		\item \label{subtask:homogene-differentialgleichung-a}
		Lösen Sie die folgende homogene Differentialgleichung.
		\[
			2xyy^\prime = 3y^2 - x^2
		\]
		Anleitung: Führen Sie eine neue Variable $z(x)$ gemäß $y(x)\definedby x\cdot z(x)$ ein und behandeln Sie die für $z(x)$ entstehende Differentialgleichung mit der Methode der Trennung der Variablen.

		\item
		Lösen Sie die folgende Differentialgleichung, die nicht homogen ist.
		\[
			y^\prime = \frac{y+x-2}{y-x+4}
		\]
		Schuld daran sind die beiden additiven Konstanten in Zähler und Nenner des Bruches auf der rechten Seite.
		Gehen Sie in zwei Schritten nach folgender Anleitung vor.
		\begin{atiItems}
			\item Führen Sie die neue Variablen $v\define y-y_0$ und $u\define x-x_0$ ein und bestimmen Sie $x_0$ und $y_0$, sodass die neue Differentialgleichung in den Variablen $u$ und $v$ homogen ist (Gleichungssystem mit zwei Unbekannten).

			\item Verfahren Sie mit der Substitution $v(u)\definedby u\cdot z(u)$ weiter, wie in Teilaufgabe \ref{subtask:homogene-differentialgleichung-a}.
		\end{atiItems}

		\item
		Machen Sie in beiden Fällen die Probe durch Einsetzen Ihrer Lösung $y(x)$ in die ursprüngliche Differentialgleichung.
	\end{atiSubtasks}
\end{atiTask}
% \begin{atiSolution}

% \end{atiSolution}
	% \begin{atiTask}[
	topic = Gewöhnliche Differentialgleichungen,
	subtopic = Separable Differentialgleichungen,
	title = Orthogonaltrajektorien,
	language = Deutsch
]
	\begin{atiSubtasks}
		\item {
			Skizzieren Sie die folgende Kurvenschar, wobei $c$ eine eine reelle Konstante darstellt.
			\[
				xy = c
			\]
			Bestimmen Sie dazu die Schar der Orthogonaltrajektorien ud tragen Sie diese in Ihre Skizze ein.
		}
		\item{
			Skizzieren Sie die folgende Kurvenschar und bestimmen Sie die Differentialgleichung, die dieser Kurvenschar genügt.
			Auch hier stellt $c$ eine reelle Konstante dar.
			\[
				y^2 = 4c(x+c)
			\]
			Zeigen Sie dann, dass diese Differentialgleichung die gleiche bleibt, wenn $y'$ durch $-\frac{1}{y'}$ ersetzt wird.
			Welche Schlussfolgerungen ziehen Sie aus dieser Eigenschaft?
		}
	\end{atiSubtasks}
\end{atiTask}
	% \begin{atiTask}[
	topic = Gewöhnliche Differentialgleichungen,
	subtopic = Separable Differentialgleichungen,
	title = Die Methode der Variablentrennung,
	language = deutsch,
]
	Lösen Sie die folgenden Differentialgleichungen durch Trennung der Variablen und bestimmen Sie gegebenenfalls die Integrationskonstante, sodass die nebenstehenden Anfangsbedingungen erfüllt sind.
	\begin{atiSubequations}
		\item{
			\label{dgl-1}
			y' = \frac{x e^{-y}}{x^2 + 1} \separate y(1) = 0
		}
		\item{
			\label{dgl-2}
			xyy' = \frac{x^2 + 2}{y-1}
		}
		\item{
			\label{dgl-3}
			y' = \frac{x+y}{x+y+2} \separate y(1) = -1
		}
	\end{atiSubequations}
	Machen Sie in allen Fällen die Probe durch Einsetzen Ihrer Lösung in die ursprüngliche Differentialgleichung. Dies ist auch dann verlangt, wenn die Lösung nur in impliziter Form angebbar ist.

	\begin{atiNote}
		Führen Sie in Teilaufgabe \ref{dgl-3} die neue Variable $z(x)\define x + y(x)$ ein.
	\end{atiNote}
\end{atiTask}

\begin{atiSolution}
	\begin{atiSubtaskSolutions}
		\item[\ref{dgl-1}]{
			Separieren Sie $x$ und $y(x)$ auf jeweils eine Seite der Differentialgleichung und integrieren Sie die erhaltene Gleichung.
			\[
				y'(x) = \frac{xe^{-y(x)}}{x^2 + 1} \implies e^{y(x)} y'(x) = \frac{x}{x^2 + 1}
			\]
			\[
				\implies \integral{}{}{e^{y(x)}y'(x)}{x} = \integral{}{}{\frac{x}{x^2+1}}{x}
				\atiPoints[1]
			\]
			Lösen Sie das Integral mithilfe einer logarithmischen Integration oder durch Substitution, indem Sie $x^2$ durch eine geeignete Variable ersetzen.
			\[
				\integral{}{}{e^y}{y} = \frac{1}{2} \integral{}{}{\frac{2x}{x^2+1}}{x}
			\]
			\[
				\implies e^{y(x)} = \frac{1}{2}\ln(x^2 + 1) + C = \ln\sqrt{x^2 + 1} + C
				\atiPoints[1]
			\]
			Notieren Sie die explizite Lösung durch die Anwendung von $\ln$.
			\[
				y(x) = \ln\boxb{\ln\curvb{A\sqrt{x^2+1}}} \separate A\define e^C
				\atiPoints[1]
			\]
			Fordern Sie nun $y(1)\demand 0$, bestimmen Sie die Konstante $A$ und setzen Sie die erhaltene Lösung in die explizite allgemeine Form ein.
			\[
				y(1) \demand 0 \implies 1 = \ln \curvb{A\sqrt{2}} \implies A = \frac{\sqrt{2}}{2}e
				\atiPoints[1]
			\]
			\[
				y(x) = \ln\boxb{\ln\curvb{\frac{e}{2}\sqrt{2(x^2 + 1)}}} = \ln\boxb{1 + \ln\curvb{\frac{1}{2}\sqrt{2(x^2+1)}}}
				\atiPoints[1]
			\]

			Sei $y\definedby \ln u$ mit $u(x) = \ln\curvb{A\sqrt{x^2+1}}$.
			Dann erhält man durch die Anwendung der Kettenregel die folgende Aussage.
			\[
				y'(x) = u'(x) \ln' u(x) = \frac{1}{A\sqrt{x^2+1}} \cdot A \cdot \frac{2x}{2\sqrt{x^2+1}} \cdot \frac{1}{u(x)} = e^{-y(x)} \frac{x}{x^2+1}
				\atiPoints[1]
			\]
		}

		\item[\ref{dgl-2}]{
			Separieren Sie $x$ und $y(x)$ wieder auf jeweils eine Seite der Differentialgleichung
			\[
				xy(x)y'(x) = \frac{x^2+2}{y(x)-1} \implies y(x)\boxb{y(x)-1}y'(x) = \frac{x^2+2}{x}
				\atiPoints[1]
			\]
			Integrieren Sie die rechte Seite der erhaltenen Gleichung durch Polynomintegration und der Umkehrregel.
			\[
				\integral{}{}{\boxb{y^2(x)-y(x)}y'(x)}{x} = \integral{}{}{\curvb{x+\frac{2}{x}}}{x}
			\]
			\[
				\implies \integral{}{}{y^2-y}{y} = \frac{x^2}{2} + 2\ln\abs{x}
				\atiPoints[1]
			\]
			Lösen Sie nun auch das Integral der rechten Seite durch Polynomintegration und notieren Sie die allgemeine Lösung in impliziter Form.
			\[
				\frac{y^3}{3} - \frac{y^2}{2} = \frac{x^2}{2} + 2\ln\abs{x} + C
			\]
			\[
				\implies 2y^3 - 3y^2 = 3x^2 + 12 \ln \abs{x} + D
				\atiPoints[1]
			\]

			Durch implizite Ableitung der allgemeinen Form erhalten Sie Folgendes.
			\[
				6y^2(x)y'(x) - 6y(x)y'(x) = 6x + \frac{12}{x} \implies y(x)\boxb{y(x)-1}y'(x) = x + \frac{2}{x}
			\]
			\[
				\implies xy(x)\boxb{y(x)-1}y'(x) = x^2+2
				\atiPoints[1]
			\]
		}

		\item[\ref{dgl-3}]{
			Definieren Sie $z(x)\define x + y(x)$ und bestimmen Sie die Ableitung von $z$.
			\[
				z'(x) = 1 + y'(x) \implies y'(x) = z'(x) - 1
			\]
			Substituieren Sie nun $x+y(x)$ in der Differentialgleichung durch $z(x)$.
			\[
				y'(x) = \frac{x+y(x)}{x+y(x)+2} \implies z'(x) - 1 = \frac{z}{z+2}
				\atiPoints[1]
			\]
			Führen Sie für die erhaltene Differentialgleichung das Verfahren der Trennung der Variablen durch.
			Separieren Sie $z(x)$ und $x$ auf jeweils eine Seite und integrieren Sie die erhaltene Gleichung.
			\[
				z'(x) = \frac{z(x)}{z(x)+2} + 1 = \frac{2z(x)+2}{z+2} = 2\,\frac{z(x)+1}{z(x)+2}
				\atiPoints[1]
			\]
			\[
				\implies \integral{}{}{\frac{z(x)+2}{z(x)+1}z'(x)}{x} = \integral{}{}{\curvb{1 + \frac{1}{z+1}}}{z} = \integral{}{}{2}{x}
			\]
			\[
				\implies z(x) + \ln\abs{z(x)+1} = 2x + C
				\atiPoints[1]
			\]
			Führen Sie die Resubstitution durch und geben Sie die allgemeine Lösung in impliziter Form an.
			\[
				x + y(x) + \ln\abs{x+ y(x) + 1} = 2x + C
			\]
			\[
				\implies x+y(x)+1 = \exp(x-y(x)+C) = Ae^{x-y(x)} \separate A\define e^C
			\]
			\[
				\implies x+y(x) = Ae^{x-y(x)} -1
				\atiPoints[1]
			\]
			Fordern Sie die gegebenen Anfangsbedingungen und bestimmen Sie die Konstante $A$.
			\[
				y(1)\demand -1 \implies 0=Ae^2-1 \implies A= e^{-2}
			\]
			\[
				\implies x+y(x) = \frac{e^{x-y(x)}}{e^2}-1 = e^{x-y(x)-2}-1
				\atiPoints[1]
			\]

			Auch hier ist wieder eine implizite Ableitung notwendig.
			\[
				1+y'(x) = Ae^{x-y(x)} \boxb{1-y'(x)}
			\]
			Durch Verwendung der allgemeinen Lösung erhalten Sie für die Konstante $A$ den folgenden Ausdruck.
			\[
				A = e^{y(x)-x}\boxb{x+y(x)+1}
			\]
			Das Einsetzen dieser Gleichung resultiert dann in der gewünschten Differentialgleichung.
			\[
				1+y'(x) = e^{y(x)-x}\boxb{x+y(x)+1} e^{x-y(x)} \boxb{1-y'(x)}
			\]
			\[
				\implies 1+y'(x) = x+y(x)+1 - \boxb{x+y(x)+1}y'(x)
			\]
			\[
				\implies y'(x)\boxb{x+y(x)+2} = x+y(x)
				\atiPoints[1]
			\]
		}
	\end{atiSubtaskSolutions}
\end{atiSolution}

	% \begin{atiTask}[
	title = Die Methode der Variation der Konstanten I,
	topic = Gewöhnliche Differentialgleichungen,
	subtopic = Lineare Differentialgleichungen 1. Ordnung,
	language = Deutsch,
]
	Lösen Sie die folgenden Differentialgleichungen mithilfe der Methode der Variation der Konstanten.
	\begin{atiSubequations}
		\item{
			\frac{y'}{x^3}-\frac{2y}{x^4} = \sin x
		}
		\item{
			y' \tan x + y = \sin x
		}
	\end{atiSubequations}
	Machen Sie in allen Fällen die Probe durch Einsetzen Ihrer Lösung in die ursprüngliche Differentialgleichung.
\end{atiTask}
	% \begin{atiTask}[
	title = Die Methode der Variation der Konstanten II,
	topic = Gewöhnliche Differentialgleichungen,
	subtopic = Lineare Differentialgleichungen 1. Ordnung,
	language = Deutsch
]
	Lösen Sie die folgenden Differentialgleichungen mithilfe der Methode der Variation der Konstanten und bestimmen Sie eine spezielle Lösung, sodass die nebenstehenden Anfangsbedingungen erfüllt sind.
	\begin{atiSubequations}
		\item{
			2y' = y + 2\sin x \separate y(0) = -1
		}
		\item{
			(x+1)y' = 2y + (x+1)^\frac{5}{2} \separate y(0) = 3
		}
	\end{atiSubequations}
	Machen Sie in allen Fällen die Probe durch Einsetzen der allgemeinen Lösung in die ursprüngliche Differentialgleichung.
\end{atiTask}
	% \begin{atiTask}[
	title = Eine nichtlineare Differentialgleichung
]
	Lösen Sie die folgende nichtlineare Differentialgleichung, indem Sie diese auf eine lineare Differentialgleichung durch die Substitution $z = y^{-2}$ zurückführen und mithilfe der Methode der Variation der Konstanten behandeln.
	\[
		y' - y + 3x^2y^3 = 0
	\]
	Machen Sie die Probe durch Einsetzen Ihrer Lösung in die ursprüngliche Differentialgleichung.
\end{atiTask}

	% \begin{atiTask}[
	title = Zwei exakte Differentialgleichungen,
	topic = Gewöhnliche Differentialgleichungen,
	subtopic = Exakte Differentialgleichungen: Der integrierende Faktor,
	language = Deutsch,
]
	Weisen Sie nach, dass die beiden folgenden Differentialgleichungen exakt sind und konstruieren Sie deren Lösung in \enquote{vollständigen Differentialen}.
	\begin{atiSubequations}
		\item{
			\frac{y^2}{x} + 2yy'\ln x = 0
		}
		\item{
			\curvb{8y - x^2y}y' + \curvb{x - xy^2} = 0
		}
	\end{atiSubequations}
	Machen Sie in beiden Fällen die Probe durch Einsetzen der Lösung in die ursprüngliche Differentialgleichung.
\end{atiTask}
	% \begin{atiTask}[
	title = Eine nichtexakte Differentialgleichung,
	topic = Gewöhnliche Differentialgleichungen,
	subtopic = Exakte Differentialgleichungen: Der integrierende Faktor,
	language = Deutsch,
]
	\begin{atiSubtasks}
		\item{
			Weisen Sie nach, dass die folgende Differentialgleichung nicht exakt ist.
			\[
				x^2y' - xy = \frac{2}{x}
			\]
		}
		\item{
			Bestimmen Sie einen integrierenden Faktor $\lambda$, der nur von der Variablen $x$ abhängt, und überzeugen Sie sich, dass dieser die Differentialgleichung exakt macht.
		}
		\item{
			Lösen Sie die Differentialgleichung in \enquote{vollständigen Differentialen} und machen Sie die Probe durch Einsetzen Ihrer Lösung in die ursprüngliche Differentialgleichung.
		}
		\item{
			Geben Sie diejenige spezielle Lösung an, die der folgenden Bedingung genügt.
			\[
				y(1) = 2
			\]
		}
	\end{atiSubtasks}
\end{atiTask}
	% \begin{atiTask}[
	title = Integrierende Faktoren,
	topic = Gewöhnliche Differentialgleichungen,
	subtopic = Exakte Differentialgleichungen: Der integrierende Faktor,
	language = Deutsch
]
	\begin{atiSubtasks}
		\item{
			Die folgende Differentialgleichung sei gegeben.
			\[
				A(x,y) + B(x,y)y' = 0
			\]
			Zeigen Sie durch Verwendung der Integrabilitätsbedingung die beiden folgenden Aussagen.
			\begin{atiItems}
				\item{
					Kann der folgende Ausdruck als eine Funktion $f$ der Variablen $z(x,y)\define xy$ geschrieben werden, so hängt auch der integrierende Faktor $\lambda$ nur von dieser Variable $z$ ab.
					\[
						\frac{1}{xA(x,y)-yB(x,y)}\curvb{\partial_x B(x,y) - \partial_y A(x,y)}
					\]
					Geben Sie den Zusammenhang von $\lambda$ und $f$ an.
				}
				\item{
					Kann der folgende Ausdruck als eine Funktion $g$ der Variablen $w(x,y)\define x + y$ geschrieben werden, so hängt auch der integrierende Faktor $\lambda$ nur von dieser Variable $z$ ab.
					\[
						\frac{1}{A(x,y)-B(x,y)}\curvb{\partial_x B(x,y) - \partial_y A(x,y)}
					\]
					Geben Sie den Zusammenhang von $\lambda$ und $g$ an.

				}
			\end{atiItems}
		}
		\item{
			Eine der beiden zuvor genannten Eigenschaften trifft auf eine der beiden folgenden Differentialgleichungen zu.
			Finden Sie diesen Fall heraus und berechnen Sie einen integrierenden Faktor mit dem Ergebnis aus dem vorherigen Aufgabenteil.
			\begin{atiSubequations}
				\item{
					(xy-1) + \curvb{x^2-xy}y' = 0
				}
				\item{
					y + \curvb{x - 2x^2y^3}y' = 0
				}
			\end{atiSubequations}
			Lösen Sie die Differentialgleichung mit diesem integrierenden Faktor und machen Sie anschließend die Probe anhand der ursprünglichen Differentialgleichung.
			Die Lösung der verbleibenden Differentialgleichung ist nicht verlangt.
		}
	\end{atiSubtasks}
\end{atiTask}

	% \begin{atiTask}[
	title = Die charakteristische Gleichung,
	topic = Gewöhnliche Differentialgleichungen,
	subtopic = Die lineare homogene Differentialgleichung 2. Ordnung mit konstanten Koeffizienten,
	language = Deutsch,
]
	Konstruieren Sie für jede der beiden folgenden Differentialgleichungen deren allgemeine Lösung und bestimmen Sie die spezielle Lösung, die den nebenstehenden Anfangsbedingungen genügt.
	\begin{atiSubequations}
		\item{
			y'' - 2y' + y = 0 \separate y(1) = 2 \separate y(0) = 1
		}
		\item{
			y'' + 6y' + 13y = 0 \separate y(0) = 3 \separate y'(0) = 7
		}
	\end{atiSubequations}
	Machen Sie in beiden Fällen für die allgemeine Lösung die Probe durch Einsetzen in die ursprüngliche Differentialgleichung.
\end{atiTask}
	% \begin{atiTask}[
	title = Die homogene Euler-Gleichung,
	topic = Gewöhnliche Differentialgleichungen,
	subtopic = Die lineare homogene Differentialgleichung 2. Ordnung mit konstanten Koeffizienten,
	language = Deutsch,
]
	Die Eulersche Differentialgleichung ist durch die folgende Form gegeben.
	Dabei stellen die Koeffizienten $a$,$b$ und $c$ reelle Konstanten dar.
	\[
		ax^2y'' + bxy' + cy = 0
	\]
	\begin{atiSubtasks}
		\item{
			Überführen Sie diese Differentialgleichung mithilfe der Substitution $x = e^{t(x)}$ in eine Differentialgleichung mit konstanten Koeffizienten.
		}
		\item{
			Ein wichtiges Beispiel für die Potentialtheorie stellt die folgende Differentialgleichung dar.
			Dabei beschreibt $n$ eine nichtnegative reelle Konstante, $r$ eine nichtnegative reelle Variable und $R$ eine Funktion, welche von $r$ abhängt.
			\[
				\nfdrv[2]{R}{r} + \frac{2}{r}\nfdrv{R}{r} - \frac{n(n+1)}{r^2}R = 0
			\]
			\begin{atiSubsubtasks}
				\item{
					Behandeln Sie die genannte Differentialgleichung nach der zuvor beschriebenen Methode.
				}
				\item{
					Konstruieren Sie die allgemeine Lösung $R$ der entstehenden Differentialgleichung mit konstanten Koeffizienten.
				}
				\item{
					Geben Sie diejenige spezielle Lösung an, welche die folgende Bedingung erfüllt.
					\[
						R\conv[r\conv\infty]0
					\]
				}
			\end{atiSubsubtasks}
		}
		\item{
			\textbf{Zusatz:}
			Lösen Sie mit derselben Methode die folgende Differentialgleichung.
			\[
				x^2y'' - xy' + 10y = 0
			\]
		}
	\end{atiSubtasks}
\end{atiTask}
	% \begin{atiTask}[
	title = Die Wronski-Determinante
]
	Gegeben sei die folgende Differentialgleichung, wobei $p$ und $q$ Funktionen in Abhängigkeit von $x$ darstellen.
	Weiterhin seien $y_1$ und $y_2$ zwei Lösungen dieser Gleichung.
	\[
		y'' + p(x)y' + q(x)y = 0
	\]
	\begin{atiSubtasks}
		\item{\locallabel{a}
			Zeigen Sie, dass die Wronski-Determinante $W$ der folgenden separablen Differentialgleichung genügt und lösen Sie diese Gleichung durch Trennung der Variablen.
			\[
				W' = -p(x)\cdot W
			\]
			Damit ist es möglich die Wronski-Determinante direkt aus der zugehörigen Differentialgleichung ohne Kenntnis der Lösung dieser zu bestimmen.
		}
		\item{
			Es sei die Lösung $y_1$ bereits bekannt.
			Interpretieren Sie die folgende Gleichung als inhomogene Differentialgleichung erster Ordnung für $y_2$ und lösen Sie diese durch Variation der Konstanten.
			\[
				y_1(x)y_2'(x)-y_2(x)y_1'(x) = W(x)
			\]
			So können Sie die zweite Fundamentallösung aus der Kenntnis der ersten Fundamentallösung und der Wronski-Determinante bestimmen.
		}
		\item{
			Betrachten Sie die folgende Differentialgleichung mit den reellen konstanten Koeffizienten $a$, $b$ und $c$.
			\[
				ay'' + by' + cy = 0
			\]
			\begin{atiSubsubtasks}
				\item{
					Bestimmen Sie die Wronski-Determinante dieser Differentialgleichung mit der in \localref{a} beschriebenen Methode.
				}
				\item{
					Nehmen Sie an, dass die folgende Lösung bekannt ist, und bestimmen Sie durch Verwendung des zuvor beschriebenen Verfahrens die zweite Fundamentallösung.
					\[
						y_1(x) = e^{\lambda_1 x}
						\separate
						\lambda_1 \define -\frac{b}{2a} + \frac{1}{2a}\sqrt{b^2-4ac}
					\]
				}
			\end{atiSubsubtasks}
		}
		\item{
			\textbf{Für Interessierte:}
			Gegeben sei die folgende Differentialgleichung mit der reellen Konstanten $m$.
			\[
				y'' + \frac{1}{x} y' - \frac{m^2}{x}y = 0
			\]
			\begin{atiSubsubtasks}
				\item{
					Bestimmen Sie die Wronski-Determinante dieser Differentialgleichung.
				}
				\item{
					Überzeugen Sie sich, dass die folgende Funktion eine Lösung dieser Differentialgleichung ist und berechnen Sie nach dem zuvor beschriebenen Verfahren die zweite Fundamentallösung.
					\[
						y_1(x) = x^m
					\]
					Machen Sie die Probe für diese zweite Lösung und geben Sie auch die allgemeine Lösung der Differentialgleichung an.
				}
			\end{atiSubsubtasks}
		}
	\end{atiSubtasks}
\end{atiTask}


	% \printsolutions

\end{document}
