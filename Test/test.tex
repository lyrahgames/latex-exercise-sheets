\documentclass[a4paper,fleqn, 11pt]{article}

\usepackage[utf8]{inputenc}
\usepackage[T1]{fontenc}
\usepackage[ngerman]{babel}
\usepackage[bottom=25mm,left=30mm,right=30mm,bottom=35mm]{geometry}
\usepackage{times}
\linespread{1.15}

\usepackage{turnthepage}
\renewcommand{\turnthepage}{\it bitte wenden}



\usepackage{import}

\usepackage{multicol}

\usepackage{float}
\usepackage{import}
\usepackage{graphicx}

\usepackage[footnotesize,hang]{caption}


\usepackage[pgfkeys,custom]{ati}
% \SetupExSheets{solution/print=true, question/type=exam}
% \usepackage{utilities}
% \usepackage{uniinput}
\allowdisplaybreaks

\begin{document}
  \[
    \set{1}{}
    \separate
    \set{2}{3}
  \]
  \[
    \define
    \separate
    \reverseDefine
    \separate
    \demand
    \separate
    \function{f}{X}{Y}
  \]
  \[
    \setReal
    \separate
    \setNatural
    \separate
    \setInteger
    \separate
    \setRational
    \separate
    \setComplex
  \]
  \[
    \roundBrackets{a}
    \separate
    \curlyBrackets{b}
    \separate
    \boxBrackets{c}
    \separate
    \angleBrackets{d}
    \separate
    \floorBrackets{e}
    \separate
    \ceilBrackets{f}
  \]
  \[
    \absolute{a}
    \separate
    \norm{b}
    \separate
    \inverse{c}
  \]
  \[
    \dotProduct{a}{b}
    \separate
    \scalarProduct{c}{d}
    \separate
    \crossProduct{e}{f}
  \]
  \[
    \timeDerivative{a}
    \separate
    \timeSecondDerivative{b}
    \separate
    \timeThirdDerivative{c}
    \separate
    \leibnizDerivative{y(t)}{t}
    \separate
    \leibnizDerivativeOperator{t}{y(t)}
    \separate
    \leibnizDerivativeOperator{t}{}y(t)
    \separate
    \leibnizDerivative[3]{y(t)}{t}
  \]
  \[
    \appendValue{x}{a}
    \separate
    \leibnizDerivativeOperatorValue{a}{b}{c}
    \separate
    \leibnizPartialDerivative{f}{x}
    \separate
    \leibnizPartialDerivativeOperator{x}{f}
    \separate
    \leibnizPartialDerivativeOperatorValue{x}{f}{a}
    \separate
    \leibnizPartialDerivative[n]{f}{x}
  \]
  \[
    f\composition g
    \separate
    \infinitesimal f
  \]
  \[
      \gradient f
      \separate
      \curl f
      \separate
      \divergence f
      \separate
      \laplacian f
      \separate
      \nabla f
      \separate
      \nabla\cdot f
      \separate
      \nabla\times f
  \]
  \[
    a\appendUnit{m}
    \separate
    b\appendUnit{km\cdot s^{-2}}
    \separate
    c\appendUnit{\frac{m}{s^2}}
  \]
  \begin{equation}
 	\vector{x}
  	\separate
  	\vectorX
  \end{equation}
  \begin{atiTask}
  	this is a test!
  \end{atiTask}
\end{document}