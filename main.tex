\documentclass[a4paper,fleqn]{article}

\usepackage[utf8]{inputenc}
\usepackage[T1]{fontenc}
\usepackage[ngerman]{babel}
\usepackage[bottom=25mm,left=30mm,right=30mm,bottom=35mm]{geometry}
\usepackage{times}
\linespread{1.15}

\usepackage{turnthepage}
\renewcommand{\turnthepage}{\it bitte wenden}

\usepackage[pgfkeys,custom]{ati}
\SetupExSheets{solution/print=true, question/type=exam}

\allowdisplaybreaks

\usepackage{import}

\begin{document}

	% \pagestyle{empty}

	\section*{\centering Mathematische Methoden der Physik I \\ Übungsserie 1}

	\begin{atiTask}[
	topic = Gewöhnliche Differentialgleichungen,
	subtopic = Separable Differentialgleichungen,
	title = Homogene Differentialgleichungen,
	language = Deutsch
]
	Eine Funktion von zwei Variablen heißt homogen vom Grad $k$, wenn für einen beliebigen Parameter $\lambda$ das Folgende gilt.
	\[
		f(\lambda x, \lambda y) = \lambda^k f(x,y)
	\]
	Dementsprechend nennt man eine Differentialgleichungen der folgenden Form auch homogen, wenn $f$ und $g$ homogene Funktionen vom gleichen Grad sind.
	\[
		y' = \frac{f(x,y)}{g(x,y)}
	\]

	\begin{atiSubtasks}
		\item \label{subtask:homogene-differentialgleichung-a}
		Lösen Sie die folgende homogene Differentialgleichung.
		\[
			2xyy^\prime = 3y^2 - x^2
		\]
		Anleitung: Führen Sie eine neue Variable $z(x)$ gemäß $y(x)\definedby x\cdot z(x)$ ein und behandeln Sie die für $z(x)$ entstehende Differentialgleichung mit der Methode der Trennung der Variablen.

		\item
		Lösen Sie die folgende Differentialgleichung, die nicht homogen ist.
		\[
			y^\prime = \frac{y+x-2}{y-x+4}
		\]
		Schuld daran sind die beiden additiven Konstanten in Zähler und Nenner des Bruches auf der rechten Seite.
		Gehen Sie in zwei Schritten nach folgender Anleitung vor.
		\begin{atiItems}
			\item Führen Sie die neue Variablen $v\define y-y_0$ und $u\define x-x_0$ ein und bestimmen Sie $x_0$ und $y_0$, sodass die neue Differentialgleichung in den Variablen $u$ und $v$ homogen ist (Gleichungssystem mit zwei Unbekannten).

			\item Verfahren Sie mit der Substitution $v(u)\definedby u\cdot z(u)$ weiter, wie in Teilaufgabe \ref{subtask:homogene-differentialgleichung-a}.
		\end{atiItems}

		\item
		Machen Sie in beiden Fällen die Probe durch Einsetzen Ihrer Lösung $y(x)$ in die ursprüngliche Differentialgleichung.
	\end{atiSubtasks}
\end{atiTask}
% \begin{atiSolution}

% \end{atiSolution}
	\begin{atiTask}[
	topic = Gewöhnliche Differentialgleichungen,
	subtopic = Separable Differentialgleichungen,
	title = Orthogonaltrajektorien,
	language = Deutsch
]
	\begin{atiSubtasks}
		\item {
			Skizzieren Sie die folgende Kurvenschar, wobei $c$ eine eine reelle Konstante darstellt.
			\[
				xy = c
			\]
			Bestimmen Sie dazu die Schar der Orthogonaltrajektorien ud tragen Sie diese in Ihre Skizze ein.
		}
		\item{
			Skizzieren Sie die folgende Kurvenschar und bestimmen Sie die Differentialgleichung, die dieser Kurvenschar genügt.
			Auch hier stellt $c$ eine reelle Konstante dar.
			\[
				y^2 = 4c(x+c)
			\]
			Zeigen Sie dann, dass diese Differentialgleichung die gleiche bleibt, wenn $y'$ durch $-\frac{1}{y'}$ ersetzt wird.
			Welche Schlussfolgerungen ziehen Sie aus dieser Eigenschaft?
		}
	\end{atiSubtasks}
\end{atiTask}
	\begin{atiTask}[
	topic = Gewöhnliche Differentialgleichungen,
	subtopic = Separable Differentialgleichungen,
	title = Die Methode der Variablentrennung,
	language = deutsch,
]
	Lösen Sie die folgenden Differentialgleichungen durch Trennung der Variablen und bestimmen Sie gegebenenfalls die Integrationskonstante, sodass die nebenstehenden Anfangsbedingungen erfüllt sind.
	\begin{atiSubequations}
		\item{
			\label{dgl-1}
			y' = \frac{x e^{-y}}{x^2 + 1} \separate y(1) = 0
		}
		\item{
			\label{dgl-2}
			xyy' = \frac{x^2 + 2}{y-1}
		}
		\item{
			\label{dgl-3}
			y' = \frac{x+y}{x+y+2} \separate y(1) = -1
		}
	\end{atiSubequations}
	Machen Sie in allen Fällen die Probe durch Einsetzen Ihrer Lösung in die ursprüngliche Differentialgleichung. Dies ist auch dann verlangt, wenn die Lösung nur in impliziter Form angebbar ist.

	\begin{atiNote}
		Führen Sie in Teilaufgabe \ref{dgl-3} die neue Variable $z(x)\define x + y(x)$ ein.
	\end{atiNote}
\end{atiTask}

\begin{atiSolution}
	\begin{atiSubtaskSolutions}
		\item[\ref{dgl-1}]{
			Separieren Sie $x$ und $y(x)$ auf jeweils eine Seite der Differentialgleichung und integrieren Sie die erhaltene Gleichung.
			\[
				y'(x) = \frac{xe^{-y(x)}}{x^2 + 1} \implies e^{y(x)} y'(x) = \frac{x}{x^2 + 1}
			\]
			\[
				\implies \integral{}{}{e^{y(x)}y'(x)}{x} = \integral{}{}{\frac{x}{x^2+1}}{x}
				\atiPoints[1]
			\]
			Lösen Sie das Integral mithilfe einer logarithmischen Integration oder durch Substitution, indem Sie $x^2$ durch eine geeignete Variable ersetzen.
			\[
				\integral{}{}{e^y}{y} = \frac{1}{2} \integral{}{}{\frac{2x}{x^2+1}}{x}
			\]
			\[
				\implies e^{y(x)} = \frac{1}{2}\ln(x^2 + 1) + C = \ln\sqrt{x^2 + 1} + C
				\atiPoints[1]
			\]
			Notieren Sie die explizite Lösung durch die Anwendung von $\ln$.
			\[
				y(x) = \ln\boxb{\ln\curvb{A\sqrt{x^2+1}}} \separate A\define e^C
				\atiPoints[1]
			\]
			Fordern Sie nun $y(1)\demand 0$, bestimmen Sie die Konstante $A$ und setzen Sie die erhaltene Lösung in die explizite allgemeine Form ein.
			\[
				y(1) \demand 0 \implies 1 = \ln \curvb{A\sqrt{2}} \implies A = \frac{\sqrt{2}}{2}e
				\atiPoints[1]
			\]
			\[
				y(x) = \ln\boxb{\ln\curvb{\frac{e}{2}\sqrt{2(x^2 + 1)}}} = \ln\boxb{1 + \ln\curvb{\frac{1}{2}\sqrt{2(x^2+1)}}}
				\atiPoints[1]
			\]

			Sei $y\definedby \ln u$ mit $u(x) = \ln\curvb{A\sqrt{x^2+1}}$.
			Dann erhält man durch die Anwendung der Kettenregel die folgende Aussage.
			\[
				y'(x) = u'(x) \ln' u(x) = \frac{1}{A\sqrt{x^2+1}} \cdot A \cdot \frac{2x}{2\sqrt{x^2+1}} \cdot \frac{1}{u(x)} = e^{-y(x)} \frac{x}{x^2+1}
				\atiPoints[1]
			\]
		}

		\item[\ref{dgl-2}]{
			Separieren Sie $x$ und $y(x)$ wieder auf jeweils eine Seite der Differentialgleichung
			\[
				xy(x)y'(x) = \frac{x^2+2}{y(x)-1} \implies y(x)\boxb{y(x)-1}y'(x) = \frac{x^2+2}{x}
				\atiPoints[1]
			\]
			Integrieren Sie die rechte Seite der erhaltenen Gleichung durch Polynomintegration und der Umkehrregel.
			\[
				\integral{}{}{\boxb{y^2(x)-y(x)}y'(x)}{x} = \integral{}{}{\curvb{x+\frac{2}{x}}}{x}
			\]
			\[
				\implies \integral{}{}{y^2-y}{y} = \frac{x^2}{2} + 2\ln\abs{x}
				\atiPoints[1]
			\]
			Lösen Sie nun auch das Integral der rechten Seite durch Polynomintegration und notieren Sie die allgemeine Lösung in impliziter Form.
			\[
				\frac{y^3}{3} - \frac{y^2}{2} = \frac{x^2}{2} + 2\ln\abs{x} + C
			\]
			\[
				\implies 2y^3 - 3y^2 = 3x^2 + 12 \ln \abs{x} + D
				\atiPoints[1]
			\]

			Durch implizite Ableitung der allgemeinen Form erhalten Sie Folgendes.
			\[
				6y^2(x)y'(x) - 6y(x)y'(x) = 6x + \frac{12}{x} \implies y(x)\boxb{y(x)-1}y'(x) = x + \frac{2}{x}
			\]
			\[
				\implies xy(x)\boxb{y(x)-1}y'(x) = x^2+2
				\atiPoints[1]
			\]
		}

		\item[\ref{dgl-3}]{
			Definieren Sie $z(x)\define x + y(x)$ und bestimmen Sie die Ableitung von $z$.
			\[
				z'(x) = 1 + y'(x) \implies y'(x) = z'(x) - 1
			\]
			Substituieren Sie nun $x+y(x)$ in der Differentialgleichung durch $z(x)$.
			\[
				y'(x) = \frac{x+y(x)}{x+y(x)+2} \implies z'(x) - 1 = \frac{z}{z+2}
				\atiPoints[1]
			\]
			Führen Sie für die erhaltene Differentialgleichung das Verfahren der Trennung der Variablen durch.
			Separieren Sie $z(x)$ und $x$ auf jeweils eine Seite und integrieren Sie die erhaltene Gleichung.
			\[
				z'(x) = \frac{z(x)}{z(x)+2} + 1 = \frac{2z(x)+2}{z+2} = 2\,\frac{z(x)+1}{z(x)+2}
				\atiPoints[1]
			\]
			\[
				\implies \integral{}{}{\frac{z(x)+2}{z(x)+1}z'(x)}{x} = \integral{}{}{\curvb{1 + \frac{1}{z+1}}}{z} = \integral{}{}{2}{x}
			\]
			\[
				\implies z(x) + \ln\abs{z(x)+1} = 2x + C
				\atiPoints[1]
			\]
			Führen Sie die Resubstitution durch und geben Sie die allgemeine Lösung in impliziter Form an.
			\[
				x + y(x) + \ln\abs{x+ y(x) + 1} = 2x + C
			\]
			\[
				\implies x+y(x)+1 = \exp(x-y(x)+C) = Ae^{x-y(x)} \separate A\define e^C
			\]
			\[
				\implies x+y(x) = Ae^{x-y(x)} -1
				\atiPoints[1]
			\]
			Fordern Sie die gegebenen Anfangsbedingungen und bestimmen Sie die Konstante $A$.
			\[
				y(1)\demand -1 \implies 0=Ae^2-1 \implies A= e^{-2}
			\]
			\[
				\implies x+y(x) = \frac{e^{x-y(x)}}{e^2}-1 = e^{x-y(x)-2}-1
				\atiPoints[1]
			\]

			Auch hier ist wieder eine implizite Ableitung notwendig.
			\[
				1+y'(x) = Ae^{x-y(x)} \boxb{1-y'(x)}
			\]
			Durch Verwendung der allgemeinen Lösung erhalten Sie für die Konstante $A$ den folgenden Ausdruck.
			\[
				A = e^{y(x)-x}\boxb{x+y(x)+1}
			\]
			Das Einsetzen dieser Gleichung resultiert dann in der gewünschten Differentialgleichung.
			\[
				1+y'(x) = e^{y(x)-x}\boxb{x+y(x)+1} e^{x-y(x)} \boxb{1-y'(x)}
			\]
			\[
				\implies 1+y'(x) = x+y(x)+1 - \boxb{x+y(x)+1}y'(x)
			\]
			\[
				\implies y'(x)\boxb{x+y(x)+2} = x+y(x)
				\atiPoints[1]
			\]
		}
	\end{atiSubtaskSolutions}
\end{atiSolution}

	\begin{atiTask}[
	title = Die Methode der Variation der Konstanten I,
	topic = Gewöhnliche Differentialgleichungen,
	subtopic = Lineare Differentialgleichungen 1. Ordnung,
	language = Deutsch,
]
	Lösen Sie die folgenden Differentialgleichungen mithilfe der Methode der Variation der Konstanten.
	\begin{atiSubequations}
		\item{
			\frac{y'}{x^3}-\frac{2y}{x^4} = \sin x
		}
		\item{
			y' \tan x + y = \sin x
		}
	\end{atiSubequations}
	Machen Sie in allen Fällen die Probe durch Einsetzen Ihrer Lösung in die ursprüngliche Differentialgleichung.
\end{atiTask}
	\begin{atiTask}[
	title = Die Methode der Variation der Konstanten II,
	topic = Gewöhnliche Differentialgleichungen,
	subtopic = Lineare Differentialgleichungen 1. Ordnung,
	language = Deutsch
]
	Lösen Sie die folgenden Differentialgleichungen mithilfe der Methode der Variation der Konstanten und bestimmen Sie eine spezielle Lösung, sodass die nebenstehenden Anfangsbedingungen erfüllt sind.
	\begin{atiSubequations}
		\item{
			2y' = y + 2\sin x \separate y(0) = -1
		}
		\item{
			(x+1)y' = 2y + (x+1)^\frac{5}{2} \separate y(0) = 3
		}
	\end{atiSubequations}
	Machen Sie in allen Fällen die Probe durch Einsetzen der allgemeinen Lösung in die ursprüngliche Differentialgleichung.
\end{atiTask}
	\begin{atiTask}[
	title = Eine nichtlineare Differentialgleichung
]
	Lösen Sie die folgende nichtlineare Differentialgleichung, indem Sie diese auf eine lineare Differentialgleichung durch die Substitution $z = y^{-2}$ zurückführen und mithilfe der Methode der Variation der Konstanten behandeln.
	\[
		y' - y + 3x^2y^3 = 0
	\]
	Machen Sie die Probe durch Einsetzen Ihrer Lösung in die ursprüngliche Differentialgleichung.
\end{atiTask}

	\begin{atiTask}[
	title = Zwei exakte Differentialgleichungen,
	topic = Gewöhnliche Differentialgleichungen,
	subtopic = Exakte Differentialgleichungen: Der integrierende Faktor,
	language = Deutsch,
]
	Weisen Sie nach, dass die beiden folgenden Differentialgleichungen exakt sind und konstruieren Sie deren Lösung in \enquote{vollständigen Differentialen}.
	\begin{atiSubequations}
		\item{
			\frac{y^2}{x} + 2yy'\ln x = 0
		}
		\item{
			\curvb{8y - x^2y}y' + \curvb{x - xy^2} = 0
		}
	\end{atiSubequations}
	Machen Sie in beiden Fällen die Probe durch Einsetzen der Lösung in die ursprüngliche Differentialgleichung.
\end{atiTask}
	\begin{atiTask}[
	title = Eine nichtexakte Differentialgleichung,
	topic = Gewöhnliche Differentialgleichungen,
	subtopic = Exakte Differentialgleichungen: Der integrierende Faktor,
	language = Deutsch,
]
	\begin{atiSubtasks}
		\item{
			Weisen Sie nach, dass die folgende Differentialgleichung nicht exakt ist.
			\[
				x^2y' - xy = \frac{2}{x}
			\]
		}
		\item{
			Bestimmen Sie einen integrierenden Faktor $\lambda$, der nur von der Variablen $x$ abhängt, und überzeugen Sie sich, dass dieser die Differentialgleichung exakt macht.
		}
		\item{
			Lösen Sie die Differentialgleichung in \enquote{vollständigen Differentialen} und machen Sie die Probe durch Einsetzen Ihrer Lösung in die ursprüngliche Differentialgleichung.
		}
		\item{
			Geben Sie diejenige spezielle Lösung an, die der folgenden Bedingung genügt.
			\[
				y(1) = 2
			\]
		}
	\end{atiSubtasks}
\end{atiTask}
	\begin{atiTask}[
	title = Integrierende Faktoren,
	topic = Gewöhnliche Differentialgleichungen,
	subtopic = Exakte Differentialgleichungen: Der integrierende Faktor,
	language = Deutsch
]
	\begin{atiSubtasks}
		\item{
			Die folgende Differentialgleichung sei gegeben.
			\[
				A(x,y) + B(x,y)y' = 0
			\]
			Zeigen Sie durch Verwendung der Integrabilitätsbedingung die beiden folgenden Aussagen.
			\begin{atiItems}
				\item{
					Kann der folgende Ausdruck als eine Funktion $f$ der Variablen $z(x,y)\define xy$ geschrieben werden, so hängt auch der integrierende Faktor $\lambda$ nur von dieser Variable $z$ ab.
					\[
						\frac{1}{xA(x,y)-yB(x,y)}\curvb{\partial_x B(x,y) - \partial_y A(x,y)}
					\]
					Geben Sie den Zusammenhang von $\lambda$ und $f$ an.
				}
				\item{
					Kann der folgende Ausdruck als eine Funktion $g$ der Variablen $w(x,y)\define x + y$ geschrieben werden, so hängt auch der integrierende Faktor $\lambda$ nur von dieser Variable $z$ ab.
					\[
						\frac{1}{A(x,y)-B(x,y)}\curvb{\partial_x B(x,y) - \partial_y A(x,y)}
					\]
					Geben Sie den Zusammenhang von $\lambda$ und $g$ an.

				}
			\end{atiItems}
		}
		\item{
			Eine der beiden zuvor genannten Eigenschaften trifft auf eine der beiden folgenden Differentialgleichungen zu.
			Finden Sie diesen Fall heraus und berechnen Sie einen integrierenden Faktor mit dem Ergebnis aus dem vorherigen Aufgabenteil.
			\begin{atiSubequations}
				\item{
					(xy-1) + \curvb{x^2-xy}y' = 0
				}
				\item{
					y + \curvb{x - 2x^2y^3}y' = 0
				}
			\end{atiSubequations}
			Lösen Sie die Differentialgleichung mit diesem integrierenden Faktor und machen Sie anschließend die Probe anhand der ursprünglichen Differentialgleichung.
			Die Lösung der verbleibenden Differentialgleichung ist nicht verlangt.
		}
	\end{atiSubtasks}
\end{atiTask}

	% \begin{atiTask}[
	topic = Gewöhnliche Differentialgleichungen,
	subtopic = Separable Differentialgleichungen,
	title = Die Methode der Variablentrennung,
	language = deutsch,
]
	Lösen Sie die folgenden Differentialgleichungen durch Trennung der Variablen und bestimmen Sie gegebenenfalls die Integrationskonstante, sodass die nebenstehenden Anfangsbedingungen erfüllt sind.
	\begin{atiSubequations}
		\item{
			\label{dgl-1}
			y' = \frac{x e^{-y}}{x^2 + 1} \separate y(1) = 0
		}
		\item{
			\label{dgl-2}
			xyy' = \frac{x^2 + 2}{y-1}
		}
		\item{
			\label{dgl-3}
			y' = \frac{x+y}{x+y+2} \separate y(1) = -1
		}
	\end{atiSubequations}
	Machen Sie in allen Fällen die Probe durch Einsetzen Ihrer Lösung in die ursprüngliche Differentialgleichung. Dies ist auch dann verlangt, wenn die Lösung nur in impliziter Form angebbar ist.

	\begin{atiNote}
		Führen Sie in Teilaufgabe \ref{dgl-3} die neue Variable $z(x)\define x + y(x)$ ein.
	\end{atiNote}
\end{atiTask}

\begin{atiSolution}
	\begin{atiSubtaskSolutions}
		\item[\ref{dgl-1}]{
			Separieren Sie $x$ und $y(x)$ auf jeweils eine Seite der Differentialgleichung und integrieren Sie die erhaltene Gleichung.
			\[
				y'(x) = \frac{xe^{-y(x)}}{x^2 + 1} \implies e^{y(x)} y'(x) = \frac{x}{x^2 + 1}
			\]
			\[
				\implies \integral{}{}{e^{y(x)}y'(x)}{x} = \integral{}{}{\frac{x}{x^2+1}}{x}
				\atiPoints[1]
			\]
			Lösen Sie das Integral mithilfe einer logarithmischen Integration oder durch Substitution, indem Sie $x^2$ durch eine geeignete Variable ersetzen.
			\[
				\integral{}{}{e^y}{y} = \frac{1}{2} \integral{}{}{\frac{2x}{x^2+1}}{x}
			\]
			\[
				\implies e^{y(x)} = \frac{1}{2}\ln(x^2 + 1) + C = \ln\sqrt{x^2 + 1} + C
				\atiPoints[1]
			\]
			Notieren Sie die explizite Lösung durch die Anwendung von $\ln$.
			\[
				y(x) = \ln\boxb{\ln\curvb{A\sqrt{x^2+1}}} \separate A\define e^C
				\atiPoints[1]
			\]
			Fordern Sie nun $y(1)\demand 0$, bestimmen Sie die Konstante $A$ und setzen Sie die erhaltene Lösung in die explizite allgemeine Form ein.
			\[
				y(1) \demand 0 \implies 1 = \ln \curvb{A\sqrt{2}} \implies A = \frac{\sqrt{2}}{2}e
				\atiPoints[1]
			\]
			\[
				y(x) = \ln\boxb{\ln\curvb{\frac{e}{2}\sqrt{2(x^2 + 1)}}} = \ln\boxb{1 + \ln\curvb{\frac{1}{2}\sqrt{2(x^2+1)}}}
				\atiPoints[1]
			\]

			Sei $y\definedby \ln u$ mit $u(x) = \ln\curvb{A\sqrt{x^2+1}}$.
			Dann erhält man durch die Anwendung der Kettenregel die folgende Aussage.
			\[
				y'(x) = u'(x) \ln' u(x) = \frac{1}{A\sqrt{x^2+1}} \cdot A \cdot \frac{2x}{2\sqrt{x^2+1}} \cdot \frac{1}{u(x)} = e^{-y(x)} \frac{x}{x^2+1}
				\atiPoints[1]
			\]
		}

		\item[\ref{dgl-2}]{
			Separieren Sie $x$ und $y(x)$ wieder auf jeweils eine Seite der Differentialgleichung
			\[
				xy(x)y'(x) = \frac{x^2+2}{y(x)-1} \implies y(x)\boxb{y(x)-1}y'(x) = \frac{x^2+2}{x}
				\atiPoints[1]
			\]
			Integrieren Sie die rechte Seite der erhaltenen Gleichung durch Polynomintegration und der Umkehrregel.
			\[
				\integral{}{}{\boxb{y^2(x)-y(x)}y'(x)}{x} = \integral{}{}{\curvb{x+\frac{2}{x}}}{x}
			\]
			\[
				\implies \integral{}{}{y^2-y}{y} = \frac{x^2}{2} + 2\ln\abs{x}
				\atiPoints[1]
			\]
			Lösen Sie nun auch das Integral der rechten Seite durch Polynomintegration und notieren Sie die allgemeine Lösung in impliziter Form.
			\[
				\frac{y^3}{3} - \frac{y^2}{2} = \frac{x^2}{2} + 2\ln\abs{x} + C
			\]
			\[
				\implies 2y^3 - 3y^2 = 3x^2 + 12 \ln \abs{x} + D
				\atiPoints[1]
			\]

			Durch implizite Ableitung der allgemeinen Form erhalten Sie Folgendes.
			\[
				6y^2(x)y'(x) - 6y(x)y'(x) = 6x + \frac{12}{x} \implies y(x)\boxb{y(x)-1}y'(x) = x + \frac{2}{x}
			\]
			\[
				\implies xy(x)\boxb{y(x)-1}y'(x) = x^2+2
				\atiPoints[1]
			\]
		}

		\item[\ref{dgl-3}]{
			Definieren Sie $z(x)\define x + y(x)$ und bestimmen Sie die Ableitung von $z$.
			\[
				z'(x) = 1 + y'(x) \implies y'(x) = z'(x) - 1
			\]
			Substituieren Sie nun $x+y(x)$ in der Differentialgleichung durch $z(x)$.
			\[
				y'(x) = \frac{x+y(x)}{x+y(x)+2} \implies z'(x) - 1 = \frac{z}{z+2}
				\atiPoints[1]
			\]
			Führen Sie für die erhaltene Differentialgleichung das Verfahren der Trennung der Variablen durch.
			Separieren Sie $z(x)$ und $x$ auf jeweils eine Seite und integrieren Sie die erhaltene Gleichung.
			\[
				z'(x) = \frac{z(x)}{z(x)+2} + 1 = \frac{2z(x)+2}{z+2} = 2\,\frac{z(x)+1}{z(x)+2}
				\atiPoints[1]
			\]
			\[
				\implies \integral{}{}{\frac{z(x)+2}{z(x)+1}z'(x)}{x} = \integral{}{}{\curvb{1 + \frac{1}{z+1}}}{z} = \integral{}{}{2}{x}
			\]
			\[
				\implies z(x) + \ln\abs{z(x)+1} = 2x + C
				\atiPoints[1]
			\]
			Führen Sie die Resubstitution durch und geben Sie die allgemeine Lösung in impliziter Form an.
			\[
				x + y(x) + \ln\abs{x+ y(x) + 1} = 2x + C
			\]
			\[
				\implies x+y(x)+1 = \exp(x-y(x)+C) = Ae^{x-y(x)} \separate A\define e^C
			\]
			\[
				\implies x+y(x) = Ae^{x-y(x)} -1
				\atiPoints[1]
			\]
			Fordern Sie die gegebenen Anfangsbedingungen und bestimmen Sie die Konstante $A$.
			\[
				y(1)\demand -1 \implies 0=Ae^2-1 \implies A= e^{-2}
			\]
			\[
				\implies x+y(x) = \frac{e^{x-y(x)}}{e^2}-1 = e^{x-y(x)-2}-1
				\atiPoints[1]
			\]

			Auch hier ist wieder eine implizite Ableitung notwendig.
			\[
				1+y'(x) = Ae^{x-y(x)} \boxb{1-y'(x)}
			\]
			Durch Verwendung der allgemeinen Lösung erhalten Sie für die Konstante $A$ den folgenden Ausdruck.
			\[
				A = e^{y(x)-x}\boxb{x+y(x)+1}
			\]
			Das Einsetzen dieser Gleichung resultiert dann in der gewünschten Differentialgleichung.
			\[
				1+y'(x) = e^{y(x)-x}\boxb{x+y(x)+1} e^{x-y(x)} \boxb{1-y'(x)}
			\]
			\[
				\implies 1+y'(x) = x+y(x)+1 - \boxb{x+y(x)+1}y'(x)
			\]
			\[
				\implies y'(x)\boxb{x+y(x)+2} = x+y(x)
				\atiPoints[1]
			\]
		}
	\end{atiSubtaskSolutions}
\end{atiSolution}
	% \begin{atiTask}[
	title = Homogene Differentialgleichungen,
	points = 13
]
	Eine Funktion von zwei Variablen heißt homogen vom Grad $k$, wenn für einen beliebigen Parameter $\lambda$ das Folgende gilt.
	\[
		f(\lambda x, \lambda y) = \lambda^k f(x,y)
	\]
	Dementsprechend nennt man eine Differentialgleichungen der folgenden Form auch homogen, wenn $f$ und $g$ homogene Funktionen vom gleichen Grad sind.

	\begin{atiSubtasks}
		\item \label{subtask:homogene-differentialgleichung-a}
		Lösen Sie die folgende homogene Differentialgleichung.
		\[
			2xyy^\prime = 3y^2 - x^2
		\]
		Anleitung: Führen Sie eine neue Variable $z(x)$ gemäß $y(x)\definedby x\cdot z(x)$ ein und behandeln Sie die für $z(x)$ entstehende Differentialgleichung mit der Methode der Trennung der Variablen.

		\item
		Lösen Sie die folgende Differentialgleichung, die nicht homogen ist.
		\[
			y^\prime = \frac{y+x-2}{y-x+4}
		\]
		Schuld daran sind die beiden additiven Konstanten in Zähler und Nenner des Bruches auf der rechten Seite.
		Gehen Sie in zwei Schritten nach folgender Anleitung vor.
		\begin{atiItems}
			\item Führen Sie die neue Variablen $v\define y-y_0$ und $u\define x-x_0$ ein und bestimmen Sie $x_0$ und $y_0$, sodass die neue Differentialgleichung in den Variablen $u$ und $v$ homogen ist (Gleichungssystem mit zwei Unbekannten).

			\item Verfahren Sie mit der Substitution $v(u)\definedby u\cdot z(u)$ weiter, wie in Teilaufgabe \ref{subtask:homogene-differentialgleichung-a}.
		\end{atiItems}

		\item
		Machen Sie in beiden Fällen die Probe durch Einsetzen Ihrer Lösung $y(x)$ in die ursprüngliche Differentialgleichung.
	\end{atiSubtasks}
\end{atiTask}
% \begin{atiSolution}

% \end{atiSolution}
	% \begin{atiTask}[
	title = Orthogonaltrajektorien,
	points = 10
]
\label{task:3}
	\begin{atiSubtasks}
		\item Skizzieren Sie die folgende Kurvenschar, wobei $c$ eine Konstante ist.
		\[
			xy = c
		\]
		Bestimmen Sie dazu die Schar der Orthogonaltrajektorien ud tragen Sie diese in Ihre Skizze ein.

		\item
		Skizzieren Sie die folgende Kurvenschar und bestimmen Sie die Differentialgleichung, diese Kurvenschar genügt.
		\[
			y^2 = 4c(x+c)
		\]
		Zeigen Sie dann, dass diese Differentialgleichung die gleiche bleibt, wenn $y^\prime$ durch $-\frac{1}{y^\prime}$.
		Welche Schlussfolgerungen ziehen Sie aus dieser Eigenschaft?
	\end{atiSubtasks}
\end{atiTask}
% \begin{atiSolution}[\ref{task:3}]

% \end{atiSolution}
	% \printsolutions

\end{document}
