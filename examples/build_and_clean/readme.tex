\documentclass{article}
\usepackage{atistandard}
\begin{document}
  \section{Abhängigkeiten} % (fold)
  \label{sec:abhängigkeiten}
    Die folgenden Abhängigkeiten müssen auf dem verwendeten Linux-Betriebssystem installiert sein.
    \begin{description}
      \item[find:] Standardtool unter Linux
      \item[realpath:] Standardtool unter Linux
      \item[make:]
        Make wird hier als Build-System verwendet.
        Sollte es nicht bereits installiert sein, muss es wahrscheinlich mithilfe des Package-Managers heruntergeladen werden.
      \item[latexmk:]
        Dieses Tool kümmert sich um die Kompilierung von \LaTeX-Code ohne mehrmaliges Ausführen.
        Es ist standardmäßig nicht installiert, sollte aber in den offiziellen Paketquellen zu finden sein.
    \end{description}
  % section abhängigkeiten (end)

  \section{Building} % (fold)
  \label{sec:building}
    \begin{enumerate}
      \item Öffne ein Terminal im Hauptverzeichnis dieses Projektes.
      \item Führe den Befehl \texttt{make} aus.
    \end{enumerate}
    Dieser Build-Prozess kann öfters wiederholt werden.
    Er kümmert sich um die Installation von Paketen und die Generierung des Aufgabenkatalogs.
  % section building (end)

  \section{Cleaning} % (fold)
  \label{sec:cleaning}
    Unter Umständen möchte man die temporären Dateien, die bei der Generierung des Aufgabenkatalogs entstehen löschen, um so eine weitere neue Kompilierung zu erzwingen.
    \begin{enumerate}
      \item Öffne ein Terminal im Hauptverzeichnis dieses Projekts.
      \item Führe \texttt{cd catalog} aus, um dich in das Verzeichnis \texttt{catalog} zu bewegen.
      \item Für das eigentliche Säubern führt man nun \texttt{make clean} aus.
      \item Anschließend lässt sich der Katalog noch einmal ohne fehlerhaften Cache generieren, indem man \texttt{make} aufruft.
    \end{enumerate}
  % section cleaning (end)
\end{document}