\documentclass{article}
\usepackage{atistandard}
\usepackage{listings}

\newcommand{\showExample}[1]{%
  \medskip
  \hrule
  \begin{minipage}[c]{0.49\textwidth}
    \footnotesize
    \lstinputlisting[breaklines]{#1}
  \end{minipage}
  \hfill
  \begin{minipage}[c]{0.49\textwidth}
    % \footnotesize
    \atiImportTask{.}{#1}
  \end{minipage}
  \hrule
  \medskip
}

\begin{document}
  \section{\texttt{atiTask}-Environment} % (fold)
    \noindent
    \showExample{task_01.tex}
    \showExample{task_02.tex}
    \showExample{task_03.tex}
    \showExample{task_04.tex}

    Im Folgenden wird das Kommando \texttt{atiSetup} verwendet, um dessen Optionen an Beispielen zu zeigen.
    Dieses Kommando sollte jedoch in Aufgabendateien vorkommen, sondern nur in Übungsserien.

    \showExample{task_05.tex}
    \showExample{task_06.tex}
  % section texttt (end)

  \section{\texttt{atistandalonetask}-Klasse} % (fold)
    Diese Klasse ermöglicht es Aufgaben unabhängig von Sheets zu kompilieren und diese so auf ihre Korrektheit zu überprüfen.
    In einer solchen Aufgabendatei sollte darauf geachtet werden, maximal den Titel und die Punktzahl der Aufgabe zu definieren und nicht etwa deren Typ, der vom jeweiligen Aufgabenblatt abhängt.
    \noindent
    \showExample{standalonetask_01.tex}

    Somit ist es möglich auch lokal Pakete einzubinden, die für die Formulierung der Aufgabe notwendig sind, ohne diese in eine global Datei hinzuzufügen.
    \showExample{standalonetask_02.tex}
  % section texttt (end)

  \section{\texttt{atiImportTask} - Einbinden von Aufgaben} % (fold)
    Ob nun standalone oder nicht, jede Aufgabe lässt sich in ein Sheet durch den Befehl \texttt{atiImportTask} einbinden.
    Dieser ermöglicht durch ein optionales Argument eine lokale Abänderung der Aufgabenparameter für das derzeitige Sheet (zum Beispiel Punktzahl, Titel, Lösungen, Typ, Nummerierung).
    \showExample{import_01.tex}

  % section texttt (end)
\end{document}