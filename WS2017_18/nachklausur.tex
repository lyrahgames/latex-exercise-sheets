\RequirePackage{import}
\import{../}{exercise_sheet_preamble}
\begin{document}
\renewcommand{\vec}[1]{\mathbf{#1}}
  \examSheetHead{%
    title = Mathematische Methoden der Physik I,
    sheetType = Nachklausur,
   % sheetNumber = 1, % can be omitted
    author = Dr. Agnes Sambale,
    authorMail = agnes.sambale@uni-jena.de,
    %submissionDate = {Mittwoch, 17.8.17},
    %submissionCall = ,
    extraInfo = Wintersemester 17/18 % can be omitted
  }
  	%	\[\leibnizDerivative[n]{I}{t}\]
  	%	\[\integral{C}{}{x}{x}\]
  	%	\[\indefiniteIntegral{x}{x}\]
\noindent\textbf{Bitte für jede Aufgabe ein eigenes Blatt verwenden!}		
		\begin{atiTask}[
  title = Variation der Konstanten
]
Lösen Sie die Differentialgleichung eines $R-L$-Schwingkreis
%  \[
%    \function{F}{\setReal^2}{\setReal^2}
%    \separate
%    F(x,y)\define (x^2+y^2)\vectorX + 4xy\vectorY
%  \]
\begin{equation*}
\frac{\D I}{\D t}+\frac{R}{L}I=\frac{U_0}{L}\sin (\omega t)
\end{equation*}
wobei $R, L, U_0, \omega$ Konstanten sind. Verwenden Sie dazu das Verfahren der Variation der Konstanten.  \end{atiTask}
%Lösung: skript Lotze S. 13\marginpar{5 Punkte}
		\begin{atiTask}[
  title = Inhomogene Differentialgleichung 2. Ordnung
]
Bestimmen Sie die \textit{allgemeine} Lösung der Differentialgleichung
%  \[
%    \function{F}{\setReal^2}{\setReal^2}
%    \separate
%    F(x,y)\define (x^2+y^2)\vectorX + 4xy\vectorY
%  \]
\begin{equation*}
y''-2y'-3y=64xe^{x}
\end{equation*}
indem Sie für die Lösung der inhomogenen Gleichung einen speziellen Ansatz machen.
\end{atiTask}
\marginpar{5 Punkte}
		%\begin{atiTask}[
  title = Eulersche Differentialgleichung (unfertig)
]
Bestimmen Sie die Lösung der Differentialgleichung
%  \[
%    \function{F}{\setReal^2}{\setReal^2}
%    \separate
%    F(x,y)\define (x^2+y^2)\vectorX + 4xy\vectorY
%  \]
\begin{equation*}
x^2y''-xy'+2y=0,\quad x>0
\end{equation*}
in 3 Schritten:
\begin{atiSubtasks}
\item Zeigen Sie, dass die Substitution $x=e^t$ auf die tranformierte Gleichung mit konstanten Koeffizienten
\begin{equation*}
\ddot{u}-2\dot{u}+2u=0
\end{equation*}
führt.
\item Lösen Sie die transformierte Gleichung und substitutieren Sie anschließend zurück.
\item Bestimmen Sie diejenige Lösung, die mit den Anfangsbedingungen $y(1)=1$ und $y'(0)=1$ verträglich ist.
\end{atiSubtasks}
\end{atiTask}

		%\begin{atiTask}[
  title = Wegintegrale berechnen
]
  Berechnen Sie das Kurvenintegral
%  \[
%    \function{F}{\setReal^2}{\setReal^2}
%    \separate
%    F(x,y)\define (x^2+y^2)\vectorX + 4xy\vectorY
%  \]
\begin{equation*}
W=\int_C \vec{F} \D\vec{r}=\int _C\roundBrackets{\frac{x^2}{x^2+y^2}\D x+\frac{x}{x^2+y^2}\D y}
\end{equation*}
wobei die Kurve $C$ durch $C: x^2+y^2=9$ (entgegen dem Uhrzeigersinn) gegeben ist.
\end{atiTask}
%Lösung: Wallner S 76\marginpar{5 Punkte}
		\import{../database/mathematische_methoden_der_physik_i/vector_analysis/}{task-konservative_vektorfelder_ii}\marginpar{5 Punkte}
	%	\begin{atiTask}[
  title = Wegintegrale berechnen
]
Gegeben sei das Vektorfeld $\vec{v}=(x+y)\vec{i}+z\vec{j}+3\vec{k}$. Berechnen Sie das Vektorfeld
%  \[
%    \function{F}{\setReal^2}{\setReal^2}
%    \separate
%    F(x,y)\define (x^2+y^2)\vectorX + 4xy\vectorY
%  \]
\begin{equation*}
W=\int_C \vec{F} \D\vec{r}
\end{equation*}
wobei die Kurve $C$ die Schnittkurve der Flächen $z=1-x^2$ und $x^2+y^2=1$ ist. \end{atiTask}
%Lösung: Wallner S 76
		\import{../database/mathematische_methoden_der_physik_i/vector_analysis/}{task-wegintegrale_berechnen3.tex}\marginpar{5 Punkte}
		\import{../database/mathematische_methoden_der_physik_i/ordinary_deq/exact_deq/}{task-exakte_differentialgleichungen_ii}\marginpar{5 Punkte}
		\import{../database/mathematische_methoden_der_physik_i/exam_questions/}{task-nachklausurws1718.tex}\marginpar{11 Punkte}
		\import{../database/mathematische_methoden_der_physik_i/vector_analysis/}{task-doppelintegral_i.tex}\marginpar{3 Punkte}
\end{document}
