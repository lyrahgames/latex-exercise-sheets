\begin{atiTask}[
  title = Spezielle integrierende Faktoren
  ]
  Es sei die folgende gewöhnliche Differentialgleichung erster Ordnung für zwei stetig differenzierbare Funktionen $\function{A,B}{\setReal^2}{\setReal}$ gegeben.
  \[
    A(x,y) + B(x,y)y' = 0
  \]
  Weiterhin seien der Term $M(x,y)$ für alle $x,y\in\setReal$ mit $xA(x,y)\neq yB(x,y)$ und der Term $N(x,y)$ für alle $x,y\in\setReal$ mit $A(x,y)\neq B(x,y)$ gegeben.
  \[
    M(x,y) \define \frac{\partial_1B(x,y) - \partial_2A(x,y)}{xA(x,y)-yB(x,y)}
    \separate
    N(x,y) \define \frac{\partial_1B(x,y) - \partial_2A(x,y)}{A(x,y)-B(x,y)}
  \]
  \begin{atiSubtasks}
    \item{\locallabel{a}
      Man nehme nun an, dass der Term $M(x,y)$ nur in Abhängigkeit von $x\cdot y$ beschrieben werden kann, das heißt es gibt eine Funktion $\function{f}{\setReal}{\setReal}$, sodass $f(x\cdot y) = M(x,y)$ gilt.
      Zeigen, dass es in diesem Falle auch einen integrierenden Faktor $\function{\lambda}{\setReal^2}{\setReal}$ gibt, sodass sich $\lambda(x,y)$ nur in Abhängigkeit von $x\cdot y$ darstellen lässt.
      Es muss also eine Funktion $\function{\gamma}{\setReal}{\setReal}$ geben, sodass $\gamma(x\cdot y) = \lambda(x,y)$ gilt.
      Geben Sie zudem den Zusammenhang zwischen $\gamma$ und $f$ an.
    }
    \item{\locallabel{b}
      Zeigen Sie, dass die folgende Differentialgleichung der Bedingung aus \localref{a} genügt und bestimmen Sie den entsprechenden integrierenden Faktor.
      Berechnen Sie dann eine implizite Lösung durch das Auffinden einer Potentialfunktion.
      \[
        y + \roundBrackets{x - 2x^2y^3}y' = 0
      \]
    }
    \item{\locallabel{c}
      \textbf{Zusatz:}
      Man nehme nun an, dass der Term $N(x,y)$ nur in Abhängigkeit von $x+ y$ beschrieben werden kann, das heißt es gibt eine Funktion $\function{f}{\setReal}{\setReal}$, sodass $f(x+ y) = N(x,y)$ gilt.
      Zeigen, dass es in diesem Falle auch einen integrierenden Faktor $\function{\lambda}{\setReal^2}{\setReal}$ gibt, sodass sich $\lambda(x,y)$ nur in Abhängigkeit von $x+ y$ darstellen lässt.
      Es muss also eine Funktion $\function{\gamma}{\setReal}{\setReal}$ geben, sodass $\gamma(x+ y) = \lambda(x,y)$ gilt.
      Geben Sie zudem den Zusammenhang zwischen $\gamma$ und $f$ an.
    }
    \item{\locallabel{d}
      \textbf{Zusatz:}
      Zeigen Sie, dass die folgende Differentialgleichung der Bedingung aus \localref{c} genügt und bestimmen Sie den entsprechenden integrierenden Faktor.
      Berechnen Sie dann eine implizite Lösung durch das Auffinden einer Potentialfunktion.
      \[
        0 = 1 + \frac{e^{x-y}}{\cos\roundBrackets{x+y}} + \boxBrackets{1 + \frac{1-e^{x-y}}{\cos\roundBrackets{x+y}}}y'
      \]
    }
  \end{atiSubtasks}
\end{atiTask}