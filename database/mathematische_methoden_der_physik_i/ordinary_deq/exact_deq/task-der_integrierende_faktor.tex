\begin{atiTask}[
  title = Der integrierende Faktor
]
  Untersuchen Sie die folgenden Differentialgleichungen auf Exaktheit.
  % und lösen Sie die nicht-exakten Differentialgleichungen.
  \begin{atiSubequations}
    \item{
      \curvb{x+y}x^2y' + xy^2 + 3x^2y = 0
    }
    \item{
      yx^3 - 2x^4 = \curvb{3y^2x^3-x^4} y'
    }
    \item{
      \curvb{x\cos y - xy\sin y}y' + 2y\cos y + x = 0
    }
  \end{atiSubequations}
  Berechnen Sie die allgemeinen Lösungen der nicht-exakten Differentialgleichungen in impliziter Form, indem Sie die folgende Anleitung verwenden.
  \begin{atiItems}
    \item{
      Bestimmen Sie einen integrierenden Faktor, der nur von der ersten freien Variable abhängt.
    }
    \item{
      Notieren Sie die neue, mit dem integrierenden Faktor multiplizierte, Differentialgleichung.
    }
    \item{
      Zeigen Sie die Exaktheit der neuen Differentialgleichung.
    }
    \item{
      Lösen Sie die neue Differentialgleichung durch das Auffinden einer Potentialfunktion.
    }
    \item{
      Führe Sie die Probe durch, indem Sie die erhaltene Lösung implizit differenzieren und auf die ursprüngliche Differentialgleichung zurückführen.
    }
  \end{atiItems}
  % \begin{atiSubtasks}
  %   \item{
  %     Überprüfen Sie, ob die gegebenen Differentialgleichungen exakt sind.
  %   }
  %   \item{
  %     Lösen Sie die nicht-exakten Differentialgleichungen, indem Sie nach der folgenden Anleitung vorgehen.
  %     \begin{atiItems}
  %       \item{
  %         Bestimmen Sie einen integrierenden Faktor, der nur von der ersten freien Variable $x$ abhängt.
  %       }
  %       \item{
  %         Überprüfen Sie die Exaktheit der mit dem integrierenden Faktor multiplizierten Differentialgleichung.
  %       }
  %       \item{
  %         Lösen Sie nun die mit dem integrierenden Faktor multiplizierte Differentialgleichung durch das Auffinden einer Potentialfunktion.
  %       }
  %       \item{
  %         Führe Sie die Probe durch, indem Sie Ihre Lösung implizit differenzieren und auf die ursprüngliche Differentialgleichung zurückführen.
  %       }
  %     \end{atiItems}
  %   }
  % \end{atiSubtasks}
\end{atiTask}