\begin{atiTask}[
  title = Vollständiges Differential
]
  Es sei die folgende skalare Funktion für alle $x,y,z\in\setR$ gegeben.
  \[
    \func{U}{\setR^3}{\setR}\separate U(x,y,z)\define x^4yz^2 + 2y^2x^3e^z
  \]
  \begin{atiSubtasks}
    \item{\locallabel{a}
      Bestimmen Sie das totale Differential $\diff U$ der Funktion $U$.
    }
    \item{\locallabel{b}
      Untersuchen Sie, ob es eine Funktion $\func{V}{\setR^3}{\setR}$ gibt, sodass das totale Differential $\diff V$ von $V$ für alle $x,y,z\in\setR$ mit $x\neq 0$ die folgende Gleichung erfüllt.
      \[
        \diff V(x,y,z)\define \frac{1}{x^2} \diff U(x,y,z)
      \]
      \begin{atiNote}
        Überprüfen Sie dafür die Integrabilitätsbedingung des gegebenen Differentials.
      \end{atiNote}
      % Es sei nun für ein $c\in\setR$ eine Fläche im Raum durch die Gleichung $U(x,y,z)=c$ gegeben.
      % Stellen Sie durch Verwendung des totalen Differentials die zugehörige exakte partielle Differentialgleichung dieser Fläche auf und überprüfen Sie diese nach der Multiplikation mit $\frac{1}{x^2}$ auf Exaktheit.
    }
  \end{atiSubtasks}
\end{atiTask}
\begin{atiSolution}
  \begin{atiSubtaskSolutions}
    \item[\localref{a}]{
      Der Gradient der skalaren Funktion $U$ berechnet sich wie folgt, da es sich bei $U$ um eine stetig differenzierbare Funktion handelt.
      \[
        \nabla U(x,y,z) =
        \begin{pmatrix}
          \partial_1U(x,y,z) \\
          \partial_2U(x,y,z) \\
          \partial_3U(x,y,z)
        \end{pmatrix}
        =
        \begin{pmatrix}
          4x^3yz^2 + 6x^2y^2e^z \\
          x^4z^2 + 4x^3ye^z \\
          2x^4yz + 2x^3y^2e^z
        \end{pmatrix}
      \]
      Das totale Differential existiert, da sich $U$ zweimal stetig differenzieren lässt und damit durch den Satz von Schwarz die Integrabilitätsbedingung automatisch erfüllt ist.
      Es ist dann durch den folgenden Ausdruck gegeben.
      \[
        \diff U = \dotp{\nabla U}{\diff r}
        \separate
        \diff r \define
        \begin{pmatrix}
          \diff x \\
          \diff y \\
          \diff z
        \end{pmatrix}
      \]
      \[
        \implies \diff U(x,y,z) = \partial_1U(x,y,z)\diff x + \partial_2U(x,y,z)\diff y + \partial_3U(x,y,z)\diff z\atiPoints[1]
      \]
      \[
        \implies \diff U(x,y,z) = \curvb{4x^3yz^2 + 6x^2y^2e^z}\diff x + \curvb{x^4z^2 + 4x^3ye^z}\diff y + \curvb{2x^4yz + 2x^3y^2e^z}\diff z \atiPoints[1]
      \]
    }
    \item[\localref{b}]{
      Für alle $x,y,z\in\setR$ mit $x\neq 0$ gilt nun die folgende Aussage.
      Wir definieren hierfür die Funktionen $\func{f,g,h}{\setR\setminus\set{0}\times\setR^2}{\setR}$.
      \[
        \frac{1}{x^2}\diff U(x,y,z) = \underbrace{\curvb{4xyz^2 + 6y^2e^z}}_{\definedby f(x,y,z)}\diff x + \underbrace{\curvb{x^2z^2 + 4xye^z}}_{\definedby g(x,y,z)}\diff y + \underbrace{\curvb{2x^2yz + 2xy^2e^z}}_{\definedby h(x,y,z)}\diff z
      \]
      Ein notwendiges (aber nicht hinreichendes) Kriterium für die Existenz der in der Aufgabenstellung beschriebenen Funktion $V$, ist die Erfüllung der Integrabilitätsbedingung.
      Es gibt nun $x,y,z\in\setR$ mit $x\neq 0$, die die folgenden Implikationen erfüllen.
      \[
        \partial_2f(x,y,z) = \pdrvopb{s}{\curvb{\frac{\partial_1U(x,s,z)}{x^2}}}{y} = \frac{\partial_2\partial_1U(x,y,z)}{x^2}
      \]
      \[
        \partial_1g(x,y,z) = \pdrvopb{s}{\curvb{\frac{\partial_2U(s,y,z)}{s^2}}}{x} = \frac{\partial_2\partial_1U(x,y,z)}{x^2} - \frac{2\ \partial_2U(x,y,z)}{x^3}
      \]
      \[
        \implies \partial_2f(x,y,z) - \partial_1g(x,y,z) = \frac{2\ \partial_2U(x,y,z)}{x^3} = xz^2 + 4ye^z \neq 0
      \]
      \[
        \partial_3f(x,y,z) = \pdrvopb{s}{\curvb{\frac{\partial_1U(x,y,s)}{x^2}}}{z} = \frac{\partial_3\partial_1U(x,y,z)}{x^2}
      \]
      \[
        \partial_1h(x,y,z) = \pdrvopb{s}{\curvb{\frac{\partial_3U(s,y,z)}{s^2}}}{x} = \frac{\partial_3\partial_1U(x,y,z)}{x^2} - \frac{2\ \partial_3U(x,y,z)}{x^3}
      \]
      \[
        \implies \partial_3f(x,y,z) - \partial_1h(x,y,z) = \frac{2\ \partial_3U(x,y,z)}{x^3} = 2xyz + 2y^2e^z \neq 0
      \]
      \[
        \partial_3g(x,y,z) = \pdrvopb{s}{\curvb{\frac{\partial_2U(x,y,s)}{x^2}}}{z} = \frac{\partial_3\partial_2U(x,y,z)}{x^2}
      \]
      \[
        \partial_2h(x,y,z) = \pdrvopb{s}{\curvb{\frac{\partial_3U(x,s,z)}{x^2}}}{y} = \frac{\partial_2\partial_3U(x,y,z)}{x^2}
      \]
      \[
        \implies \partial_3g(x,y,z) - \partial_2h(x,y,z) = 0
      \]
      \atiPoints[1]Die ersten beiden Folgerungen zeigen, dass die Integrabilitätsbedingung nicht erfüllt ist.
      Es reicht eine dieser beiden Aussagen zu zeigen.
      \atiPoints[1]Demzufolge kann es eine solche Funktion $V$ nicht geben.
    }
  \end{atiSubtaskSolutions}
\end{atiSolution}