\begin{atiTask}[
  title = Vollständiges Differential
]
  Es sei die folgende skalare Funktion für alle $x,y,z\in\setReal$ gegeben.
  \[
    \function{U}{\setReal^3}{\setReal}
    \separate
    U(x,y,z)\define x^4yz^2 + 2y^2x^3e^z
  \]
  \begin{atiSubtasks}
    \item{\locallabel{a}
      Bestimmen Sie das totale Differential $\infinitesimal U$ der Funktion $U$.
    }
    \item{\locallabel{b}
      Untersuchen Sie, ob es eine Funktion $\function{V}{\setReal^3}{\setReal}$ gibt, sodass das totale Differential $\infinitesimal V$ von $V$ für alle $x,y,z\in\setReal$ mit $x\neq 0$ die folgende Gleichung erfüllt.
      \[
        \infinitesimal V(x,y,z)\define \frac{1}{x^2} \infinitesimal U(x,y,z)
      \]
      \begin{atiNote}
        Überprüfen Sie dafür die Integrabilitätsbedingung des gegebenen Differentials.
      \end{atiNote}
    }
  \end{atiSubtasks}
\end{atiTask}
\begin{atiSolution}
  \begin{atiSubtaskSolutions}
    \item[\localref{a}]{
      Der Gradient der skalaren Funktion $U$ berechnet sich wie folgt, da es sich bei $U$ um eine stetig differenzierbare Funktion handelt.
      \[
        \nabla U(x,y,z) =
        \begin{pmatrix}
          \partial_1U(x,y,z) \\
          \partial_2U(x,y,z) \\
          \partial_3U(x,y,z)
        \end{pmatrix}
        =
        \begin{pmatrix}
          4x^3yz^2 + 6x^2y^2e^z \\
          x^4z^2 + 4x^3ye^z \\
          2x^4yz + 2x^3y^2e^z
        \end{pmatrix}
      \]
      Das totale Differential existiert, da sich $U$ zweimal stetig differenzieren lässt und damit durch den Satz von Schwarz die Integrabilitätsbedingung automatisch erfüllt ist.
      Es ist dann durch den folgenden Ausdruck gegeben.
      \[
        \infinitesimal U = \dotProduct{\nabla U}{\infinitesimal r}
        \separate
        \infinitesimal r \define
        \begin{pmatrix}
          \infinitesimal x \\
          \infinitesimal y \\
          \infinitesimal z
        \end{pmatrix}
      \]
      \[
        \implies \infinitesimal U(x,y,z) = \partial_1U(x,y,z)\infinitesimal x + \partial_2U(x,y,z)\infinitesimal y + \partial_3U(x,y,z)\infinitesimal z\atiPoints[1]
      \]
      \[
        \implies \infinitesimal U(x,y,z) = \roundBrackets{4x^3yz^2 + 6x^2y^2e^z}\infinitesimal x + \roundBrackets{x^4z^2 + 4x^3ye^z}\infinitesimal y + \roundBrackets{2x^4yz + 2x^3y^2e^z}\infinitesimal z \atiPoints[1]
      \]
    }
    \item[\localref{b}]{
      Für alle $x,y,z\in\setReal$ mit $x\neq 0$ gilt nun die folgende Aussage.
      Wir definieren hierfür die Funktionen $\function{f,g,h}{\setReal\setminus\set{0}{}\times\setReal^2}{\setReal}$.
      \[
        \frac{1}{x^2}\infinitesimal U(x,y,z) = \underbrace{\roundBrackets{4xyz^2 + 6y^2e^z}}_{\reverseDefine f(x,y,z)}\infinitesimal x + \underbrace{\roundBrackets{x^2z^2 + 4xye^z}}_{\reverseDefine g(x,y,z)}\infinitesimal y + \underbrace{\roundBrackets{2x^2yz + 2xy^2e^z}}_{\reverseDefine h(x,y,z)}\infinitesimal z
      \]
      Ein notwendiges (aber nicht hinreichendes) Kriterium für die Existenz der in der Aufgabenstellung beschriebenen Funktion $V$, ist die Erfüllung der Integrabilitätsbedingung.
      Es gibt nun $x,y,z\in\setReal$ mit $x\neq 0$, die die folgenden Implikationen erfüllen.
      \[
        \partial_2f(x,y,z) = \leibnizPartialDerivativeOperatorValue{s}{\roundBrackets{\frac{\partial_1U(x,s,z)}{x^2}}}{y} = \frac{\partial_2\partial_1U(x,y,z)}{x^2}
      \]
      \[
        \partial_1g(x,y,z) = \leibnizPartialDerivativeOperatorValue{s}{\roundBrackets{\frac{\partial_2U(s,y,z)}{s^2}}}{x} = \frac{\partial_2\partial_1U(x,y,z)}{x^2} - \frac{2\ \partial_2U(x,y,z)}{x^3}
      \]
      \[
        \implies \partial_2f(x,y,z) - \partial_1g(x,y,z) = \frac{2\ \partial_2U(x,y,z)}{x^3} = xz^2 + 4ye^z \neq 0
      \]
      \[
        \partial_3f(x,y,z) = \leibnizPartialDerivativeOperatorValue{s}{\roundBrackets{\frac{\partial_1U(x,y,s)}{x^2}}}{z} = \frac{\partial_3\partial_1U(x,y,z)}{x^2}
      \]
      \[
        \partial_1h(x,y,z) = \leibnizPartialDerivativeOperatorValue{s}{\roundBrackets{\frac{\partial_3U(s,y,z)}{s^2}}}{x} = \frac{\partial_3\partial_1U(x,y,z)}{x^2} - \frac{2\ \partial_3U(x,y,z)}{x^3}
      \]
      \[
        \implies \partial_3f(x,y,z) - \partial_1h(x,y,z) = \frac{2\ \partial_3U(x,y,z)}{x^3} = 2xyz + 2y^2e^z \neq 0
      \]
      \[
        \partial_3g(x,y,z) = \leibnizPartialDerivativeOperatorValue{s}{\roundBrackets{\frac{\partial_2U(x,y,s)}{x^2}}}{z} = \frac{\partial_3\partial_2U(x,y,z)}{x^2}
      \]
      \[
        \partial_2h(x,y,z) = \leibnizPartialDerivativeOperatorValue{s}{\roundBrackets{\frac{\partial_3U(x,s,z)}{x^2}}}{y} = \frac{\partial_2\partial_3U(x,y,z)}{x^2}
      \]
      \[
        \implies \partial_3g(x,y,z) - \partial_2h(x,y,z) = 0
      \]
      \atiPoints[1]Die ersten beiden Folgerungen zeigen, dass die Integrabilitätsbedingung nicht erfüllt ist.
      Es reicht eine dieser beiden Aussagen zu zeigen.
      \atiPoints[1]Demzufolge kann es eine solche Funktion $V$ nicht geben.
    }
  \end{atiSubtaskSolutions}
\end{atiSolution}