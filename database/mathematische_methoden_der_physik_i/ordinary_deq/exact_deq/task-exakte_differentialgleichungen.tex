\begin{atiTask}[
  title = Exakte Differentialgleichungen
]
  \begin{atiSubtasks}
    \item{\locallabel{a}
      Zeigen Sie, dass die folgenden Differentialgleichungen exakt sind, und lösen Sie diese durch das Auffinden einer Potentialfunktion.
      \begin{atiSubequations}
        \item{\locallabel{ai}
          \curvb{x+y^3}y' + y = x^3
        }
        \item{\locallabel{aii}
          0 = \sin\curvb{xy^2} + xy^2\cos\curvb{xy^2} + \boxb{2x^2y\cos\curvb{xy^2} + 2y}y'
        }
      \end{atiSubequations}
    }
    \item{\locallabel{b}
      Zeigen Sie, dass es sich bei den folgenden Gleichungen um nicht exakte Differentialgleichungen handelt, indem Sie die Integrabilitätsbedingung überprüfen.
      \begin{atiSubequations}
        \item{\locallabel{bi}
          0 = 2\cos y + 4x^2y\sin y + \curvb{yx^3\cos y + x^3\sin y}y' - xy'\sin y
        }
        \item{\locallabel{bii}
          x\arctan\curvb{\frac{x}{y^2}} + \frac{x^2y^2}{x^2+y^4} = \frac{2x^3y}{x^2+y^4}y'
        }
      \end{atiSubequations}
    }
  \end{atiSubtasks}
\end{atiTask}
\begin{atiSolution}
  \begin{atiSubtaskSolutions}
    \item[\localref{a}]{
      \begin{atiSubtaskSolutions}
        \item[\localref{ai}]{
          Wir definieren zwei Funktionen $\func{f,g}{\setR^2}{\setR}$, sodass für alle $x,y\in\setR$ folgende Gleichungen gelten.
          \[
            f(x,y)\define y-x^3\separate g(x,y)\define x+y^3
          \]
          \[
            \implies f(x,y) + g(x,y)y' = 0
          \]
          Die Integrabilitätsbedingung ist nach der folgenden Aussage für alle $x,y\in\setR$ erfüllt.
          Die Differentialgleichung ist damit exakt.
          \[
            \partial_2f(x,y) = \pdrvopb{\tilde{y}}{f(x,\tilde{y})}{y} = 1 = \pdrvopb{\tilde{x}}{g(\tilde{x},y)}{x} = \partial_1g(x,y)
          \]
          Es gibt nun eine zweimal stetig differenzierbare Potentialfunktion $\func{\varphi}{\setR^2}{\setR}$, die die folgende Bedingung erfüllt.
          \[
            \partial_1\varphi = f \separate \partial_2\varphi = g
          \]
          Um diese zu berechnen, kann man zum Beispiel einen der beiden folgenden Wege verwenden.
          \[
            \varphi(x,y) = \iintegralb{\partial_1\varphi(s,y)}{s}{x} = \iintegralb{f(s,y)}{s}{x} = yx - \frac{x^4}{4} + c(y)
          \]
          \[
            \implies \partial_2\varphi(x,y) = x + c'(y) = g(x,y) = x + y^3
          \]
          \[
            \implies c'(y) = y^3 \implies c(y) = \frac{y^4}{4} + K
          \]
          \[
            \implies \varphi(x,y) = xy - \frac{x^4}{4} + \frac{y^4}{4} + K
          \]
          \[
            \varphi(x,y) = \iintegralb{\partial_2\varphi(x,s)}{s}{y} = \iintegralb{g(x,s)}{s}{y} = xy + \frac{y^4}{4} + d(x)
          \]
          \[
            \implies \partial_1\varphi(x,y) = y + d'(x) = f(x,y) = y - x^3
          \]
          \[
            \implies d'(x) = -x^3 \implies d(x) = -\frac{x^4}{4} + K
          \]
          \[
            \implies \varphi(x,y) = xy - \frac{x^4}{4} + \frac{y^4}{4} + K
          \]
          Jetzt gehen wir davon aus, dass es sich bei $y$ um eine stetig differenzierbare Funktion handelt.
          Nach Verwendung der Kettenregel und der gegebenen Differentialgleichung können wir auf das Folgende für alle $x$ einer offenen Teilmenge $U\subset\setR$ schließen.
          \[
            \fdrvopb{s}{\varphi(s,y(s))}{x} = \partial_1\varphi(x,y(x)) + \partial_2\varphi(x,y(x)) y'(x) = 0
          \]
          Dies ist nur dann möglich, wenn es für alle $x\in U$ eine Konstante $C\in\setR$ gibt, sodass das Folgende gilt.
          \[
            \varphi(x,y(x)) = C = xy(x) - \frac{x^4}{4} + \frac{y^4(x)}{4} + K
          \]
          Dies ist auch gleichzeitig die Funktion $y$ in impliziter Form.
        }
        \item[\localref{aii}]{
          Wir definieren zwei Funktionen $\func{f,g}{\setR^2}{\setR}$, sodass für alle $x,y\in\setR$ folgende Gleichungen gelten.
          \[
            f(x,y) \define \sin\curvb{xy^2} + xy^2\cos\curvb{xy^2} \separate g(x,y) \define 2x^2y\cos\curvb{xy^2} + 2y
          \]
          \[
            \implies f(x,y) + g(x,y)y' = 0
          \]
          Die Integrabilitätsbedingung ist nach der folgenden Aussage für alle $x,y\in\setR$ erfüllt.
          Die Differentialgleichung ist damit exakt.
          \[
            \partial_2f(x,y) = \pdrvopb{\tilde{y}}{f(x,\tilde{y})}{y} = 4xy\cos\curvb{xy2} - 2x^2y^3\sin\curvb{xy^2} = \pdrvopb{\tilde{x}}{g(\tilde{x},y)}{x} = \partial_1g(x,y)
          \]
          Es gibt nun eine zweimal stetig differenzierbare Potentialfunktion $\func{\varphi}{\setR^2}{\setR}$, die die folgende Bedingung erfüllt.
          \[
            \partial_1\varphi = f \separate \partial_2\varphi = g
          \]
          Um diese zu berechnen, kann man zum Beispiel einen der beiden folgenden Wege verwenden.
          \[
            \varphi(x,y) = \iintegralb{\partial_1\varphi(s,y)}{s}{x} = \iintegralb{f(s,y)}{s}{x} = -\frac{\cos\curvb{xy^2}}{y^2} + c(y)
          \]
          \[
            \implies \partial_2\varphi(x,y) = + c'(y) = g(x,y) =
          \]
          \[
            \implies c'(y) = y^3 \implies c(y) = \frac{y^4}{4} + K
          \]
          \[
            \implies \varphi(x,y) = xy - \frac{x^4}{4} + \frac{y^4}{4} + K
          \]
          \[
            \varphi(x,y) = \iintegralb{\partial_2\varphi(x,s)}{s}{y} = \iintegralb{g(x,s)}{s}{y} = xy + \frac{y^4}{4} + d(x)
          \]
          \[
            \implies \partial_1\varphi(x,y) = y + d'(x) = f(x,y) = y - x^3
          \]
          \[
            \implies d'(x) = -x^3 \implies d(x) = -\frac{x^4}{4} + K
          \]
          \[
            \implies \varphi(x,y) = xy - \frac{x^4}{4} + \frac{y^4}{4} + K
          \]
        }
      \end{atiSubtaskSolutions}
    }
    \item[\localref{b}]{
      \begin{atiSubtaskSolutions}
        \item[\localref{bi}]{
          Wir definieren zwei Funktionen $\func{f,g}{\setR^2}{\setR}$, sodass für alle $x,y\in\setR$ folgende Gleichungen gelten.
          \[
            f(x,y) \define 2\cos y + 4x^2y\sin y \separate g(x,y) \define yx^3\cos y + x^3\sin y - x\sin y
          \]
          \[
            \implies f(x,y) + g(x,y)y' = 0
          \]
          Wir überprüfen wieder die Integrabilitätsbedingung.
          Es gibt also $x,y\in\setR$, sodass die folgende Implikation gilt.
          \[
            \partial_2f(x,y) = \pdrvopb{\tilde{y}}{f(x,\tilde{y})}{y} = -2\sin y + 4x^2\sin y + 4x^2y\cos y
          \]
          \[
            \partial_1g(x,y) = \pdrvopb{\tilde{x}}{g(\tilde{x},y)}{x} = 3x^2y\cos y + 3x^2\sin y - \sin y
          \]
          \[
            \implies \partial_2f(x,y) - \partial_1g(x,y) = \curvb{x^2-1}\sin y + \curvb{1-3x^2}y\cos y \neq 0
          \]
          Die Integrabilitätsbedingung ist damit nicht erfüllt.
          Demzufolge ist diese Differentialgleichung nicht exakt.
        }
        \item[\localref{bii}]{
          Wir definieren zwei Funktionen $\func{f,g}{\setR\times\setR\setminus\set{0}}{\setR}$, sodass für alle $x,y\in\setR$ mit $y\neq 0$ folgende Gleichungen gelten.
          \[
            f(x,y) \define x\arctan\curvb{\frac{x}{y^2}} + \frac{x^2y^2}{x^2+y^4} \separate g(x,y) \define -\frac{2x^3y}{x^2+y^4}
          \]
          \[
            \implies f(x,y) + g(x,y)y' = 0
          \]
          Wir überprüfen wieder die Integrabilitätsbedingung.
          Es gibt also $x,y\in\setR$ mit $y\neq 0$, sodass die folgende Implikation gilt.
          \[
            \partial_2f(x,y) = \pdrvopb{\tilde{y}}{f(x,\tilde{y})}{y} = -\frac{4x^4y^5}{\curvb{x^2+y^4}^2}
          \]
          \[
            \partial_1g(x,y) = \pdrvopb{\tilde{x}}{g(\tilde{x},y)}{x} = -\frac{6x^2y}{x^2+y^4} + \frac{4x^4y}{\curvb{x^2+y^4}^2}
          \]
          % \[
          %   \pdrvopb{\tilde{x}}{g(\tilde{x},y)}{x} = -\frac{6x^2y}{x^2+y^4} + \frac{4x^4y}{\curvb{x^2+y^4}^2}
          % \]
          \[
            \implies \partial_2f(x,y) - \partial_1g(x,y) = \frac{6x^2y}{x^2+y^4} - \frac{8x^4y}{\curvb{x^2+y^4}^2} \neq 0
          \]
          Die Integrabilitätsbedingung ist damit nicht erfüllt.
          Demzufolge ist diese Differentialgleichung nicht exakt.
        }
      \end{atiSubtaskSolutions}
    }
  \end{atiSubtaskSolutions}
\end{atiSolution}