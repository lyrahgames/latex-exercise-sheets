\begin{atiTask}[
  title = Konstruktion eines integrierenden Faktors
]
  Wir wenden uns noch einmal der folgenden Differentialgleichung zu.
  \begin{equation}
    \locallabel{equ1}
     y' - y + 3x^2y^3 = 0
  \end{equation}
  Konstruieren Sie nun einen integrierenden Faktor $\Lambda(x,y)$ auf zwei Wegen.
  \begin{atiSubtasks}
    \item{
      Sie haben diese Differentialgleichung in der zweiten Aufgabe der dritten Übungsserie mit der Substitution $z = y^{-2}$ in eine lineare Differentialgleichung überführt, die auch nicht exakt ist.
      \begin{equation}
        \locallabel{equ2}
        z' + 2z - 6x^2 = 0
      \end{equation}
      Bestimmen Sie zuerst für diese Gleichung (\localref{equ2}) einen integrierenden Faktor $\lambda(x)$, der nur von der Variablen $x$ abhängt.
      Dieser Faktor allein macht die ursprüngliche Differentialgleichung (\localref{equ1}) noch nicht exakt.
      Ermitteln Sie nun für die mit $\lambda(x)$ multiplizierte ursprüngliche Gleichung einen integrierenden Faktor $\mu(y)$, der nur von der Variablen $y$ abhängt.
    }
    \item{
      Machen Sie von vornherein für den integrierenden Faktor $\Lambda(x,y)$ der ursprünglichen Differentialgleichung (\localref{equ1}) den Produktansatz $\Lambda(x,y)=\lambda(x)\cdot\mu(y)$ und bestimmen Sie die Funktionen $\lambda(x)$ und $\mu(y)$ ohne Rückgriff auf die in der Variablen $z$ lineare Differentialgleichung.
    }
    \item{
      \textbf{Zusatz:}
      Weisen Sie nach, dass die Ausgangsgleichung (\localref{equ1}) mit diesem integrierenden Faktor $\Lambda(x,y)$ exakt ist und lösen Sie diese.
      Vergleichen Sie das Ergebnis mit dem vorherigen Resultat aus der zweiten Aufgabe der dritten Übungsserie.
    }
  \end{atiSubtasks}
\end{atiTask}