\begin{atiTask}[
  title = Ideales Gas
]
  Ein ideales Gas befinde sich in einem Kolben, der zusammengepresst wird.
  Dabei gelten die folgenden Zustandsgleichungen.
  \[
    pV = Nk_\m{B}T \separate E = \frac{3}{2}Nk_\m{B}T
  \]
  Darin steht $p$ für den Druck, $V$ für das Volumen, $N$ für die Teilchenanzahl, $T$ für die Temperatur, $E$ für die innere Energie und $k_\m{B}$ für die Boltzmann-Konstante.
  Nach dem ersten Hauptsatz der Thermodynamik ändert sich innere Energie durch das Verrichten von Arbeit $p\diff V$ und die Abgabe von Wärme $\delta Q$.
  \[
    \diff E = \delta Q - p\diff V
  \]
  \begin{atiSubtasks}
    \item{
      Schon die Schreibweise $\delta Q$ (anstelle von $\diff Q$) deutet an, dass die Funktion $Q$ kein totales Differential besitzt.
      Bestätigen Sie diese Aussage, indem Sie die Integrabilitätsbedingung überprüfen.
    }
    \item{
      Suchen Sie nun einen integrierenden Faktor $\lambda$, der nur von $T$ abhängt, sodass $\diff S = \lambda \delta Q$ ein totales Differential beschreibt.
    }
    \item{
      Führen Sie eine Probe durch, indem Sie nun für $\diff S$ die Integrabilitätsbedingung nachprüfen.
    }
  \end{atiSubtasks}
\end{atiTask}