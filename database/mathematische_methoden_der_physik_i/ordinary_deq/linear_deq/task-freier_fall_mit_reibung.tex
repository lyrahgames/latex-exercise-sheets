\begin{atiTask}[
  title = Freier Fall mit Reibung
]
 Beim freien Fall mit Fallbeschleunigung $g\in\setR^+$ unter dem Einfluss einer geschwindigkeitsproportionalen Reibungskraft mit Reibungskoeffizient $\gamma\in\setR^+$ genügt die abwärts gerichtete Geschwindigkeitsfunktion $v$ eines Massenpunktes mit Masse $m\in\setR^+$ der folgenden Differentialgleichung.
 \[
   m\tdrv{v} + \gamma v = mg
 \]
 \begin{atiSubtasks}
   \item{\locallabel{a}
    Lösen Sie diese Differentialgleichung durch die Methode der Variation der Konstanten und bestimmen Sie eine Lösung, die den Anfangsbedingungen $t_0\define 0$ und $v_0\define v(t_0)=0$ genügt.
  }
  \item{\locallabel{b}
    Berechnen Sie die stationäre Endgeschwindigkeit $v_\infty$ mithilfe der folgenden Definition und der Lösung des Anfangswertproblems.
    \[
      v_\infty \define \lim_{t\conv\infty}v(t)
    \]
    Berechnen Sie ein zweites Mal $v_\infty$ unter der Annahme, dass die Lösung der Differentialgleichung nicht bekannt ist.
    Betrachten Sie hierfür den Grenzwert der Differentialgleichung.
  }
  \item{\locallabel{c}
    \textbf{Zusatz:}
    Lösen Sie noch einmal die Differentialgleichung des freien Falls durch die Methode der Trennung der Variablen.
  }
 \end{atiSubtasks}
\end{atiTask}
\begin{atiSolution}
  \begin{atiSubtaskSolutions}
    \item[\localref{a}]{
      Die zur gegebenen Differentialgleichung zugehörigen homogene Differentialgleichung kann durch die folgende Form beschrieben werden.
      \[
        m\tdrv{v} + \gamma v = 0 \implies \tdrv{v} = -\frac{\gamma}{m}v\atiPoints[\frac{1}{2}]
      \]
      Diese Formel lässt sich separieren und durch die Methode der Trennung der Variablen lösen.
      \[
        \frac{\tdrv{v}(t)}{v(t)} = -\frac{\gamma}{m} \implies \integral{t_0}{t}{\frac{\tdrv{v}(t)}{v(t)}}{t} = \integral{v_0}{v(t)}{\frac{1}{s}}{s} = -\integral{t_0}{t}{\frac{\gamma}{m}}{t}\atiPoints[\frac{1}{2}]
      \]
      \[
        \implies \ln\curvb{\frac{v(t)}{v_0}} = -\frac{\gamma}{m}(t-t_0) \implies v(t) = v_0 e^{-\frac{\gamma}{m}(t-t_0)}\atiPoints[\frac{1}{2}]
      \]
      Nach der Methode der Variation der Konstanten gehen wir nun davon aus, dass sich die Lösung der ursprünglichen Differentialgleichung durch die folgende Funktion beschreiben lässt.
      \[
        v(t) = \varphi(t)e^{-\frac{\gamma}{m}(t-t_0)}\atiPoints[\frac{1}{2}]
      \]
      \[
        \implies \tdrv{v}(t) = \tdrv{\varphi}(t)e^{-\frac{\gamma}{m}(t-t_0)} - \varphi(t)\frac{\gamma}{m}e^{-\frac{\gamma}{m}(t-t_0)}\atiPoints[\frac{1}{2}]
      \]
      Das Einsetzen der Ableitung in die ursprüngliche Differentialgleichung ergibt dann das Folgende.
      \[
        m\tdrv{\varphi}(t)e^{-\frac{\gamma}{m}(t-t_0)} = mg \implies \tdrv{\varphi}(t) = ge^{\frac{\gamma}{m}(t-t_0)}\atiPoints[\frac{1}{2}]
      \]
      \[
        \implies \varphi(t) = \frac{mg}{\gamma}e^{\frac{\gamma}{m}(t-t_0)} - \frac{mg}{\gamma} + \varphi(t_0)
      \]
      \[
        \implies v(t) = \varphi(t)e^{-\frac{\gamma}{m}(t-t_0)} = \frac{mg}{\gamma}\boxb{1 - e^{-\frac{\gamma}{m}(t-t_0)}} + \varphi(t_0)e^{-\frac{\gamma}{m}(t-t_0)} \atiPoints[\frac{1}{2}]
      \]
      Zu beachten ist hier, dass $v_0 = v(t_0) = \varphi(t_0)$ gilt.
      Die Lösung des Anfangswertproblems ist damit wegen $t_0=0$ und $v_0=0$ gegeben durch die folgende Funktion.
      \[
        v(t) = \frac{mg}{\gamma}\curvb{1-e^{-\frac{\gamma}{m}t}}\atiPoints[\frac{1}{2}]
      \]
    }
    \item[\localref{b}]{
      Die stationäre Endgeschwindigkeit lässt sich nun einfach mithilfe des Limes berechnen.
      \[
        v_\infty = \lim_{t\conv\infty}v(t) = \lim_{t\conv\infty} \frac{mg}{\gamma}\curvb{1-e^{-\frac{\gamma}{m}t}} = \frac{mg}{\gamma}\curvb{1-\lim_{t\conv\infty}e^{-\frac{\gamma}{m}t}} = \frac{mg}{\gamma} \atiPoints[1]
      \]
      Ohne die Kenntnis der Lösung lässt nun das Folgende notieren.
      \[
        \lim_{t\conv\infty}\boxb{m\tdrv{v}(t) + \gamma v(t)} = \lim_{t\conv\infty}mg \implies m\lim_{t\conv\infty}\tdrv{v}(t) + \gamma \lim_{t\conv\infty}v(t) = mg
      \]
      Da der Limes existiert und $v$ eine stetig differenzierbare Funktion ist, muss demnach die Ableitung im Unendlichen verschwinden.
      \[
        \lim_{t\conv\infty}\tdrv{v}(t)=0 \implies v_\infty = \lim_{t\conv\infty}v(t) = \frac{mg}{\gamma} \atiPoints[1]
      \]
    }
    \item[\localref{c}]{
      Die in der Aufgabe gegebene Differentialgleichung ist separierbar und damit durch die Methode der Trennung der Variablen lösbar.
      \[
        \tdrv{v}(t) = g-\frac{\gamma}{m}v(t) \implies \frac{\tdrv{v}(t)}{g-\frac{\gamma}{m}v(t)} = 1 \implies \integral{t_0}{t}{\frac{\tdrv{v}(t)}{g-\frac{\gamma}{m}v(t)}}{t} = \integral{t_0}{t}{}{t} \atiPoints[+\frac{1}{2}]
      \]
      \[
        \implies -\frac{m}{\gamma}\ln\boxb{\frac{g-\frac{\gamma}{m}v(t)}{g-\frac{\gamma}{m}v_0}} = t-t_0
      \]
      \[
        \implies v(t) = \frac{m}{\gamma}\boxb{g - \curvb{g-\frac{\gamma}{m}v_0}e^{-\frac{\gamma}{m}(t-t_0)}}
      \]
      \[
        \implies v(t) = \frac{mg}{\gamma}\boxb{1-e^{-\frac{\gamma}{m}(t-t_0)}} + v_0e^{-\frac{\gamma}{m}(t-t_0)}\atiPoints[+\frac{1}{2}]
      \]
    }
  \end{atiSubtaskSolutions}
\end{atiSolution}