\begin{atiTask}[
  title = Freier Fall mit Reibung
]
 Beim freien Fall mit Fallbeschleunigung $g\in\setR^+$ unter dem Einfluss einer geschwindigkeitsproportionalen Reibungskraft mit Reibungskoeffizient $\gamma\in\setR^+$ genügt die abwärts gerichtete Geschwindigkeitsfunktion $v$ eines Massenpunktes mit Masse $m\in\setR^+$ der folgenden Differentialgleichung.
 \[
   m\tdrv{v} + \gamma v = mg
 \]
 \begin{atiSubtasks}
   \item{
    Lösen Sie diese Differentialgleichung durch die Methode der Variation der Konstanten und bestimmen Sie eine Lösung, die der Anfangsbedingung $v(0)=0$ genügt.
  }
  \item{
    Berechnen Sie die stationäre Endgeschwindigkeit $v_\infty$.
    Dabei gilt die folgende Definition.
    \[
      v_\infty \define \lim_{t\conv\infty}v(t)
    \]
  }
  \item{
    \textbf{Zusatz:}
    Lösen Sie noch einmal die Differentialgleichung des freien Falls durch die Methode der Trennung der Variablen.

  }
 \end{atiSubtasks}
\end{atiTask}