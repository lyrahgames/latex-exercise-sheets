\begin{atiTask}[
	title = Ein Ausweg aus der Zombieapokalypse
]
	Wenig später bricht auch auf einer Nachbarinsel eine merkwürdige Krankheit aus.
	In einer  Zeitspanne $\tau\in\setR^+$ infiziert sich ein Anteil $\delta$ der gesunden Menschen auf der Insel und mutiert zu Zombies.
	Dieses Mal wird sofort mit der Evakuierung begonnen.
	Während derselben Zeitspanne $\tau$ können $N_0$ Menschen die Insel per Schiff verlassen.
	Infizierte können die Insel nicht verlassen.
	\begin{atiSubtasks}
		\item{
			Stellen Sie die zugehörigen Differentialgleichungen für $M(t)$ und $Z(t)$ auf und lösen Sie diese für die folgenden Anfangsbedingungen.
			\[
				t_0 \define 0 \separate M_0\define M(0)\define N \separate Z_0\define Z(t) \define 0
			\]
		}
		\item{
			\textbf{Zusatz:} Skizzieren Sie $M(t)$ und $Z(t)$ für $t\in\setR^+$ und den folgenden Parametern.
			\[
				N = 10000\separate \delta = \frac{1}{3} \separate N_0 = 100
			\]
		}
		\item{
			\textbf{Zusatz:} Wie viele gesunde Menschen konnten gerettet werden? Wie verändert sich diese Zahl, wenn mit einem zweiten Schiff doppelt so viele Menschen pro $\tau$ evakuiert werden können?
		}
	\end{atiSubtasks}
\end{atiTask}