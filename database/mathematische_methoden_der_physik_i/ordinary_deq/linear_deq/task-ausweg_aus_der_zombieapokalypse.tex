\begin{atiTask}[
	title = Ein Ausweg aus der Zombieapokalypse
]
	Auf einer kleinen Insel gerät ein Virus in Umlauf, der die Bevölkerung in Zombies verwandelt.
	% Wenig später bricht auch auf einer Nachbarinsel eine merkwürdige Krankheit mit ähnlichen Symptomen aus.
	Auf dieser Insel wird sofort mit der Evakuierung aller gesunden Menschen begonnen.
	Für kleine  Zeitspannen $\tau\in\setR^+$ und Parameter $\delta,N_0\in\setR^+$ kann der Anteil der gesunden Menschen, die sich infizieren, durch $\tau\cdot\delta$ und die Anzahl der Menschen, die die Insel per Schiff verlassen, durch $\tau\cdot N_0$ approximiert werden.
	% Für kleine Zeitspannen $\tau\in\setR^+$ kann die Anzahl der Menschen, die die Insel per Schiff verlassen, durch $\tau\cdot N_0$ genähert werden, wobei $N_0\in\setR^+$.
	% Infizierte können die Insel nicht verlassen.
	\begin{atiSubtasks}
		\item{
			Stellen Sie ein Differentialgleichungssystem auf, welches den oben genannten Bedingungen genügt.
			Verwenden Sie $M(t)\in\setR^+$ für die Anzahl der gesunden Menschen, $Z(t)\in\setR^+$ für die Anzahl der Infizierten und $t\in\setR^+$ als freien Parameter der Zeit.
			% Nehmen Sie dabei an, $M(t)$ und $Z(t)$ bezeichnen die Anzahl der gesunden Menschen und die Anzahl der Zombies für alle Zeiten $t\in\setR^+$.
			% Lösen Sie dann dieses System für die folgenden Anfangsbedingungen.
			% Stellen Sie die zugehörigen Differentialgleichungen für $M(t)$ und $Z(t)$ auf und lösen Sie diese für die folgenden Anfangsbedingungen.

			% \begin{atiNote}
			% 	Betrachten Sie zunächst nur die Infizierten und gesunden Menschen zum Zeitpunkt $t+\tau$ und überführen Sie die Differenzengleichung durch Grenzwertbildung in die gesuchte Differentialgleichung.
			% \end{atiNote}
		}
		\item{
			Lösen Sie das erhaltene Differentialgleichungssystem für die folgenden Anfangsbedingungen.
			\[
				t_0 \define 0 \separate M_0\define M(0)\define N\in\setN \separate Z_0\define Z(t) \define 0
			\]
		}
		\item{
			\textbf{Zusatz:} Skizzieren Sie $M(t)$ und $Z(t)$ für $t\in\setR^+$ und die folgenden Parameter.
			\[
				N \define 10000\separate \delta \define \frac{1}{3} \separate N_0 \define 100
			\]
		}
		\item{
			\textbf{Zusatz:}
			Bestimmen Sie die Anzahl der geretteten Menschen.
			Berechnen Sie zudem die Änderung dieser Anzahl, wenn statt einem Einzigem zwei identische Schiffe zum Einsatz kommen.
			% Wie viele gesunde Menschen konnten gerettet werden? Wie verändert sich diese Zahl, wenn mit einem zweiten Schiff doppelt so viele Menschen pro $\tau$ evakuiert werden können?
		}
	\end{atiSubtasks}
\end{atiTask}