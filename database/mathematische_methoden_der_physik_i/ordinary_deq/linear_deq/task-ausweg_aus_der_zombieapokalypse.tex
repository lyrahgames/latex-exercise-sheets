\begin{atiTask}[
	title = Ein Ausweg aus der Zombieapokalypse
]
	Auf einer kleinen Insel gerät ein Virus in Umlauf, der die Bevölkerung in Zombies verwandelt.
	Auf dieser Insel wird sofort mit der Evakuierung aller gesunden Menschen begonnen.
	Für kleine  Zeitspannen $\tau\in\setReal^+$ und Parameter $\delta,N_0\in\setReal^+$ kann der Anteil der gesunden Menschen, die sich infizieren, durch $\tau\cdot\delta$ und die Anzahl der Menschen, die die Insel per Schiff verlassen, durch $\tau\cdot N_0$ approximiert werden.
	\begin{atiSubtasks}
		\item{
			Stellen Sie ein Differentialgleichungssystem auf, welches den oben genannten Bedingungen genügt.
			Verwenden Sie $M(t)\in\setReal^+$ für die Anzahl der gesunden Menschen, $Z(t)\in\setReal^+$ für die Anzahl der Infizierten und $t\in\setReal^+$ als freien Parameter der Zeit.
		}
		\item{
			Lösen Sie das erhaltene Differentialgleichungssystem für die folgenden Anfangsbedingungen.
			\[
				t_0 \define 0
				\separate
				M_0\define M(0)\define N\in\setNatural
				\separate
				Z_0\define Z(t) \define 0
			\]
		}
		\item{
			\textbf{Zusatz:} Skizzieren Sie $M(t)$ und $Z(t)$ für $t\in\setReal^+$ und die folgenden Parameter.
			\[
				N \define 10000
				\separate
				\delta \define \frac{1}{3}
				\separate
				N_0 \define 100
			\]
		}
		\item{
			\textbf{Zusatz:}
			Bestimmen Sie die Anzahl der geretteten Menschen.
			Berechnen Sie zudem die Änderung dieser Anzahl, wenn statt einem Einzigem zwei identische Schiffe zum Einsatz kommen.
		}
	\end{atiSubtasks}
\end{atiTask}