\begin{atiTask}[
	topic = Gewöhnliche Differentialgleichungen,
	subtopic = Separable Differentialgleichungen,
	title = Homogene Differentialgleichungen,
	language = Deutsch
]
	Eine Funktion von zwei Variablen heißt homogen vom Grad $k$, wenn für einen beliebigen Parameter $\lambda$ das Folgende gilt.
	\[
		f(\lambda x, \lambda y) = \lambda^k f(x,y)
	\]
	Dementsprechend nennt man eine Differentialgleichungen der folgenden Form auch homogen, wenn $f$ und $g$ homogene Funktionen vom gleichen Grad sind.
	\[
		y' = \frac{f(x,y)}{g(x,y)}
	\]

	\begin{atiSubtasks}
		\item{
		\label{subtask:homogene-differentialgleichung-a}
			Lösen Sie die folgende homogene Differentialgleichung.
			\[
				2xyy^\prime = 3y^2 - x^2
			\]
			Anleitung: Führen Sie eine neue Variable $z(x)$ gemäß $y(x)\reverseDefine x\cdot z(x)$ ein und behandeln Sie die für $z(x)$ entstehende Differentialgleichung mit der Methode der Trennung der Variablen.
		}
		\item{
			Lösen Sie die folgende Differentialgleichung, die nicht homogen ist.
			\[
				y' = \frac{y+x-2}{y-x+4}
			\]
			Schuld daran sind die beiden additiven Konstanten in Zähler und Nenner des Bruches auf der rechten Seite.
			Gehen Sie in zwei Schritten nach folgender Anleitung vor.
			\begin{atiItems}
				\item Führen Sie die neue Variablen $v\define y-y_0$ und $u\define x-x_0$ ein und bestimmen Sie $x_0$ und $y_0$, sodass die neue Differentialgleichung in den Variablen $u$ und $v$ homogen ist (Gleichungssystem mit zwei Unbekannten).

				\item Verfahren Sie mit der Substitution $v(u)\reverseDefine u\cdot z(u)$ weiter, wie in Teilaufgabe \ref{subtask:homogene-differentialgleichung-a}.
			\end{atiItems}
		}
		\item{
			Machen Sie in beiden Fällen die Probe durch Einsetzen Ihrer Lösung $y(x)$ in die ursprüngliche Differentialgleichung.
		}
	\end{atiSubtasks}
\end{atiTask}