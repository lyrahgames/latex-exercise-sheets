\begin{atiTask}[
	title = Eine Zombieapokalypse,
	language = Deutsch
]
	Auf einer kleinen Insel gerät ein Virus in Umlauf, der die Bevölkerung in Zombies verwandelt.
	Jeder Infizierte hat in einer Zeitspanne $\tau\in\setR^+$ Kontakt mit $\tau\cdot k$ anderen Personen, die teilweise ebenfalls infiziert, teilweise aber auch gesunde Menschen sind, wobei $k\in\setR^+$ gilt.
	Gerät ein gesunder Mensch in Kontakt mit einem Zombie, so wird dieser infiziert.
	\medskip
	\begin{atiSubtasks}
		\item{
			Stellen Sie eine Differentialgleichung auf, die dieser Zombieapokalypse genügt.
			Verwenden Sie $N\in\setN$ für die Größe der Inselbevölkerung, $Z(t)\in[0,N]$ für die Anzahl der Infizierten, $M(t)\in[0,N]$ für die Anzahl der Gesunden und $t\in\setR^+$ als freien Parameter der Zeit.

			\begin{atiNote}
				Betrachten Sie zunächst nur die Infizierten zum Zeitpunkt $t+\tau$ und überführen Sie die Differenzengleichung durch Grenzwertbildung in die gesuchte Differentialgleichung.
			\end{atiNote}
		}
		\item{
			Lösen Sie diese Differentialgleichung und das folgende Anfangswertproblem.
			\[
				t_0\define 0\separate Z_0\define Z(0) \define \frac{N}{21}
			\]
		}
		\item{
			Skizzieren Sie $Z(t)$ und $M(t)$ für $k=2$, $N=1050$ und $t\in\setR^+$.
		}
		\item{
			\textbf{Zusatz:} Ab wann ist nur noch weniger als $1\unit{\%}$ der Bevölkerung nicht infiziert?
			Wie beeinflussen die Parameter $k$ und $N$ diesen Zeitpunkt?
		}
	\end{atiSubtasks}
\end{atiTask}