\begin{atiTask}[
	title = Klassifikation von gewöhnlichen Differentialgleichungen,
	language = Deutsch,
]
	Klassifizieren Sie die folgenden gewöhnlichen Differentialgleichungen durch ihre Ordnung, Homogenität, Linearität und Separabilität.
	Lösen Sie zudem die separablen Differentialgleichungen.
	% und führen Sie die Probe durch Einsetzen Ihrer Lösung in die ursprüngliche Differentialgleichung durch.
	\begin{atiSubequations}
	\begin{multicols}{2}
		% \begin{minipage}{0.5\textwidth}
		\item{\locallabel{dgl1}
			\nfdrv[2]{y(x)}{x} + 2\fdrv{y(x)}{x} = y
		}
		\item{\locallabel{dgl2}
			\fdrv{y(x)}{x} + \sin y - x^2 = 0
		}
		\item{\locallabel{dgl3}
			y' + \tan(x)\cdot y = 0
		}
		% \end{minipage}
		% \begin{minipage}{0.5\textwidth}
		\item{\locallabel{dgl4}
			\frac{y''}{y'} + x = 0
		}
		\item{\locallabel{dgl5}
			yy'-x = 0
		}
		\item{\locallabel{dgl6}
			\frac{x+1}{y+2} = \fdrv{y(x)}{x}
		}
		% \end{minipage}
	\end{multicols}
		\item{\locallabel{dgl7}
			\sqrt{y^2 + 3a^2 + ya\curvb{2-\frac{4a}{2y}}} + \fdrv{y(x)}{x}\sqrt{x^2-a^2} = \sqrt{x+a}
		}
	\end{atiSubequations}
\end{atiTask}
\begin{atiSolution}
	\begin{atiSubtaskSolutions}
		\item[\localref{dgl1}]{
			Durch eine einfache Umformung erhalten wir die folgende Differentialgleichung.
			\[
				y'' + 2y' -y = 0
			\]
			Es handelt sich um eine gewöhnliche, lineare, homogene, nicht-separable Differentialgleichung 2.Ordnung.
		}
		\item[\localref{dgl2}]{
			Durch eine einfache Umformung erhalten wir die folgende Differentialgleichung.
			\[
				y'+ \sin y = x^2
			\]
			Es handelt sich um eine gewöhnliche, nicht-lineare, nicht-separable Differentialgleichung 1.Ordnung.
		}
		\item[\localref{dgl3}]{
			Durch eine Äquivalenzumformung lässt sich die Differentialgleichungen in die beiden folgenden Formen bringen.
			\[
				y' + \tan (x)y = 0 \separate y' = -y\tan x
			\]
			Es handelt sich um eine gewöhnliche, lineare, homogene, separable Differentialgleichung 1.Ordnung.
			Durch Anwendung der Methode der Trennung der Variablen, erhält man die folgende Lösung.
			\[
				\frac{y'(x)}{y(x)} = -\tan x \implies \integral{y_0}{y}{\frac{1}{s}}{s} = \integral{x_0}{x}{-\tan s}{s}
			\]
			\[
				\implies \ln\abs{\frac{y(x)}{y_0}} = \ln\abs{\frac{\cos x}{\cos x_0}} \implies \abs{y(x)} = \abs{\frac{y_0}{\cos x_0}} \abs{\cos x}
			\]
		}
		\item[\localref{dgl4}]{
			Wir formen wieder die gegebene Differentialgleichung um und erhalten die folgenden beiden Ausdrücke.
			\[
				y'' = -xy' \separate y'' + xy' = 0
			\]
			Es handelt sich um eine gewöhnliche, lineare, homogene, nicht-separable Differentialgleichung 2.Ordnung.
			Durch die Substitution mit $z\define y'$ lässt sich zudem noch zeigen, dass sie in eine separable Differentialgleichung umgeformt werden kann.
		}
		\item[\localref{dgl5}]{
			Es handelt sich um eine gewöhnliche, nicht-lineare, separable Differentialgleichung 1.Ordnung.
			Es lässt sich hier keine Homogenität definieren, da sie nicht linear ist.
			Die Methode der Trennung der Variablen liefert dann die folgende Lösung.
			\[
				\integral{x_0}{x}{y(s)y'(s)}{s} = \integral{x_0}{x}{s}{s} \implies \integral{y_0}{y(x)}{s}{s} = \frac{y^2(x)- y^2_0}{2} = \frac{x^2-x_0^2}{2}
			\]
			\[
				\implies y^2(x) = x^2 - x_0^2 + y_0^2 \implies y(x) = \pm\sqrt{x^2-x_0^2+y_0^2}
			\]

		}
		\item[\localref{dgl6}]{
			Durch die Trennung von Zähler und Nenner lässt sich die Differentialgleichung in eine Form bringen, an der sich ihre Eigenschaften ablesen lassen.
			\[
				y' = (x+1)\frac{1}{y+2}
			\]
			Es handelt sich um eine gewöhnliche, nicht-lineare, separable, Differentialgleichung 1.Ordnung.
			Die Methode der Trennung der Variablen liefert dann die folgende Lösung.
			\[
				\boxb{y(x)+2}y'(x) = x+1 \implies \integral{y_0}{y(x)}{s+2}{s} = \integral{x_0}{x}{s+1}{s}
			\]
			\[
				\implies \frac{1}{2}\curvb{\boxb{y(x)+2}^2 - \boxb{y_0 + 2}^2} = \frac{1}{2}\curvb{(x+1)^2 - (x_0+1)^2}
			\]
			\[
				\implies \boxb{y(x)+2}^2 = (x+1)^2 - (x_0+1)^2 + \curvb{y_0+2}^2
			\]
			\[
				\implies y(x) = -2 \pm \sqrt{(x+1)^2 - (x_0+1)^2 + \curvb{y_0+2}^2}
			\]
		}
		\item[\localref{dgl7}]{
			Berechnet man den ersten Term auf der linken Seite dieser Gleichung, so ist es möglich die zweite binomische Formel zu verwenden.
			In diesem Falle erhält man das folgende Resultat.
			\[
				\abs{y+a} + y'\sqrt{x^2-a^2} = \sqrt{x+a}
			\]
			Es handelt sich um eine gewöhnliche, nicht-lineare, nicht-separable Differentialgleichung 1.Ordnung.
			Ist jedoch durch gewisse Bedingungen festgelegt, dass $y+a$ nicht negativ wird, so kann die gegebene Differentialgleichung auch als lineare, inhomogene Differentialgleichung geschrieben werden.
		}
	\end{atiSubtaskSolutions}
\end{atiSolution}