\begin{atiTask}[
  title = Der Resonanzfall
]
  Betrachten Sie die Differentialgleichung
  \[
    ay'' + by' + cy = A e^{\varrho x}
  \]
  mit Konstanten $a$, $b$, $c$, $A$ und $\varrho$.
  $\varrho$ erfülle die charakteristische Gleichung
  \[
    a\varrho^2 + b \varrho + c = 0 \ .
  \]
  Hier versagt der Ansatz $y_\text{p} = \alpha e^{\varrho x}$.
  Bestimmen Sie die Partikulärlösung, indem Sie für den Parameter $\alpha$ eine Variation der Konstanten durchführen.
  Unterscheiden Sie dabei die folgenden Fälle.
  \begin{atiSubtasks}
    \item{
      Die Inhomogenität fällt mit einer der Lösungen der charakteristischen Gleichung zusammen.
    }
    \item{
      Die Inhomogenität fällt mit beiden Lösungen der charakteristischen Gleichung zusammen.
    }
  \end{atiSubtasks}
\end{atiTask}