\begin{atiTask}[
  title = Das Phasenportrait
]
  Die von der Ortskoordinate $x$ als Abszisse und der Geschwindigkeit $v$ aufgespannte Ebene heißt Phasenebene.
  Die Kurven $v=v(x)$ heißen Phasenbahnen oder Phasentrajektorien und zusammengenommen Phasenportrait einer Bewegung.
  \begin{atiSubtasks}
    \item{
      Konstruieren Sie das Phasenportrait des harmonischen Oszillators, indem Sie aus den Lösungen $x(t)$ und $v=\tdrv{x}(t)$ der Schwingungsgleichung
      \[
        \ttdrv{x} + \omega_0^2 x = 0
      \]
      den Zeitparameter $t$ eliminieren.
    }
    \item{
      Skizzieren Sie die Phasenbahnen für verschiedene Anfangsbedingungen.
      Wie verhalten sich diese zueinander?
    }
    \item{
      Berechnen Sie die von Ihnen umschlossene Fläche in der $(\tdrv{x},x)$-Phasenebene.
    }
    \item{
      Ein schwingendes System werde durch die Differentialgleichung
      \[
        \ttdrv{y} + 2a\tdrv{y} + (a+2)y=0
      \]
      beschrieben.
      Bestimmen Sie den Parameter $a$, sodass der aperiodische Grenzfall eintritt.
    }
  \end{atiSubtasks}
\end{atiTask}