\begin{atiTask}[
  title = Räuber und Beute
]
  Betrachten Sie die folgenden gekoppelten Differentialgleichungen mit den Koeffizienten $\alpha,\beta,\gamma,\delta\in\setReal^+$.
  \begin{align*}
    \timeDerivative{H} &= \alpha H - \beta HF \\
    \timeDerivative{F} &= -\gamma F + \delta HF
  \end{align*}
  \begin{atiSubtasks}
    \item{
      Interpretieren Sie kurz, was die einzelnen Terme im Kontext eines einfachen Räuber-Beute-Modells bedeuten.
      Erweitern Sie die Differentialgleichungen der Räuber, so dass ein konstanter Abschluss (etwa durch Jäger) berücksichtigt wird.
    }
    \item{
      Leiten Sie aus dem gegebenem Differentialgleichungssystem (ohne Jäger) eine neue Differentialgleichung für $F(H)$ her.
      Bringen Sie diese in die bekannte Form einer exakten Differentialgleichung.
    }
    \item{
      Zeigen Sie, dass $\Lambda = \frac{1}{FH}$ ein integrierender Faktor ist und lösen Sie die gefundene Differentialgleichung nach der Ihnen bekannten Methode.
    }
    \item{
      Bestimmen Sie die Gleichgewichtspunkte, an denen die folgende Gleichung gilt, ohne die Verwendung der impliziten Lösung.
      \[
        \timeDerivative{F} + \timeDerivative{H} = 0
      \]
    }
  \end{atiSubtasks}
\end{atiTask}