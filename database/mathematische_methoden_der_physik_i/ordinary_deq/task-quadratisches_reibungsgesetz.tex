\begin{atiTask}[
  title = Quadratisches Reibungsgesetz
]
  Betrachten Sie die folgende Differentialgleichung mit den Koeffizienten $m,\gamma, k\in\setReal^+$.
  \[
    m\timeSecondDerivative{y} \pm \gamma \timeDerivative{y}^2 + ky = 0
  \]
  Das Vorzeichen vor dem quadratischen Reibungsterm wirke immer so, dass die Reibung die Bewegung behindert.
  Diese nichtlineare Differentialgleichung kann gelöst werden, wenn man sich zunutze macht, dass die unabhängige Variable $t$ nicht vorkommt.
  Substituieren Sie $p=\timeDerivative{y}$, um auch die Ableitungen nach $t$ zu eliminieren.
  Vergleichen Sie Ihr Ergebnis mit der Bernoulli-Gleichung und substituieren Sie angemessen.
  Bestimmen Sie die Lösung für die Anfangsbedingungen
  \[
    \timeDerivative{y}(0) = 0
  \]
  und $y>0$ für kleine $t$ (so kleine Zeiten, dass sich das Vorzeichen im Reibungsterm nicht umkehrt).
  Es soll hier genügen $\timeDerivative{y}$ zu bestimmen.
\end{atiTask}