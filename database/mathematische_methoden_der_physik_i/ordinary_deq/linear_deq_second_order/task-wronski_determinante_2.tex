\begin{atiTask}[
  title = Die Wronski-Determinante
]
  \begin{atiSubtasks}
    \item{
      Drücken Sie die Ableitung $W'$ der Wronski-Determinante
      \[
        W(x) = y_1(x)y'_2(x)-y_2(x)y'_1(x)
      \]
      mithilfe der Differentialgleichung
      \[
        ay'' + by' + cy = 0
      \]
      durch die Wronski-Determinante selbst aus.
    }
    \item{
      Lösen Sie die so entstehende gewöhnliche Differentialgleichung für $W$.
      Zeigen Sie anhand dieser Lösung die folgende Aussage.
      \[
        \exists x\in\setR\colon W(x) \neq 0 \iff \forall x\in\setR\colon W(x) \neq 0
      \]
      % Zeigen Sie anhand dieser Lösung, dass die Wronski-Determinante entweder identisch (d.h. für alle $x$) oder gar nicht verschwindet.
    }
    \item{
      Betrachten Sie nun eine lineae Differentialgleichung zweiter Ordnung mit nicht konstanten Koeffizienten.
      \[
        y'' + p(x)y' + q(x)y = 0
      \]
      Geben Sie die zugehörige Wronski-Determinante an.
    }
    \item{
      Bestimmen Sie mit diesem Verfahren die Wronski-Determinante der Differentialgleichung
      \[
        y'' + \frac{1}{x}y' - \frac{m^2}{x^2}y = 0
      \]
    }
    \item{
      \textbf{Zusatz:}
      Überzeugen Sie sich, dass $y_1(x)=x^m$ eine Lösung der Gleichung ist.
      Konstruieren Sie eine zweite Lösung aus dem Ansatz $y_2(x) = u(x)y_1(x)$, indem Sie eine Differentialgleichung für $u(x)$ aufstellen und diese durch zweimalige Integration lösen.
      Überprüfen Sie durch Einsetzen, dass auch $y_2$ eine Lösung der obigen Differentialgleichung ist.
    }
  \end{atiSubtasks}
\end{atiTask}