\begin{atiTask}[
	title = Die Wronski-Determinante
]
	Gegeben sei die folgende Differentialgleichung, wobei $p$ und $q$ Funktionen in Abhängigkeit von $x$ darstellen.
	Weiterhin seien $y_1$ und $y_2$ zwei Lösungen dieser Gleichung.
	\[
		y'' + p(x)y' + q(x)y = 0
	\]
	\begin{atiSubtasks}
		\item{\locallabel{a}
			Zeigen Sie, dass die Wronski-Determinante $W$ der folgenden separablen Differentialgleichung genügt und lösen Sie diese Gleichung durch Trennung der Variablen.
			\[
				W' = -p(x)\cdot W
			\]
			Damit ist es möglich die Wronski-Determinante direkt aus der zugehörigen Differentialgleichung ohne Kenntnis der Lösung dieser zu bestimmen.
		}
		\item{
			Es sei die Lösung $y_1$ bereits bekannt.
			Interpretieren Sie die folgende Gleichung als inhomogene Differentialgleichung erster Ordnung für $y_2$ und lösen Sie diese durch Variation der Konstanten.
			\[
				y_1(x)y_2'(x)-y_2(x)y_1'(x) = W(x)
			\]
			So können Sie die zweite Fundamentallösung aus der Kenntnis der ersten Fundamentallösung und der Wronski-Determinante bestimmen.
		}
		\item{
			Betrachten Sie die folgende Differentialgleichung mit den reellen konstanten Koeffizienten $a$, $b$ und $c$.
			\[
				ay'' + by' + cy = 0
			\]
			\begin{atiSubsubtasks}
				\item{
					Bestimmen Sie die Wronski-Determinante dieser Differentialgleichung mit der in \localref{a} beschriebenen Methode.
				}
				\item{
					Nehmen Sie an, dass die folgende Lösung bekannt ist, und bestimmen Sie durch Verwendung des zuvor beschriebenen Verfahrens die zweite Fundamentallösung.
					\[
						y_1(x) = e^{\lambda_1 x}
						\separate
						\lambda_1 \define -\frac{b}{2a} + \frac{1}{2a}\sqrt{b^2-4ac}
					\]
				}
			\end{atiSubsubtasks}
		}
		\item{
			\textbf{Für Interessierte:}
			Gegeben sei die folgende Differentialgleichung mit der reellen Konstanten $m$.
			\[
				y'' + \frac{1}{x} y' - \frac{m^2}{x}y = 0
			\]
			\begin{atiSubsubtasks}
				\item{
					Bestimmen Sie die Wronski-Determinante dieser Differentialgleichung.
				}
				\item{
					Überzeugen Sie sich, dass die folgende Funktion eine Lösung dieser Differentialgleichung ist und berechnen Sie nach dem zuvor beschriebenen Verfahren die zweite Fundamentallösung.
					\[
						y_1(x) = x^m
					\]
					Machen Sie die Probe für diese zweite Lösung und geben Sie auch die allgemeine Lösung der Differentialgleichung an.
				}
			\end{atiSubsubtasks}
		}
	\end{atiSubtasks}
\end{atiTask}