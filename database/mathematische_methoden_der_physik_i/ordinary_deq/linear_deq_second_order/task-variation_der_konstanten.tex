\begin{atiTask}[
  title = Die Variation der Konstanten
]
  Betrachten Sie die Differentialgleichung
  \[
    y'' + 5y' + 4y = \cos 2x \ .
  \]
  Zur Bestimmung der allgemeinen Lösung soll nicht die Methode des Koeffizientenvergleichs benutzt werden, sondern nach einem allgemeineren Verfahren vorgegangen werden:
  \begin{atiSubtasks}
    \item{
      Bestimmen Sie zunächst die Fundamentallösungen $y_1$ und $y_2$ der homogenen Gleichung.
    }
    \item{
      Variieren Sie die beiden Konstanten in
      \[
        y_\m{h} = c_1e^{-4x} + c_2e^{-x} \ ,
      \]
      indem Sie diese durch Funktionen $u(x)$ und $v(x)$ ausdrücken.
      Berechnen Sie die erste und die zweite Ableitung.
    }
    \item{
      Setzen Sie die dadurch erhaltenen Funktionen in die inhomogene Differentialgleichung ein und vereinfachen Sie die daraus resultierende Gleichung, indem Sie benutzen, dass $y_1$ und $y_2$ die homogene Differentialgleichung lösen.
    }
    \item{
      Vereinfachen Sie die Gleichung weiter, indem Sie fordern, dass
      \begin{equation}
        \locallabel{equ1}
        u'(x)e^{-4x} + v'(x)e^{-x} = 0
      \end{equation}
      gilt.
      Begründen Sie, warum diese Forderung zulässig ist.
    }
    \item{
      Leiten Sie aus der Gleichung \localref{equ1} eine weitere Bedingung ab, die Ihnen bei der Vereinfachung hilft.
    }
    \item{
      Sie erhalten nun eine einfachere Form der ursprünglichen Differentialgleichung
      \begin{equation}
        \locallabel{equ2}
        -4u'(x)e^{-4x} - v'(x)e^{-x} = \cos 2x
      \end{equation}
      Bestimmen Sie ausgehend von den Gleichungen (\localref{equ1}) und (\localref{equ2}) die Funktionen $u$ und $v$.
    }
    \item{
      Führen Sie eine Probe für die so erhaltene allgemeine Lösung durch.
    }
  \end{atiSubtasks}
\end{atiTask}