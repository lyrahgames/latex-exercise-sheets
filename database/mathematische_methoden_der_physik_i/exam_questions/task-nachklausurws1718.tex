\begin{atiTask}[
	title = Weitere Fragen,
	language = Deutsch
]
	\begin{atiSubtasks}
%	\item 
%	Eine Masse hängt an einer Feder ohne Dämpfung mit Eigenfrequenz $\omega_0$ und wird aus der Anfangslage losgelassen. 
%	Berechnen Sie die Maximalgeschwindigkeit $v_\mathrm{max}$, die die Masse bei einer harmonischen Schwingung erreicht. Bestimmen Sie danach die Maximalgeschwindigeit, die die Masse bei kritischer Dämpfung (aperiodischer Grenzfall) erreicht im Verhältnis zu $v_\mathrm{max}$!
	%Bestimmen Sie danach Wie groß ist im Verhältnis dazu die Maximalgeschwindigkeit, wenn das Experiment in einer Flüssigkeit wiederholt wird, die die Bewegung gerade kritisch dämpft?
	
%	\item Begründen Sie, welche der folgenden Differentialgleichungen den exponentiellen Zerfall mit konstanter Zufuhr beschreibt:
%	\begin{atiSubequations}
%	\item {\dot{y}=-ay+b\quad a,b>0}
%	\item {\dot{y}=e^{\alpha t},\quad \alpha<0}
%	\item {\ddot{y}+\gamma \dot{y}+\omega_0^2y=c,\quad c,\gamma,\omega_0^2>0}
%	\end{atiSubequations} 
	\item Die Differentialgleichung
	\[
	y''-2yy'=0
	\]
	hat die spezielle Lösung $y_1(x)=\tan x$, jedoch ist $y_2=cy_1$ ($c\neq 0$) keine (spezielle) Lösung. Geben Sie eine kurze (!) Begründung dafür an.
	
	\item Gegeben sei die Eulersche Differentialgleichung für die Funktion $y(x)$
	\begin{equation*}
	x^2y''-xy'+2y=0,\quad x>0.
	\end{equation*}
	Zeigen Sie, dass die Transformation $x=e^t$ diese Differentialgleichung in eine Differentialgleichung zweiter Ordnung mit konstanten Koeffizienten überführt. (Die Lösung der Differentialgleichung ist nicht verlangt.)
	\item Ein Teilchen der Masse $m$ führt eine eindimensionale Bewegung unter dem Einfluss der Kraft
	\begin{equation*}
	F(x)=\frac{2a}{x^3}-\frac{b}{x^2},\quad a,b>0
	\end{equation*}
	aus.
	Bestimmen Sie den Gleichgewichtspunkt und seine Stabilität, und berechnen Sie die Frequenz kleiner Schwingungen um diese Gleichgewichtlage.
	\item Zeigen Sie durch Berechnung der Wronski-Determinante, dass die beiden Funktionen
	\begin{equation*}
	y_1(x)=e^{ax},\quad y_2(x)=xe^{ax}\quad (a\neq 0)
	\end{equation*}
	ein Fundamentalsystem bilden und bestimmen Sie die Differentialgleichung deren allgemeine Lösung durch Linearkombination dieses Funktionenpaars gegeben ist.
%	\item Gegeben sei die Differentialgleichung 
%	\begin{equation*}
%	(2y-x^2)\D x+\D y=0.
%	\end{equation*}
%	Skizzieren Sie das Richtungsfeld und geben Sie 
	\item Bestimmen Sie alle Skalarfunktionen $v(x,y,z)$, für die das folgende Kurvenintegral wegunabhängig ist!
	\begin{equation*}
	W=\int_C \left(xy\D x+\frac{x^2}{2}\D y+v(x,y,z)\D z\right) %Wallner S 109
	\end{equation*}
	\end{atiSubtasks}
\end{atiTask}
