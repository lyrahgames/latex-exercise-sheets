\begin{atiTask}[
  title = Wie ein Fisch im Wasser
]
  Die Temperaturverteilung in einem See sei gegeben durch die folgende Funktion.
  \[
    D \define \left\{ (x,y,z)\in\setR^3 \middle| z < 0 \right\}
  \]
  \[
    \function{T}{D}{\setR}
    \separate
    T(x,y,z)\define -\curvb{x^2 + \frac{y^2}{4} + 2z^2}
  \]
  \begin{atiSubtasks}
    \item{
      Bestimmen Sie die Isothermen und fertigen Sie eine Skizze dieser in der $yz$-Ebene an.
    }
    \item{
      Ein Fisch im Wasser befinde sich am Punkt $(1,2,-1)$.
      In welche Richtung verändert sich die Temperatur am stärksten?
      Wird es wärmer oder kälter?
    }
    \item{
      Der Fisch bewege sich auf dem folgenden Weg.
      \[
        \function{r}{[0,2\pi]}{\setR^3}
        \separate
        r(t)\define 2\cos t \vec{i} + \sin t \vec{j} + (\cos t - 2)\vec{k}
      \]
      Wann ist die Temperatur für ihn am größten und wann am kleinsten?
      Wo befindet er sich zu diesem Zeitpunkt?
      Skizzieren Sie die Temperatur $T\circ r$, die der Fisch entlang seines Weges erfährt.
    }
  \end{atiSubtasks}
\end{atiTask}