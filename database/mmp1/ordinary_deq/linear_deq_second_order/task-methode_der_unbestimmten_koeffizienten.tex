\begin{atiTask}[
  title = Die Methode der unbestimmten Koeffizienten
]
  Lösen Sie die folgenden Differentialgleichungen mithilfe der Methode der unbestimmten Koeffizienten für die nebenstehenden Anfangsbedingungen.
  Führen Sie eine Probe durch.
  \begin{atiSubequations}
    \item{
      3y'' - 6y' - 24y = 72x^2 - 12x - 6
      \separate
      y(0) = -1
      \separate
      y'(0) = 8
    }
    \item{\locallabel{c}
      y'' + y' + y = (2+x)\cos x
      \separate
      y(0) = -1
      \separate
      y'(0) = 1
    }
    \item{\locallabel{d}
      y'' + y' + y = 2+x + \cos x
      \separate
      y(0) = -7
      \separate
      y'(0) = 8
    }
    \item{\locallabel{e}
      y^{(3)} - 12y' + 16y = 32x - 8
      \separate
      y(0) = -
      \separate
      y'(0) = 1
      \separate
      y''(0) = -28
    }
    \item{
      y'' - 9y = e^{3x}
      \separate
      y(0) = 3
      \separate
      y'(0) = \frac{55}{6}
    }
  \end{atiSubequations}
  \begin{atiNote}
    \begin{atiItems}
      \item[zu \localref{c}:]{
        Verwenden Sie einen Ansatz der Form
        \[
          y_\text{p} = \roundBrackets{a_0 + a_1 x}\sin x + \roundBrackets{b_0 + b_1 x}\cos x
        \]
      }
      \item[zu \localref{d}:]{
        Bestimmen Sie nur eine neue Partikulärlösung.
        Die homogene Lösung kann aus \localref{c} übernommen werden.
      }
      \item[zu \localref{e}:]{
        Überlegen Sie sich, wie der Ansatz für Differentialgleichungen zweiter Ordnung auf solche dritter Ordnung erweitert werden kann.
      }
    \end{atiItems}
  \end{atiNote}
\end{atiTask}