\begin{atiTask}[
	title = Integrierende Faktoren
]
	\begin{atiSubtasks}
		\item{
			Die folgende Differentialgleichung sei gegeben.
			\[
				A(x,y) + B(x,y)y' = 0
			\]
			Zeigen Sie durch Verwendung der Integrabilitätsbedingung die beiden folgenden Aussagen.
			\begin{atiItems}
				\item{
					Kann der folgende Ausdruck als eine Funktion $f$ der Variablen $z(x,y)\define xy$ geschrieben werden, so hängt auch der integrierende Faktor $\lambda$ nur von dieser Variable $z$ ab.
					\[
						\frac{\partial_x B(x,y) - \partial_y A(x,y)}{xA(x,y)-yB(x,y)}
					\]
					Geben Sie den Zusammenhang von $\lambda$ und $f$ an.
				}
				\item{
					Kann der folgende Ausdruck als eine Funktion $g$ der Variablen $w(x,y)\define x + y$ geschrieben werden, so hängt auch der integrierende Faktor $\lambda$ nur von dieser Variable $z$ ab.
					\[
						\frac{\partial_x B(x,y) - \partial_y A(x,y)}{A(x,y)-B(x,y)}
					\]
					Geben Sie den Zusammenhang von $\lambda$ und $g$ an.

				}
			\end{atiItems}
		}
		\item{
			Eine der beiden zuvor genannten Eigenschaften trifft auf eine der beiden folgenden Differentialgleichungen zu.
			Finden Sie diesen Fall heraus und berechnen Sie einen integrierenden Faktor mit dem Ergebnis aus dem vorherigen Aufgabenteil.
			\begin{atiSubequations}
				\item{
					(xy-1) + \roundBrackets{x^2-xy}y' = 0
				}
				\item{
					y + \roundBrackets{x - 2x^2y^3}y' = 0
				}
			\end{atiSubequations}
			Lösen Sie die Differentialgleichung mit diesem integrierenden Faktor und machen Sie anschließend die Probe anhand der ursprünglichen Differentialgleichung.
			Die Lösung der verbleibenden Differentialgleichung ist nicht verlangt.
		}
	\end{atiSubtasks}
\end{atiTask}