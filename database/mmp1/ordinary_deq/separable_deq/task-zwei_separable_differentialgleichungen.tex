\begin{atiTask}[
	title = Zwei separable Differentialgleichungen
]
	Lösen Sie die folgenden Differentialgleichungen mittels Trennung der Variablen.
	Überprüfen Sie Ihr Ergebnis, indem Sie eine Probe durchführen.
	\begin{atiSubequations}
		\item{\locallabel{dgl1}
			\frac{1}{\cos x}\leibnizDerivative{y(x)}{x} = -\tan x \cdot y^{-2}
		}
		\item{\locallabel{dgl2}
			xyy' = y-1
		}
	\end{atiSubequations}
\end{atiTask}
\begin{atiSolution}
	\begin{atiSubtaskSolutions}
		\item[\localref{dgl1}]{
			Durch Umformung trennen wir in der Differentialgleichung zuerst $y(x)$ von $x$ und erhalten die folgende Gleichung.
			Hierbei definieren wir $x_0$ und $y_0\define y(x_0)$ als die Anfangsbedingungen einer Lösung der Differentialgleichung.
			\[
				y^2(x)y'(x) = -\tan x \cos x = -\sin x\atiPoints[1]
			\]
			\[
				\implies \integral{x_0}{x}{y(s)^2y'(s)}{s} = \integral{x_0}{x}{-\sin s}{s}
			\]
			Für die linke Seite der Gleichung lässt sich nun die Substitutionsregel verwenden.
			Die rechte Seite ist direkt integrierbar.
			\[
				\integral{x_0}{x}{y(s)^2y'(s)}{s} = \integral{y(x_0)}{y(x)}{s^2}{s} = \frac{1}{3}\boxBrackets{y^3(x)-y^3_0} = \cos x - \cos x_0\atiPoints[2]
			\]
			\[
				\implies y(x) = \sqrt[3]{3\roundBrackets{\cos x - \cos x_0} + y_0^3}\atiPoints[1]
			\]
			Um die Probe durchzuführen, leiten wir als Erstes die erhaltene Lösung ab und substituieren dann geeignete Terme durch $y(x)$.
			Zudem erweitern wir mit $\cos x$ um auf die ursprüngliche Differentialgleichung zu kommen.
			\[
				y'(x) = \frac{-\sin x}{
					\boxBrackets{
						3(\cos x - \cos x_0) + y_0^3
					}^\frac{2}{3}
				} = \frac{-\sin x}{y^2(x)} = -\frac{\sin x \cos x}{\cos x}y^{-2}(x)\atiPoints[\frac{1}{2}]
			\]
			\[
				\implies \frac{y'(x)}{\cos x} = -y^{-2}(x)\tan x\atiPoints[\frac{1}{2}]
			\]
		}
		\item[\localref{dgl2}]{
			Durch Umformung trennen wir in der Differentialgleichung zuerst $y(x)$ von $x$ und erhalten die folgende Gleichung.
			Hierbei definieren wir $x_0$ und $y_0\define y(x_0)$ als die Anfangsbedingungen einer Lösung der Differentialgleichung.
			\[
				\frac{y(x)y'(x)}{y(x)-1} = \frac{1}{x} \implies \integral{x_0}{x}{\frac{y(s)y'(s)}{y(s)-1}}{s} = \integral{x_0}{x}{\frac{1}{s}}{s} \atiPoints[1]
			\]
			Für die linke Seite der Gleichung lässt sich nun wieder entsprechend der Lösung für separable Differentialgleichungen die Substitutionsregel verwenden.
			Die rechte Seite ist durch die Anwendung von Integrationsregeln direkt integrierbar.
			\[
				\integral{x_0}{x}{\frac{y(s)y'(s)}{y(s)-1}}{s} = \integral{y_0}{y(x)}{\frac{s}{s-1}}{s} = \ln\absolute{\frac{x}{x_0}}\atiPoints[1]
			\]
			Die Lösung des Integrals lässt sich wie folgt durch intelligente Addition einer Null im Zähler des Bruches bestimmen.
			\[
				\integral{y_0}{y(x)}{\frac{s}{s-1}}{s} = \integral{y_0}{y(x)}{\frac{s-1+1}{s-1}}{s} = \integral{y_0}{y(x)}{}{s} + \integral{y_0}{y(x)}{\frac{1}{s-1}}{s}\atiPoints[+1]
			\]
			\[
				\implies \integral{y_0}{y(x)}{\frac{s}{s-1}}{s} = y(x)-y_0 + \ln\absolute{\frac{y(x)-1}{y_0-1}} \atiPoints[1]
			\]
			Die endgültige Lösung ist jetzt durch Einsetzen ermittelbar.
			Zu beachten sind lediglich die Singularitäten an den Stellen $x_0 = 0$ und $y_0 - 1 = 0$.
			Durch Sie folgt, dass $x$ beziehungsweise $y(x)-1$ das gleiche Vorzeichen besitzt wie $x_0$ beziehungsweise $y_0-1$.
			\[
				\absolute{\frac{y(x)-1}{y_0-1}}=\frac{y(x)-1}{y_0-1}\separate \absolute{\frac{x}{x_0}} = \frac{x}{x_0}
			\]
			\[
				\implies y(x)-y_0 + \ln\roundBrackets{\frac{y(x)-1}{y_0-1}} = \ln\roundBrackets{\frac{x}{x_0}}
			\]
			\[
				\implies \frac{y(x)-1}{y_0-1}e^{y(x)-y_0} = \frac{x}{x_0}
			\]
			\[
				\implies \boxBrackets{y(x)-1}e^{y(x)} = \roundBrackets{y_0-1}e^{y_0}\frac{x}{x_0}\atiPoints[1]
			\]
			Um die Probe durchzuführen, differenzieren wir als Erstes die erhaltene Lösung implizit und vereinfachen die erhaltenen Terme.
			\[
				y'(x)e^{y(x)} + \boxBrackets{y(x)-1}y'(x)e^{y(x)} = y(x)y'(x)e^{y(x)} = \frac{y_0-1}{x_0}e^{y_0}\atiPoints[\frac{1}{2}]
			\]
			Um zur ursprünglichen Differentialgleichung zu gelangen, erweitern wir die Gleichung mit $x$ und substituieren die daraus entstehende rechte Seite der Gleichung durch die berechnete Lösung.
			\[
				x y(x) y'(x)e^{y(x)} = \frac{y_0-1}{x_0}e^{y_0}x = \boxBrackets{y(x)-1}e^{y(x)}
			\]
			\[
				\implies x y(x) y'(x) = y(x)-1\atiPoints[\frac{1}{2}]
			\]
		}
	\end{atiSubtaskSolutions}
\end{atiSolution}