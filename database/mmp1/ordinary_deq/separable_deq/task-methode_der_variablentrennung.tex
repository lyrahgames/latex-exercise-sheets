\begin{atiTask}[
	title = Die Methode der Variablentrennung
]
	Lösen Sie die folgenden Differentialgleichungen durch Trennung der Variablen und bestimmen Sie gegebenenfalls die Integrationskonstante, sodass die nebenstehenden Anfangsbedingungen erfüllt sind.
	\begin{atiSubequations}
		\item{
			\label{dgl-1}
			y' = \frac{x e^{-y}}{x^2 + 1} \separate y(1) = 0
		}
		\item{
			\label{dgl-2}
			xyy' = \frac{x^2 + 2}{y-1}
		}
		\item{
			\label{dgl-3}
			y' = \frac{x+y}{x+y+2} \separate y(1) = -1
		}
	\end{atiSubequations}
	Machen Sie in allen Fällen die Probe durch Einsetzen Ihrer Lösung in die ursprüngliche Differentialgleichung. Dies ist auch dann verlangt, wenn die Lösung nur in impliziter Form angebbar ist.

	\begin{atiNote}
		Führen Sie in Teilaufgabe \ref{dgl-3} die neue Variable $z(x)\define x + y(x)$ ein.
	\end{atiNote}
\end{atiTask}

\begin{atiSolution}
	\begin{atiSubtaskSolutions}
		\item[\ref{dgl-1}]{
			Separieren Sie $x$ und $y(x)$ auf jeweils eine Seite der Differentialgleichung und integrieren Sie die erhaltene Gleichung.
			\[
				y'(x) = \frac{xe^{-y(x)}}{x^2 + 1} \implies e^{y(x)} y'(x) = \frac{x}{x^2 + 1}
			\]
			\[
				\implies \indefiniteIntegral{e^{y(x)}y'(x)}{x} = \indefiniteIntegral{\frac{x}{x^2+1}}{x}
				\atiPoints[1]
			\]
			Lösen Sie das Integral mithilfe einer logarithmischen Integration oder durch Substitution, indem Sie $x^2$ durch eine geeignete Variable ersetzen.
			\[
				\indefiniteIntegral{e^y}{y} = \frac{1}{2} \indefiniteIntegral{\frac{2x}{x^2+1}}{x}
			\]
			\[
				\implies e^{y(x)} = \frac{1}{2}\ln(x^2 + 1) + C = \ln\sqrt{x^2 + 1} + C
				\atiPoints[1]
			\]
			Notieren Sie die explizite Lösung durch die Anwendung von $\ln$.
			\[
				y(x) = \ln\boxBrackets{\ln\roundBrackets{A\sqrt{x^2+1}}} \separate A\define e^C
				\atiPoints[1]
			\]
			Fordern Sie nun $y(1)\demand 0$, bestimmen Sie die Konstante $A$ und setzen Sie die erhaltene Lösung in die explizite allgemeine Form ein.
			\[
				y(1) \demand 0 \implies 1 = \ln \roundBrackets{A\sqrt{2}} \implies A = \frac{\sqrt{2}}{2}e
				\atiPoints[1]
			\]
			\[
				y(x) = \ln\boxBrackets{\ln\roundBrackets{\frac{e}{2}\sqrt{2(x^2 + 1)}}} = \ln\boxBrackets{1 + \ln\roundBrackets{\frac{1}{2}\sqrt{2(x^2+1)}}}
				\atiPoints[1]
			\]

			Sei $y\reverseDefine \ln u$ mit $u(x) = \ln\roundBrackets{A\sqrt{x^2+1}}$.
			Dann erhält man durch die Anwendung der Kettenregel die folgende Aussage.
			\[
				y'(x) = u'(x) \ln' u(x) = \frac{1}{A\sqrt{x^2+1}} \cdot A \cdot \frac{2x}{2\sqrt{x^2+1}} \cdot \frac{1}{u(x)} = e^{-y(x)} \frac{x}{x^2+1}
				\atiPoints[1]
			\]
		}

		\item[\ref{dgl-2}]{
			Separieren Sie $x$ und $y(x)$ wieder auf jeweils eine Seite der Differentialgleichung
			\[
				xy(x)y'(x) = \frac{x^2+2}{y(x)-1} \implies y(x)\boxBrackets{y(x)-1}y'(x) = \frac{x^2+2}{x}
				\atiPoints[1]
			\]
			Integrieren Sie die rechte Seite der erhaltenen Gleichung durch Polynomintegration und der Umkehrregel.
			\[
				\indefiniteIntegral{\boxBrackets{y^2(x)-y(x)}y'(x)}{x} = \indefiniteIntegral{\roundBrackets{x+\frac{2}{x}}}{x}
			\]
			\[
				\implies \indefiniteIntegral{y^2-y}{y} = \frac{x^2}{2} + 2\ln\absolute{x}
				\atiPoints[1]
			\]
			Lösen Sie nun auch das Integral der rechten Seite durch Polynomintegration und notieren Sie die allgemeine Lösung in impliziter Form.
			\[
				\frac{y^3}{3} - \frac{y^2}{2} = \frac{x^2}{2} + 2\ln\absolute{x} + C
			\]
			\[
				\implies 2y^3 - 3y^2 = 3x^2 + 12 \ln \absolute{x} + D
				\atiPoints[1]
			\]

			Durch implizite Ableitung der allgemeinen Form erhalten Sie Folgendes.
			\[
				6y^2(x)y'(x) - 6y(x)y'(x) = 6x + \frac{12}{x} \implies y(x)\boxBrackets{y(x)-1}y'(x) = x + \frac{2}{x}
			\]
			\[
				\implies xy(x)\boxBrackets{y(x)-1}y'(x) = x^2+2
				\atiPoints[1]
			\]
		}

		\item[\ref{dgl-3}]{
			Definieren Sie $z(x)\define x + y(x)$ und bestimmen Sie die Ableitung von $z$.
			\[
				z'(x) = 1 + y'(x) \implies y'(x) = z'(x) - 1
			\]
			Substituieren Sie nun $x+y(x)$ in der Differentialgleichung durch $z(x)$.
			\[
				y'(x) = \frac{x+y(x)}{x+y(x)+2} \implies z'(x) - 1 = \frac{z}{z+2}
				\atiPoints[1]
			\]
			Führen Sie für die erhaltene Differentialgleichung das Verfahren der Trennung der Variablen durch.
			Separieren Sie $z(x)$ und $x$ auf jeweils eine Seite und integrieren Sie die erhaltene Gleichung.
			\[
				z'(x) = \frac{z(x)}{z(x)+2} + 1 = \frac{2z(x)+2}{z+2} = 2\,\frac{z(x)+1}{z(x)+2}
				\atiPoints[1]
			\]
			\[
				\implies \indefiniteIntegral{\frac{z(x)+2}{z(x)+1}z'(x)}{x} = \indefiniteIntegral{\roundBrackets{1 + \frac{1}{z+1}}}{z} = \indefiniteIntegral{2}{x}
			\]
			\[
				\implies z(x) + \ln\absolute{z(x)+1} = 2x + C
				\atiPoints[1]
			\]
			Führen Sie die Resubstitution durch und geben Sie die allgemeine Lösung in impliziter Form an.
			\[
				x + y(x) + \ln\absolute{x+ y(x) + 1} = 2x + C
			\]
			\[
				\implies x+y(x)+1 = \exp(x-y(x)+C) = Ae^{x-y(x)} \separate A\define e^C
			\]
			\[
				\implies x+y(x) = Ae^{x-y(x)} -1
				\atiPoints[1]
			\]
			Fordern Sie die gegebenen Anfangsbedingungen und bestimmen Sie die Konstante $A$.
			\[
				y(1)\demand -1 \implies 0=Ae^2-1 \implies A= e^{-2}
			\]
			\[
				\implies x+y(x) = \frac{e^{x-y(x)}}{e^2}-1 = e^{x-y(x)-2}-1
				\atiPoints[1]
			\]

			Auch hier ist wieder eine implizite Ableitung notwendig.
			\[
				1+y'(x) = Ae^{x-y(x)} \boxBrackets{1-y'(x)}
			\]
			Durch Verwendung der allgemeinen Lösung erhalten Sie für die Konstante $A$ den folgenden Ausdruck.
			\[
				A = e^{y(x)-x}\boxBrackets{x+y(x)+1}
			\]
			Das Einsetzen dieser Gleichung resultiert dann in der gewünschten Differentialgleichung.
			\[
				1+y'(x) = e^{y(x)-x}\boxBrackets{x+y(x)+1} e^{x-y(x)} \boxBrackets{1-y'(x)}
			\]
			\[
				\implies 1+y'(x) = x+y(x)+1 - \boxBrackets{x+y(x)+1}y'(x)
			\]
			\[
				\implies y'(x)\boxBrackets{x+y(x)+2} = x+y(x)
				\atiPoints[1]
			\]
		}
	\end{atiSubtaskSolutions}
\end{atiSolution}