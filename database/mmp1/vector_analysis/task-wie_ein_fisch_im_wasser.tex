\begin{atiTask}[
  title = Wie ein Fisch im Wasser
]
  Die Temperaturverteilung in einem See sei gegeben durch die folgende Funktion.
  \[
    \function{T}{\set{(x,y,z)\in\setReal^3}{z<0}}{\setReal}
    \separate
    T(x,y,z)\define -\roundBrackets{x^2 + \frac{y^2}{4} + 2z^2}
  \]
  \begin{atiSubtasks}
    \item{
      Bestimmen Sie die Isothermen und fertigen Sie eine Skizze dieser in der $yz$-Ebene an.
    }
    \item{
      Ein Fisch im Wasser befinde sich am Punkt $(1,2,-1)$.
      Bestimmen Sie die Richtung, in die sich die Temperatur am stärksten verändert und entscheiden Sie, ob es wärmer oder kälter wird.
    }
    \item{
      Der Fisch bewege sich auf dem folgenden Weg.
      \[
        \function{r}{[0,2\pi]}{\setReal^3}
        \separate
        r(t)\define 2\cos t \vectorX + \sin t \vectorY + (\cos t - 2)\vectorZ
      \]
      Bestimmen Sie die Zeitpunkte, an denen Fisch die größte und die kleinste Temperatur empfindet und berechnen sie die zugehörigen Position des Fisches.
      Skizzieren Sie die Temperatur $T\composition r$, die der Fisch entlang seines Weges in Abhängigkeit der Zeit erfährt.
    }
  \end{atiSubtasks}
\end{atiTask}