\begin{atiTask}[
  title = Skalarprodukte im Integranden
  %call = Zusatzaufgabe,
]

Berechnen Sie das folgende Integral
\[
I=\iint (3\vec{e}_r+r^2\vec{e}_\phi-2\vec{e}_z)\cdot \D \vec{f}
\]
\begin{atiSubtasks}
\item für die Mantelfläche
\item für Boden- und Deckfläche
\end{atiSubtasks}
des Zylinders $0\leq x^2+z^2\leq R^2, \quad 0\leq z<L$.
Dabei sind $\vec{e}_r$, $\vec{e}_\phi$ und $\vec{e}_z$ die Einheitsvektoren in Zylinderkoordinaten. Überlegen Sie sich $\D\vec{f}$, indem Sie dazu eine Skizze anfertigen. (Der Flächennormalenvektor soll immer aus der Fläche herauszeigen.)
\end{atiTask}

\begin{atiSolution}
\begin{atiSubtasks}
\item Mantelfläche: $\D \vec{f}=\vec{e}_rR\D \vec{\phi}\D z$
\[I=\int_{z=0}^L\int_{\phi=0}^{2\pi} (3\vec{e}_r+r^2\vec{e}_\phi-2\vec{e}_z)\cdot \vec{e}_r R\D \vec{\phi}\D z
\]
Das ergibt
\[I=\int_{z=0}^L\int_{\phi=0}^{2\pi}3R \D \phi\D z=3R\cdot 2\pi L=6\pi R L
\]
\item Bodenfläche: $\D \vec{f}=-\vec{e}_z r\D r\D \vec{\phi}$
\[I=\int_{r=0}^R\int_{\phi=0}^{2\pi} 2r\D r \D \phi = 2\pi r^2|_0^R=2\pi R^2
\]
Deckelfläche: $\D \vec{f}=+\vec{e}_z r\D r\D \vec{\phi}$ $\Rightarrow$ $I=-2\pi R^2$.
\end{atiSubtasks}
\end{atiSolution}