\begin{atiTask}[
  title = Anwendbarkeit des GAUSSschen Satzes,
  %call = Zusatzaufgabe,
]

Gegeben sei das Vektorfeld
\[
\vec{F}=\frac{\vec{r}}{r^3}
\]
\begin{atiSubtasks}
\item Berechen Sie die Divergenz dieses Vektorfeldes.
\item Berechnen Sie den Fluss

\[\oiint\limits_{S_1} \vec{F}\;\mathrm{d}\vec{f}
\]


worin $S_1$ die Einheitskugel mit ihrem Mittelpunkt im Korrdiantenursprung sein soll, direkt aus dem Oberflächenintegral. Kann man dieses Integral mit Hilfe des \textsc{Gauss}schen Satzes berechnen?
\item Wiederholen Sie die Berechnung des Flusses für eine Fläche $S_2$, die die Einheitskugel ist, deren Mittelpunkt der Punkt $M(0,0,2)$ ist. Verifizieren Sie das Resultat mit Hilfe des \textsc{Gauss}schen Satzes, falls dieser anwendbar ist.
\end{atiSubtasks}
\atiNote{für das Oberflächenintegral: Verschieben Sie den Ursprung des Koordinatensystems in den Punkt $M$.}

\end{atiTask}

\begin{atiSolution}
\includepdf[pages=-]{solution-gauss_ii.pdf}
\end{atiSolution}