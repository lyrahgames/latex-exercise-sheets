\begin{atiTask}[
  title = Rotation berechnen
  %call = Zusatzaufgabe,
]
Berechnen Sie die Rotation der gegebenen Vektorfelder im Punkt $(3,4,0)$!
\begin{atiSubequations}
\item{\vec{u}(x,y,z)=3\vec{i}-\pi \vec{j}+16\vec{k}}
\item{\vec{v}(x,y,z)=x\vec{i}}
\item{\vec{w}(x,y,z)=y\vec{i}}
\item{\vec{f}(x,y,z)=(x+2y+3z)\vec{i}+4xyz\vec{j}+x\sin(\pi y+z)\vec{k}}
\item{\vec{g}(x,y,z)=\ln (x+2z)\vec{i}+\frac{x}{y}\vec{k}}
\item{\vec{h}(x,y,z)=\left[\frac{x+z}{y}\vec{i}+\cos^2z\vec{j}+x^5\vec{k}\right]}
\end{atiSubequations}

\end{atiTask}
%aus Felder S 414
%+Mathematika Plot einfügen, wohin zeigt rot, was ist die Aussage (Achse Schaufelrad)
\begin{atiSolution}
%\begin{atiSubequations}
(i) $\curl u=0$, da Vektorfeld konstant (homogen) - (1 Pkt) \\
(ii) $\curl v=0 $, da auf $x$-Richtung nur $y$- und $z$-Ableitungen wirken.(1 Pkt) \\
(iii) $\curl w=-\vec{k}$ - (1 Pkt)\\
(iv) $\curl f=(x\pi\cos (\pi y+z)-4yx)\vec{i}+(3-\sin(\pi y+z)\vec{j}+(4yz-2)\vec{k}\rightarrow (3\pi-48)\vec{i}+3\vec{j}-2\vec{k}$  (1 Pkt)\\
(v) $\curl g=-x/y^2\vec{i}-\vec{j}(1/y-\frac{2}{x+2z})+0\vec{k}$, am Punkt also $(-\frac{3}{16}\vec{i}+\frac{5}{12}\vec{j})$ (1 Pkt)\\
(vi) $\curl f=(2\cos z\sin z)\vec{i}-\vec{j}(5x^4-1/y)+\frac{x+z}{y^2}\vec{k}$, am Punkt also $(1/4-5*81)\vec{j}+\frac{3}{16}\vec{k}=-\frac{1619}{4}\vec{j}+\frac{3}{16}\vec{k}$ (1 Pkt)
%\end{atiSubequation}
\end{atiSolution}