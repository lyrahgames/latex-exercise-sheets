\begin{atiTask}[
  title = Laplaceoperator in 3 Dimensionen
  %call = Zusatzaufgabe,
]

Der \textsc{Laplace}-Operator wird u.a. in der Elektrodynamik auch auf Vektoren angewendet. Allgemein gilt:
\[
\Delta\vec{a}=\Delta (a_1\vec{e}_1)+\Delta (a_2\vec{e}_2)+\Delta (a_3\vec{e}_3).
\]
Im Kartesischen vereinfacht sich die Definition zu
\[
\Delta\vec{a}=(\Delta a_1)\vec{e}_1+(\Delta a_2)\vec{e}_2+(\Delta a_3)\vec{e}_3,
\]
was einer komponentenweisen Anwendung entspricht.
%wobei die $a_i$ die Komponenten von $\vec{a}$ sind und $\Delta=\sum_{i=1}^3\frac{\partial^2}{\partial x_i^2}$. 
Berechnen Sie nun $\Delta \vec{a}$ in Zylinderkoordinaten und zeigen Sie, dass -- im Gegensatz zu kartesischen Koordinaten --
\[
\Delta a_i\neq (\Delta \vec{a})_i.
\]


\end{atiTask}

% \begin{atiSolution}
% 	\includepdf{solution-krummlinig_iii.pdf}
% \end{atiSolution}