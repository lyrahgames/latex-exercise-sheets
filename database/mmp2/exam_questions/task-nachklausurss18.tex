\documentclass{atistandalonetask}
\usepackage{atistandard}
\begin{document}

\begin{atiTask}[
	title = Weitere Fragen
]
	\providecommand{\D}{\mathrm{d}}

\begin{atiSubtasks}
	\item Geben Sie die Bedingungen an unter denen periodische Funktionen der Periode $L$ durch $\sin \frac{2\pi n x}{L}$ und $\cos \frac{2\pi nx}{L}$ entwickelt werden können.
	\item Beweisen Sie die folgende Relation im Indexkalkül
	\[\curl(\lambda\vec{a})=(\gradient\lambda) \times\vec{a}+\lambda\curl\vec{a}
	\]
	\item Zu welchem Zeitpunkt wird das Vektorfeld 
	\[
	\vec{F}(\vec{r},t)=(x+t\cdot y+z)\vec{i}+(x+y+t\cdot z)\vec{j}+(t\cdot x+y+z)\vec{k}
	\]
	\item Gegeben seien die Funktion $f(t)$ und $g(t)$, sowie ihre Fouriertransformierten $\hat{f}(k)$ und $\hat{g}(k)$. Bestimmen Sie die Fouriertransformierte der Funktion 
	\[
	h(t)=af(t)+bg(t),\quad a,b\in \mathbb{C}.
	\]
	\item Betrachten Sie die \textsc{Gauss}sche Glockenkurve
	\[
	f(x)=Ae^{-Bx^2}
	\]
	und die folgenden Aussagen: auf der linken Seite stehen Veränderungen, die die Funktion $f(x)$ betreffen, auf der rechten Seite jene, die ihre Fouriertransformierte $\tilde{f}(\omega)$ betreffen. Finden Sie \underline{alle} richtigen Zuordnungen.\\
	
	\begin{minipage}{6cm}
	I Die Amplitude wird vergrößert.\\
	II Der Peak wird verbreitert.
	\end{minipage}
	\begin{minipage}{6cm}
	A Die Amplitude vergrößert sich.\\
	B Die Amplitude verkleinert sich.\\
	C Der Peak verbreitert sich.\\
	D Der Peak wird schmaler.
	\end{minipage}\\
		
	\atiNote{Es müssen nicht zwangsläufig alle Aussagen verwendet werden. Eine Begründung wird nicht verlangt.}
	
	%\item \textbf{Zusatz:} Harry und Ron brüten über ihren Arithmantik-Hausaufgaben. \glqq Der Laplace-Operator dieses trimagischen 
	%Feldes zeigt nach oben, oder?\grqq{}, fragt Harry. Ron kaut auf seiner Feder herum: \glqq Bei mir zeigt er nach unten\dots\grqq{} Ohne den Blick von ihrem Pergament zu lösen, ruft Hermine: \glqq Ihr liegt beide falsch.\grqq{} Warum?
\end{atiSubtasks}

\end{atiTask}

\end{document}
