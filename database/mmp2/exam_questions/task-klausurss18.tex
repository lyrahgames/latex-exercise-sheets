\documentclass{atistandalonetask}
\usepackage{atistandard}
\begin{document}

\begin{atiTask}[
	title = Weitere Fragen
]
	\providecommand{\D}{\mathrm{d}}

\begin{atiSubtasks}
	\item Berechnen Sie den Fluss einer Flüssigkeit mit konstanter Geschwindigkeit 
	\[
	\vec{v}=v_{0x}\vec{i}+v_{0y}\vec{j}
	\]
	durch die Kurve $C: x^2+y^2=1$.
	\item Beweisen Sie die folgende Relation im Indexkalkül
	\[\divergence(\vec{a}\times\vec{b})=\vec{b}\curl \vec{a}-\vec{a}\curl\vec{b}
	\]
	\item Berechnen Sie folgende Integrale
	\begin{atiSubequations}
		\item{\integral{1,5}{2,5}{\delta(x+2)}{x}}
		\item{\integral{0}{2\pi}{\sin x\delta(x-\pi/2)}{x}}
	    \item{\integral{-\infty}{+\infty}{\ln x\delta(x-1)}{x}}
	    \item{\integral{0}{\infty}{\frac{e^x}{x^2}\delta(x+1)}{x}}
	    \item{\integral{0}{\infty}{f(x)\delta(ax^2-b)}{x}}
	\end{atiSubequations}
	\item Gegeben seien die Funktion $f(t)$ und ihre Fouriertransformierte $\hat{f}(k)$. Bestimmen Sie die Fouriertransformierte der Funktion $f(x-a)$.
	\item Ordnen Sie die Bilder der Funktionen $f(t)$ ihren Fouriertransformierten zu.
	\item \textbf{Zusatz:} Harry und Ron brüten über ihren Arithmantik-Hausaufgaben. "Der Laplace-Operator" dieses trimagischen Feldes zeigt nach oben, oder?", fragt Harry. Ron kaut auf seiner Feder herum: "Bei mir zeigt er nach unten..." Ohne den Blick von ihrem Pergament zu lösen, ruft Hermine: "Ihr liegt beide falsch." Warum?
\end{atiSubtasks}

\end{atiTask}

\end{document}
