\RequirePackage{import}
\import{../}{exercise_sheet_preamble}
 \geometry{
 a4paper,
 %total={170mm,257mm},
 left=20mm,
 right=20mm,
 top=20mm,
 bottom=20mm
 }
\usepackage{pdfpages}
\begin{document}
\renewcommand{\vec}[1]{\mathbf{#1}}
  \exerciseSheetHead{%
    title = Mathematische Methoden der Physik II,
    sheetType = Übungsserie,
    sheetNumber = 6 - Indexkalkül, % can be omitted
    author = Dr. Agnes Sambale,
    authorMail = agnes.sambale@uni-jena.de,
    submissionDate = {28.05.2018},
    extraInfo = Sommersemester 2018 % can be omitted  }
}

\noindent Alle Aufgaben sind im Indexkalkül zu lösen (sonst gibt es keine Punkte!). \\
Nützliche Relation:
\[\boxed{\varepsilon_{ijk}\varepsilon_{ilm}=\delta_{jl}\delta_{km}-\delta_{jm}\delta_{kl}}\]
			\import{../database/mmp2/vector_analysis_ii/Indexkalkuel/index_i/}{task-index_i.tex}
			\import{../database/mmp2/vector_analysis_ii/Indexkalkuel/index_ii/}{task-index_ii.tex}
			\newpage
			\import{../database/mmp2/vector_analysis_ii/Indexkalkuel/index_iii/}{task-index_iii.tex}
			\import{../database/mmp2/vector_analysis_ii/Indexkalkuel/index_iv/}{task-index_iv.tex}
			\import{../database/mmp2/vector_analysis_ii/Indexkalkuel/index_v/}{task-index_v.tex}
			
		
\end{document}
